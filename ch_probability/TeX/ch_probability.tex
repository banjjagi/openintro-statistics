\chapter[Probability (special topic)]{확률 (특별 주제)}
\label{probability}

\index{확률|(}

확률이 통계학의 근본을 구성한다. 이미 확률의 여러면에 친숙할 수도 있지만, 확률 개념을 형식한 것은 대부분 사람들에게 새로울 수 있다. 이번장에서 대부분 사람들이 이전에 봤던 과정을 사용해서 확률을 친숙한 용어로 소개하려고 한다.

\section{확률 정의하기 (특별 주제)}
\label{basicsOfProbability}


\begin{example}{``주사위(die, 복수형 dice)''는 \resp{1}, \resp{2}, \resp{3}, \resp{4}, \resp{5}, \resp{6} 숫자가 6 면에 세겨진 정육면체다. 주사위를 던졌을 때 1 이 나올 확률은 얼마인가?}\label{probOf1}
만약 주사위가 공정하다면, \resp{1}이 나올 확률은 다른 어떤 숫자가 나올 확률과 같다. 6 개 결과가 나올 수 있기 때문에, 가능성은 6개 중에 1개로, 즉, $1/6$다.
\end{example}

\begin{example}{다음 주사위 던지기에서 \resp{1} 혹은 \resp{2}가 나올 확률은 얼마인가?}\label{probOf1Or2}
\resp{1} 과 \resp{2} 는 6 가지 동일한 가능한 결과 중 둘로, 두 가지 결과 중에 하나를 얻을 가능성은 $2/6 = 1/3$이 된다.
\end{example}

\begin{example}{다음번 주사위를 던져서 \resp{1}, \resp{2}, \resp{3}, \resp{4}, \resp{5}, \resp{6} 중에서 하나가 나올 확률은 얼마인가?}\label{probOf123456}
100\%. 결과는 상기 6개 숫자 중에 하나가 되어야만 된다.
\end{example}

\begin{example}{ \resp{2}가 나오지 않을 가능성은 얼마인가?}\label{probNot2}
\resp{2}가 나올 가능성이 $1/6$ 혹은 $16.\bar{6}\%$이기 때문에, \resp{2}가 나오지 않을 가능성은 $100\% - 16.\bar{6}\%=83.\bar{3}\%$ 혹은 $5/6$가 된다.

다른 방식으로, \resp{2}가 나오지 않는 것은 \resp{1}, \resp{3}, \resp{4}, \resp{5}, \resp{6} 중 하나를 얻는 것과 동일하다는 것을 알아챘을 것이다. 즉, 6개 동일한 가능성을 갖는 결과 중에서 5개로 확률이 $5/6$가 된다.
\end{example}

\begin{example}{주사위를 두개 던진다고 가정하자. 첫번째 주사위가 \resp{1}이 나올 가능성 $1/6^{th}$이고, 두번째 주사위도 \resp{1}이 나올 가능성이 $1/6^{th}$이라면, 모두 \resp{1}이 나올 확률은 얼마인가?}\label{probOf2Ones}
만약 첫번째 주사위가 \resp{1}이 나올 확률이 $16.\bar{6}$\%이고, 두번째 주사위도 \resp{1}이 나올 확률이 $16.\bar{6}$\%이라면, 양쪽 주사위 모두 \resp{1}이 나올 확률은 $(1/6)\times (1/6)$ 혹은 $1/36$이 된다.
\end{example}

\subsection{확률}

\index{확률 과정|(}

명백한 임의성(randomness)을 기술하고 이해하는 도구를 구축하는데 확률을 사용한다. 
\term{결과}(outcome)를 생기게 하는 \term{확률 과정}(random process) 관점으로 확률을 종종 표현한다.

\begin{center}
\begin{tabular}{lll}
주사위 굴리기 &$\rightarrow$ & \resp{1}, \resp{2}, \resp{3}, \resp{4}, \resp{5}, \resp{6} \\
동전 던지기 &$\rightarrow$ & \resp{H} 혹은 \resp{T} \\
\end{tabular}
\end{center}

주사위 굴리기 혹은 동전 던지기는 확률 과정이며 각각은 결과를 생기게 한다.

\begin{termBox}{\tBoxTitle{확률}
결과의 \term{확률}은 만약 무한번 확률과정을 관측했다면 결과가 일어난 횟수의 비율이다.}
\end{termBox}

확률은 비율로 정의되고, 항상 0~과~1 사이 값을 갖는다 (전부 통틀어).또한 0\% 에서 100\% 사이 퍼센티지로 표현될 수도 있다.

확률은 주사위를 많이 굴려서 실증할 수도 있다. $\hat{p}_n$을 $n$번 주사위를 굴려서 \resp{1}이 나온 결과 비율로 놓자. 주사위 굴리기 횟수가 증가함에 따라, $\hat{p}_n$은 \resp{1}을 굴려 나온 확률에 수렴한다, 즉, $p = 1/6$. 그림~\ref{dieProp}은 주사위를 100,000번 굴린 수렴결과가 도시되어 있다. $\hat{p}_n$이 $p$ 주위 안정화되는 경향성을 \term{대수의 법칙}(Law of Large Numbers)으로 기술된다.


\begin{figure}[bt]
\centering
\includegraphics[width=0.85\textwidth]{ch_probability/figures/dieProp/dieProp}
\caption{모의실험에서 각 단계에 1 인 주사위 던지기 분율. 던지기 횟수가 증가함에 따라 비율은 확률 $1/6 \approx 0.167$에 가까워지는 경향이 있다.}
\label{dieProp}
\end{figure}

\begin{termBox}{\tBoxTitle{대수의 법칙}
더 많은 관측점이 수집됨에 따라, 특정 결과를 갖는 출현 비율, $\hat{p}_n$은 해당 결과 $p$에 수렴한다.}
\end{termBox}

경우에 따라서, 그림~\ref{dieProp}에서 $\hat{p}_n$이 여러번 하듯이, 비율이 확률로부터 방향을 바꾸어 벗어나 대수의 법칙을 부정하는 것처럼 보인다. 하지만, 이러한 편차는 던기지 횟수가 증가함에 따라 점점 작아진다.

위에서 굴려서 \resp{1}이 나온 확률을 $p$로 표기했다. 또한 해당 확률을 다음과 같이 작성할 수도 있다.

\begin{eqnarray*}
P(\text{\resp{1} 굴리기})
\end{eqnarray*}
\marginpar[\raggedright\vspace{-13mm}

$P(A)$\vspace{1mm}\\\footnotesize Probability of\\outcome $A$]{\raggedright\vspace{-13mm}

$P(A)$\vspace{1mm}\\\footnotesize 결과 $A$ \\ 확률} 상기 표기에 편안해지면, 더 축약한다. 예를 들어, 만약 과정이 ``주사위 굴리기''가 명확하면, $P($\resp{1} 굴리기$)$ 을~$P($\resp{1}$)$으로 축약한다.

\begin{exercise} \label{randomProcessExercise}
확률과정은 주사위 굴리기와 동전던지기가 포함된다. (a) 또다른 확률과정을 생각해보라. 
(b) 생각해본 과정의 가능한 모든 결과를 기술하라. 예를 들어, 주사위 굴리기는 확률과정으로 \mbox{\resp{1}, \resp{2}, ..., \resp{6}} 사이 가능한 결과가 나온다.\footnote{네가지 예제가 있다. (i) 누군가 다음달에 아플지 아프지 않을지는 명백하게 \resp{아프다} 와 \resp{아프지 않다}라는 결과가 있는 확률과정이다. (ii) 사람을 임의로 골라서 신장을 측정해서 확률과정을 \emph{생성}할 수 있다. 이 과정 결과는 양수다. (iii) 주식시장이 다음주에 오르고 내리는 것은 \resp{상승}, \resp{하락}, \resp{변동 없음} 가능한 결과를 갖는 확률 과정이다. 다른 방법으로 숫자 결과로 주식시장 변동율을 사용할 수 수도 있다. (iv) 방짝이 오늘밤 설겆이를 했는지는 \resp{설겆이\_\hspace{0.3mm}했음} 과 \resp{설겆이\_\hspace{0.3mm}안했음} 가능한 결과를 갖는 확률과정처럼 보인다.}
\end{exercise}

확률과정으로 생각하는 것이 꼭 확률적일 필요는 없다. 하지만, 너무나 어려워서 정확하게 이해될 수는 없다. Guided Practice~\ref{randomProcessExercise}의 네번째 예제 주석 해답으로 방짝 행동은 확률과정이라고 제시했다. 하지만, 방짝 행동이 완전히 확률적이지는 않을지라도, 방짝의 행동을 확률과정으로 모형화하는 것은 여전히 유용하다.

\begin{tipBox}{\tipBoxTitle{과정을 확률로 모형화}
설사 과정이 엄밀히 확률적이는 않지만, 확률로 모형화하는 것이 도움이 될 수 있다.}
\end{tipBox}

\index{확률 과정|)}

\subsection{서로 겹치지 않거나 상호 배반적인 결과}

\index{서로 겹치지 않는|(}
\index{상호 배반적|(}

만약 두 결과가 함께 발생하지 않는다면, 두 결과는 \term{서로 겹치지 않는}(disjoint) 혹은 \term{상호 배반적}(mutually exclusive)이라고 부른다. 예를 들어, 주사위를 던지면, 결과 \resp{1} 와 \resp{2}는 서로 겹치지 않는다라고 하는데 둘다 함께 일어날 수 없기 때문이다. 다른 한편으로, 결과 \resp{1} 과 ``홀수 굴리기는'' 서로 겹친다 왜냐 하면 주사위 굴린 결과 \resp{1} 이 나오면 두 사건이 동시에 일어난 것이기 때문이다. 용어 \term{서로 겹치지 않는}(disjoint), \term{상호 배반적}(mutually exclusive)은 동치이고 서로 교환가능하다. 

서로 겹치지 않는 결과의 확률을 계산하는 것은 쉽다. 주사위를 굴릴 때, \resp{1} 와 \resp{2} 결과는 서로 겹치지 않아서 이들 중 하나가 일어날 확률은 개별 확률을 더해서 계산한다:

\begin{eqnarray*}
P(\text{\resp{1} 혹은 \resp{2}}) = P(\text{\resp{1}})+P(\text{\resp{2}}) = 1/6 + 1/6 = 1/3
\end{eqnarray*}

주사위를 굴려서 \resp{1}, \resp{2}, \resp{3}, \resp{4}, \resp{5}, \resp{6} 이 나올 확률은 얼마인가? 여기서 다시, 모든 결과는 서로 겹치지 않아서 각각의 확률을 더한다:

\begin{eqnarray*}
&&P(\text{\resp{1} 혹은 \resp{2} 혹은 \resp{3} 혹은 \resp{4} 혹은 \resp{5} 혹은 \resp{6}}) \\
	&&\quad= P(\text{\resp{1}})+P(\text{\resp{2}})+P(\text{\resp{3}})+P(\text{\resp{4}})+P(\text{\resp{5}})+P(\text{\resp{6}}) \\
	&&\quad= 1/6 + 1/6 + 1/6 + 1/6 + 1/6 + 1/6 = 1.
\end{eqnarray*}

결과가 서로 겹치지 않을 때, \term{가산 규칙}(Addition Rule)은 이러한 접근법의 정확성을 보증한다.

\begin{termBox}{\tBoxTitle{서로 겹치지 않는 결과의 가산 규칙}
만약 $A_1$ 과 $A_2$가 두 서로 겹치지 않는 결과라면, 이들 결과 중 하나가 발생할 확률은 다음과 같이 주어진다.
\begin{eqnarray*}
P(A_1\text{ 혹은 } A_2) = P(A_1) + P(A_2)
\end{eqnarray*}
만약 서로 겹치지 않는 많은 결과 $A_1$, ..., $A_k$ 이 있다면, 이들 결과 중 하나가 발생할 확률은 다음과 같다.
\begin{eqnarray}
P(A_1) + P(A_2) + \cdots + P(A_k)
\end{eqnarray}
}
\end{termBox}

\begin{exercise}
주사위를 굴려서 \resp{1}, \resp{4}, \resp{5} 이 나올 확률에 관심이 있다. (a) \resp{1}, \resp{4}, \resp{5} 결과가 왜 서로 겹치지 않는지 설명하시요. (b) 가산 규칙을 적용해서 서로 겹치지 않는 결과 $P($\resp{1} or \resp{4} or \resp{5}$)$ 확률을 결정하시요.\footnote{(a) 확률과정은 주사위 굴리기다. 많아야 주사위 결과 중 하나가 나올 수 있다. 이것이 의미하는 것은 주사위 굴린 결과는 서로 겹치지 않는 결과가 된다. (b)~$P($\resp{1} 혹은 \resp{4} 혹은 \resp{5}$) = P($\resp{1}$)+P($\resp{4}$)+P($\resp{5}$) = \frac{1}{6} + \frac{1}{6} + \frac{1}{6} = \frac{3}{6} = \frac{1}{2}$}
\end{exercise}

\index{데이터!email|(}
\begin{exercise}
~\ref{introductionToData} 장 \data{email} 데이터셋에서, \var{number} 변수는 숫자가 없는지 (표식자는 \resp{none}), 단지 작은 하나 혹은 그이상 숫자가 있는지(\resp{small}), 혹은 전자우편에 적어도 하나 큰 숫자(\resp{big})가 나타나는지를 기술한다. 3,921 개 전자우편 중에서 549는 숫자가 없고, 2,827개 전자우편은 단지 작은 하나 혹은 그이상 숫자가 있고, 545는 적어도 하나 큰 숫자가 전자우편에 있다. (a) \resp{none}, \resp{small}, \resp{big} 결과는 서로 겹치지 않는가? (b) 별도로 \resp{small} 와 \resp{big} 값을 갖는 전자우편 비율을 알아내시오. (c) 서로 겹치지 않는 결과에 대해서 가산규칙을 사용해서 데이터셋에서 무작위로 고른 전자우편 내부에 큰던 작던 숫자가를 갖을 확률을 계산하시오.\footnote{
(a) 맞습니다. \var{number} 수준 중 하나로만 각 전자우편이 분류된다. (b) \resp{small}: $\frac{2827}{3921} = 0.721$. \resp{big}: $\frac{545}{3921} = 0.139$. (c) $P($\resp{small} 혹은 \resp{big}$) = P($\resp{small}$) + P($\resp{big}$) = 0.721 + 0.139 = 0.860$.}
\end{exercise}
\index{데이터!email|)}

\index{사건|(}
통계학자는 거의 개별 결과로 작업하지 않고 대신에 결과 \indexthis{\emph{집합}}{집합}(set) or \indexthis{\emph{모임}}{collections}(collections)을 고려한다. $A$가 주사위를 굴렸을 때 \resp{1} 혹은 \resp{2}가 나올 사건을 나타내고, $B$가 주사위 굴렸을 때 \resp{4} 혹은 \resp{6} 이 나올 사건을 나타낸다고 하자. $A$ 를 $\{$\resp{1},~\resp{2}$\}$ 의 결과 집합으로 적고, $B=\{$\resp{4}, \resp{6}$\}$ 와 같이 적는다. 이러한 집합을 일반적으로 \termsub{사건}{사건}(events)으로 불린다. $A$ 와 $B$ 는 공통된 요소가 없기 때문에, 서로 겹치지 않는 사건이다. $A$ 와 $B$ 가 그림~\ref{disjointSets}에 나타나 있다.

\begin{figure}[hhh]
\centering
\includegraphics[width=0.55\textwidth]{ch_probability/figures/disjointSets/disjointSets}
\caption{$A$, $B$, $D$, 세가지 사건은 주사위를 굴려 나온 결과로 구성된다. $A$ 와 $B$는 서로 겹치지 않는데 이유는 어떤 공통된 결과도 없기 때문이다.}
\label{disjointSets}
\end{figure}

가산규칙은 서로 겹치지 않는 결과와 서로 겹치지 않는 사건에 모두 적용된다. 서로 겹치지 않는 사건 $A$ 혹은 $B$ 가 일어날 확률은 개별 확률의 합이다:

\begin{align*}
P(A\text{ 혹은 }B) = P(A) + P(B) = 1/3 + 1/3 = 2/3
\end{align*}

\begin{exercise}
(a) 가산 규칙을 사용해서 사건 $A$의 확률, $P(A)$가 $1/3$임을 입증하시오. (b) 사건 $B$에 대해 동일한 작업을 수행하시오.\footnote{(a) $P(A) = P($\resp{1} 혹은 \resp{2}$) = P($\resp{1}$) + P($\resp{2}$) = \frac{1}{6} + \frac{1}{6} = \frac{2}{6} = \frac{1}{3}$. (b) 마찬가지 방식으로, $P(B) = 1/3$.}
\end{exercise}

\begin{exercise} \label{exerExaminingDisjointSetsABD}
(a) 그림~\ref{disjointSets}을 참조하여, 사건 $D$는 어떤 결과를 나타내는가? (b) 사건 $B$ 와 $D$는 서로 겹치지 않는가? (c) 사건 $A$ 와 $D$는 서로 겹치지 않는가? \footnote{(a)~결과 \resp{2} 와 \resp{3}. (b)~예, 사건 $B$ 와 $D$ 는 서로 겹치지 않는데 이유는 두 사건 모두 어떤 결과도 공유하지 않는다. (c)~사건 $A$ 와 $D$ 는 공통된 결과, \resp{2}를 공유해서, 서로 겹친다.}
\end{exercise}

\begin{exercise}
Guided Practice~\ref{exerExaminingDisjointSetsABD}에서 그림~\ref{disjointSets}으로부터 $B$ 와 $D$가 서로 겹치지 않는다는 것을 확인했다. 사건 $B$ 혹은 $D$ 이 일어날 확률을 계산하시오.\footnote{$B$ 와 $D$ 는 서로 겹치지 않는 사건이기 때문에, 가산 규칙을 사용한다: $P(B$ 혹은 $D) = P(B) + P(D) = \frac{1}{3} + \frac{1}{3} = \frac{2}{3}$.}
\end{exercise}

\index{사건|)}
\index{서로 겹치지 않는|)}
\index{상호 배반적|)}

\subsection{사건이 서로 겹칠 때 확률}

표~\ref{deckOfCards}에 나타낸 \indexthis{52개 정규 카드 한벌}{deck of cards} 맥락에서 서로 겹치는 두 사건을 계산하는 문제를 생각해 보자. 만약 정규 한벌 카드에 친숙하지 않다면, 아래 주석\footnote{카드 52개는 4 \term{묶음}(suite)으로 쪼개진다: $\clubsuit$ (클럽, club), {\color{redcards}$\diamondsuit$} (다이아몬드, diamond), {\color{redcards}$\heartsuit$} (하트, heart), $\spadesuit$ (스페이드, spade). 묶음 각각은 다음 라벨이 붙은 13개 카드로 구성된다: \resp{2}, \resp{3}, ..., \resp{10}, \resp{J} (잭, jack), \resp{Q} (퀸, queen), \resp{K} (킹, king), \resp{A} (에이스, ace). 따라서, 카드 각각은 묶음과 라벨의 유일무이한 조합이 된다, 즉, {\color{redcards}\resp{4$\heartsuit$}}, \resp{J$\clubsuit$}. 잭, 퀸, 킹을 나타내는 카드 12개는 \termsub{\resp{그림카드}}{그림카드}(face card)로 불린다. 일반적으로 {\color{redcards}$\diamondsuit$} 혹은 {\color{redcards}$\heartsuit$} 카드묶음은 {\color{redcards}빨간색}으로 다른 두 묶음 카드는 검은색이다.}을 참조한다.


\begin{table}[h]
\centering
\begin{tabular}{lll lll lll lll l}
\resp{2$\clubsuit$} & \resp{3$\clubsuit$} & \resp{4$\clubsuit$} & \resp{5$\clubsuit$} & \resp{6$\clubsuit$} & \resp{7$\clubsuit$} & \resp{8$\clubsuit$} & \resp{9$\clubsuit$} & \resp{10$\clubsuit$} & \resp{J$\clubsuit$} & \resp{Q$\clubsuit$} & \resp{K$\clubsuit$} & \resp{A$\clubsuit$}  \\
\color{redcards} \resp{2$\diamondsuit$} & \color{redcards}\resp{3$\diamondsuit$} & \color{redcards}\resp{4$\diamondsuit$} & \color{redcards}\resp{5$\diamondsuit$} & \color{redcards}\resp{6$\diamondsuit$} & \color{redcards}\resp{7$\diamondsuit$} & \color{redcards}\resp{8$\diamondsuit$} & \color{redcards}\resp{9$\diamondsuit$} & \color{redcards}\resp{10$\diamondsuit$} & \color{redcards}\resp{J$\diamondsuit$} & \color{redcards}\resp{Q$\diamondsuit$} & \color{redcards}\resp{K$\diamondsuit$} & \color{redcards}\resp{A$\diamondsuit$} \\
\color{redcards}\resp{2$\heartsuit$} & \color{redcards}\resp{3$\heartsuit$} & \color{redcards}\resp{4$\heartsuit$} & \color{redcards}\resp{5$\heartsuit$} & \color{redcards}\resp{6$\heartsuit$} & \color{redcards}\resp{7$\heartsuit$} & \color{redcards}\resp{8$\heartsuit$} & \color{redcards}\resp{9$\heartsuit$} & \color{redcards}\resp{10$\heartsuit$} & \color{redcards}\resp{J$\heartsuit$} & \color{redcards}\resp{Q$\heartsuit$} & \color{redcards}\resp{K$\heartsuit$} & \color{redcards}\resp{A$\heartsuit$} \\
\resp{2$\spadesuit$} & \resp{3$\spadesuit$} & \resp{4$\spadesuit$} & \resp{5$\spadesuit$} & \resp{6$\spadesuit$} & \resp{7$\spadesuit$} & \resp{8$\spadesuit$} & \resp{9$\spadesuit$} & \resp{10$\spadesuit$} & \resp{J$\spadesuit$} & \resp{Q$\spadesuit$} & \resp{K$\spadesuit$} & \resp{A$\spadesuit$}
\end{tabular}
\caption{카드 한벌에 52개 유일무이한 카드를 표현.}
\label{deckOfCards}
\end{table}

\begin{exercise}
(a) 무작위로 선택된 카드가 다이아몬드일 확률은 얼마인가? (b)무작위로 선택된 카드가 그림카드일 확률은 얼마인가?\footnote{(a) 카드가 52개, 다이아몬드가 13개 있다. 만약 카드를 철저히 뒤섞는다면, 카드 각각이 뽑힐 가능성은 동일하다. 그래서 무작위로 선택된 카드가 다이아몬드일 확률은 $P({\color{redcards}\diamondsuit}) = \frac{13}{52} = 0.250$. (b)~마찬가지 방식으로, 그림카드 12개가 있다. 그래서 $P($그림카드$) = \frac{12}{52} = \frac{3}{13} = 0.231$.}
\end{exercise}

둘 혹은 셋 변수, 속성, 확률과정에 대한 결과를 ``포함(in)'' 혹은 ``제외(out)''로 범주화할 때 \term{벤다이어그램}(Venn diagrams)이 유용하다. 그림~\ref{cardsDiamondFaceVenn}에 벤다이어그램은 원을 사용해서 다이아몬드를 나타내고, 또다른 원을 사용해서 그림카드를 나타낸다. 만약 카드가 다이아몬드이고 그림카드라면, 해당 카드는 두 원의 교차지점에 포함된다. 만약 카드가 다이아몬드지만 그림카드가 아니라면, 오른쪽 원이 아닌 왼쪽 원부분이 된다. 다이아몬드인 카드 전체 숫자는 다이아몬드 원의 카드 전체 숫자로 주어진다: $10+3=13$. 또한 확률도 도시되어 있다 (즉, $10/52 = 0.1923$).

\begin{figure}
\centering
\includegraphics[width=0.65\textwidth]{ch_probability/figures/cardsDiamondFaceVenn/cardsDiamondFaceVenn}
\caption{다이아몬드와 그림카드를 위한 벤다이어그램.}
\label{cardsDiamondFaceVenn}
\end{figure}

%\begin{exercise}
%Using Figure~\ref{cardsDiamondFaceVenn}, verify $P($face card$) = 12/52=3/13$.\footnote{The Venn diagram shows face cards split up into ``face card but not {\color{redcards}$\diamondsuit$}'' and ``face card and {\color{redcards}$\diamondsuit$}''. Since these correspond to disjoint events, $P($face card$)$ is found by adding the two corresponding probabilities: $\frac{3}{52} + \frac{9}{52} = \frac{12}{52} = \frac{3}{13}$.}
%\end{exercise}

$A$가 무작위로 선택된 카드가 다이아몬드를 나타내고, $B$는 그림카드 사건을 나타낸다. $P(A$ 혹은 $B)$ 확률을 어떻게 계산하나요? 사건 $A$ 와 $B$ 는 서로 겹친다 -- 카드 {\color{redcards}$J\diamondsuit$}, {\color{redcards}$Q\diamondsuit$}, {\color{redcards}$K\diamondsuit$} 는 양쪽 범주에 포함된다 -- 그래서 서로 겹치지 않는 사건에 대한 가산 법칙을 사용할 수는 없다. 대신에 벤다이어그램을 사용한다. 두 사건에 대한 확률을 더해서 시작한다:

\begin{eqnarray*}
P(A) + P(B) = P({\color{redcards}\diamondsuit}) + P(\text{그림카드}) = 12/52 + 13/52
\label{overCountFaceDiamond}
\end{eqnarray*}
하지만, 양쪽 사건에 해당되는 세 카드가, 확률 각각에 대해서 한번씩, 중복 계수되었다. 이중으로 계수된 것을 수정해야만 된다:
\begin{eqnarray}
P(A\text{ or } B) &=&P(\text{그림카드 혹은 }{\color{redcards}\diamondsuit})  \notag \\
 &=& P(\text{그림카드}) + P({\color{redcards}\diamondsuit}) - P(\text{그림카드 그리고 }{\color{redcards}\diamondsuit}) \label{diamondFace} \\
 &=& 13/52 + 12/52 - 3/52 \notag \\
 &=& 22/52 = 11/26 \notag
\end{eqnarray}

방정식~(\ref{diamondFace})은 \term{일반 가산법칙}(General Addition Rule)의 한 사례다.

\begin{termBox}{\tBoxTitle{일반 가산규칙} 만약 $A$ 와 $B$ 를 서로 겹치든, 겹치지 않든 임의 두 사건이라고 가정하면, 두 사건 중 하나가 일어날 확률은 다음과 같다.
\begin{eqnarray}
P(A\text{ 혹은 }B) = P(A) + P(B) - P(A\text{ 그리고 }B)
\label{generalAdditionRule}
\end{eqnarray}
여기서 $P(A$ 그리고 $B)$ 는 두 사건이 모두 일어날 확률이다.}
\end{termBox}

\begin{tipBox}{\tipBoxTitle{``혹은(or)'' 의 의미는 일체를 포함함.}
통계학에서 ``혹은(or)''을 적을 때, 명시적으로 달리 언급되지 않는다면 ``그리고/혹은 (and/or)'' 의미가 된다. 따라서, $A$ 혹은(or) $B$ 가 일어난다는 의미는 $A$, $B$, 혹은 $A$ 와 $B$ 함께 일어난다는 의미가 된다.}
\end{tipBox}

\begin{exercise}
(a) 만약 $A$ 그리고 $B$ 가 서로 겹치지 않는다면, 왜 이것이 $P(A$ 그리고 $B) = 0$를 암시하는지 기술하시오. (b) (a) 부분을 사용해서, 만약 $A$ 그리고 $B$ 가 서로 겹치지 않는다면 일반가산법칙이 서로 겹치지 않는 사건에 대해서 더 간단한 가산규칙으로 단순화됨을 확증하시오.\footnote{(a) 만약 $A$ 그리고 $B$ 가 서로 겹치지 않는다면, 사건 $A$ 그리고 $B$ 는 결코 동시에 일어날 수 없다. (b) 만약 $A$ 그리고 $B$ 가 서로 겹치지 않는다면, 방정식~(\ref{generalAdditionRule}) 의 마지막 항은 0 이 되고 ((a) 부분 참조), 서로 겹치지 않는 사건에 대한 가산 규칙으로 된다.}
\end{exercise}

\index{데이터!email|(}

\begin{exercise}\label{emailSpamNumberVennExer}
% library(openintro); data(email); table(email[,c("spam", "number")]); table(email[,c("number")]); table(email[,c("spam")])
전자우편 3,921개를 갖는 \data{email} 데이터셋에는 367개 스팸 전자우편이, 전자우편 2,827개는 일부 작은 숫자 하지만 크지 않은 숫자를 갖는 있으며, 전자우편 168개는 두 특성을 모두 지니고 있다. 이런 설정에 대한 벤다이어그램을 생성하시오.\footnote{%
\begin{minipage}[t]{0.65\textwidth}
갯수와 상응하는 {\color{oiB}확률} 이 도시되었다 (즉, $2659/3921 = 0.678$). 왼쪽 원에 나타난 전자우편 숫자는 $2659 + 168 = 2827$에 대응되고, 오른쪽 원에 나타난 숫자는 $168 + 199 = 367$ 이 된다.
\end{minipage}\ %
\begin{minipage}[c]{0.3\textwidth}
\hfill\includegraphics[height=13mm]{ch_probability/figures/emailSpamNumberVenn/emailSpamNumberVenn} \vspace{-13mm}
\end{minipage}
}
\end{exercise}

\begin{exercise}
(a) Guided Practice~\ref{emailSpamNumberVennExer}에 나온 벤다이어그램을 사용해서 \data{email} 데이터셋으로부터 무작위로 뽑힌 전자우편이 스팸이며 적은 숫자(하지만 큰 숫자는 아닌)를 포함할 확률을 알아내시오. (b) 전자우편이 이러한 속성 중 하나를 가질 확률은 얼마인가요?\footnote{
(a)~정답은 두 원의 교차지역으로 나타난다: 0.043. 해답은 원으로 보여진 서로 겹치지 않는 세 확률의 합이다: $0.678 + 0.043 + 0.051 = 0.772$.}
\index{데이터!email|)}
\end{exercise}


\subsection{확률 분포}

\term{확률분포}(probability distribution)는 서로 겹치지 않는 모든 결과와 연관된 확률의 표다. 표~\ref{diceProb}에는 주사위 두개를 굴려나온 결과의 합계에 대한 확률분포가 나와있다.

\begin{table}[h] \small
\centering
\begin{tabular}{l ccc ccc ccc cc}
  \hline
  \ \vspace{-3mm} \\
Dice sum\vspace{0.3mm} & 2 & 3 & 4 & 5 & 6 & 7 & 8 & 9 & 10 & 11 & 12  \\
Probability & $\frac{1}{36}$ & $\frac{2}{36}$ & $\frac{3}{36}$ & $\frac{4}{36}$ & $\frac{5}{36}$ & $\frac{6}{36}$ & $\frac{5}{36}$ & $\frac{4}{36}$ & $\frac{3}{36}$ & $\frac{2}{36}$ & $\frac{1}{36}$\vspace{1mm} \\
   \hline
\end{tabular}
\caption{주사위 두개를 굴려나온 결과의 합계에 대한 확률분포.}
\label{diceProb}
\end{table}

\begin{termBox}{\tBoxTitle{확률분포에 대한 규칙}
확률분포는 다음 세가지 규칙을 만족하는 가능한 결과와 대응되는 확률에 대한 목록이다: \vspace{-2mm}
\begin{enumerate}
\setlength{\itemsep}{0mm}
\item 목록에 나온 결과는 서로 겹치지 않아야 된다.
\item 확률 각각은 0 과 1 사이에 놓여야만 한다.
\item 확률은 합이 1 이다. \vspace{1mm}
\end{enumerate}}
\end{termBox}

\begin{exercise}\label{usHouseholdIncomeDistsExercise}
표~\ref{usHouseholdIncomeDists}는 미국 세가지 가구소득에 대한 분포를 제시한다. 단지 하나만 옳다. 어느 분포가 맞을까? 다른 두개 잘못된 분포는 무엇인가?\footnote{분포 (a)는 합이 1이 아니다. 분포 (b)에 두번째 확률은 음수다. 따라서 (c)가 분포 요건을 만족한다. 세가지 분포 중 하나가 실제 미국 가구소득분포이며, 그것은 (c)가 된다.}
\end{exercise}

\begin{table}
\centering
\begin{tabular}{r | rr rr}
  \hline
소득 구간 (\$1000 달러) & 0-25    & 25-50    & 50-100     & 100+    \\
  \hline
(a)\hspace{0.2mm}	 & 0.18 & 0.39 & 0.33 & 0.16 \\
(b)				 & 0.38 & -0.27 & 0.52 & 0.37 \\
(c)\hspace{0.2mm}	 & 0.28 & 0.27 & 0.29 & 0.16 \\
  \hline
\end{tabular}
\caption{미국 가구소득으로 제시된 분포들 (Guided Practice~\ref{usHouseholdIncomeDistsExercise}).}
\label{usHouseholdIncomeDists}
\end{table}

~\ref{introductionToData} 장은  신속한 요약을 위해서 데이터 도식화의 중요성을 강조했다. 확률분포도 막대그래프로 요약될 수 있다. 예를 들어, 미국 가구소득 분포를 막대그래프를 사용해서 그림~\ref{usHouseholdIncomeDistBar}으로 도식화했다.  주사위 두개를 굴려 나온 합계에 대한 확률분포도 표~\ref{diceProb}로 나와있고, 그림~\ref{diceSumDist}으로 도식화했다. %\footnote{It is also possible to construct a distribution plot when income is not artificially binned into four groups. \emph{Continuous} distributions are considered in Section~\ref{contDist}.}

\begin{figure}
\centering
\includegraphics[width=0.68\textwidth]{ch_probability/figures/usHouseholdIncomeDistBar/usHouseholdIncomeDistBar}
\caption{미국 가구소득의 확률분포.}
\label{usHouseholdIncomeDistBar}
\end{figure}

\begin{figure}
\centering
\includegraphics[width=0.73\textwidth]{ch_probability/figures/diceSumDist/diceSumDist}
\caption{주사위 두개를 던져 나온 합계의 확률분포.}
\label{diceSumDist}
\end{figure}

이러한 막대그래프에서, 막대높이는 결과의 확률을 나타낸다. 만약 결과가 숫자이고 이산형이라면, 주사위를 던져 나온 합계의 경우처럼, 대체로 (시각적으로) 히스토그램을 닮은 막대그래프로 표현하는 것이 편리하다. 각각의 위치에서 막대를 그리는 또다른 예제가 ~\pageref{bookCostDist}쪽 그림~\ref{bookCostDist}에 나와있다.

\subsection{여사건}

주사위를 굴리면 집합 $\{$\resp{1}, \resp{2}, \resp{3}, \resp{4}, \resp{5}, \resp{6}$\}$ 중 값이 하나 나온다. 주사위를 굴릴 때 모든 가능한 결과 집합을 \term{표본공간} ($S$)\marginpar[\raggedright\vspace{-5mm}

$S$\\\footnotesize 표본공간]{\raggedright\vspace{-5mm}

$S$\\\footnotesize 표본공간}\index{S@$S$}이라고 부른다. 
사건이 일어나지 않는 시나리오를 조사하는데 종종 표본 공간을 사용한다.

$D=\{$\resp{2}, \resp{3}$\}$를 주사위를 굴린 결과가 \resp{2} 혹은 \resp{3}인 사건을 나타낸다고 두자. 
그러면 $D$의 \term{여}\marginpar[\raggedright\vspace{0.2mm}

$A^c$\\\footnotesize $A$ 결과의 \\여]{\raggedright\vspace{0.2mm}

$A^c$\\\footnotesize $A$ 결과의 \\여}\index{Ac@$A^c$}는 $D$에 포함되지 않는 표본공간의 모든 결과를 나타내고, $D^c = \{$\resp{1}, \resp{4}, \resp{5}, \resp{6}$\}$로 표시한다. 즉, $D^c$는 $D$에 이미 포함되지 않는 가능한 모든 결과 집합이다. 그림~\ref{complementOfD} 에 $D$, $D^c$, 표본공간 $S$ 사이 관계가 나와있다.

\begin{figure}[hht]
\centering
\includegraphics[width=0.55\textwidth]{ch_probability/figures/complementOfD/complementOfD}
\caption{사건 $D=\{$\resp{2}, \resp{3}$\}$와 여사건 $D^c = \{$\resp{1}, \resp{4}, \resp{5}, \resp{6}$\}$. $S$~는 표본공간을 나타내는데 모든 가능한 사건 집합이다.}
\label{complementOfD}
\end{figure}

\begin{exercise}
(a) 
$P(D^c) = P($ \resp{1}, \resp{4}, \resp{5}, or \resp{6} 굴리기$)$ 을 계산하시요. (b) $P(D) + P(D^c)$ 은 얼마인가?\footnote{(a)~결과는 서로 겹치지 않고, 각각은 확률 $1/6$ 이 되고, 전체 확률은 $4/6=2/3$ 이 된다. (b)~$P(D)=\frac{1}{6} + \frac{1}{6} = 1/3$ 이 된다. $D$ 와 $D^c$ 가 서로 겹치지 않기 때문에, $P(D) + P(D^c) = 1$.}
\end{exercise}

\begin{exercise}
사건 $A=\{$\resp{1}, \resp{2}$\}$ 와 $B=\{$\resp{4}, \resp{6}$\}$ 가 ~\pageref{disjointSets} 쪽 그림~\ref{disjointSets}에 나와있다. (a) $A^c$ 와 $B^c$ 가 무엇을 나타내는지 적으시오. (b) $P(A^c)$ 와 $P(B^c)$ 확률을 계산하시오. (c) $P(A)+P(A^c)$ 와 $P(B)+P(B^c)$ 확률을 계산하시오.\footnote{간략한 해답: (a)~$A^c=\{$\resp{3}, \resp{4}, \resp{5}, \resp{6}$\}$ 와 $B^c=\{$\resp{1}, \resp{2}, \resp{3}, \resp{5}$\}$. (b)~
결과 각각은 서로 겹치지 않음에 주목한다. 개별 결과 확률을 더하면 $P(A^c)=2/3$ 와 $P(B^c)=2/3$ 을 얻게된다. (c)~$A$~와~$A^c$ 은 서로 겹치지 않는다. 그리고, 동일한 것이 $B$~와~$B^c$ 에도 사실이다. 따라서, $P(A) + P(A^c) = 1$ 와 $P(B) + P(B^c) = 1$ 이 된다.}
\end{exercise}

집합의 여사건은 매우 중요한 두가지 속성을 갖도록 만들어진다: 
(i) $A$ 에 존재하지 않는 모든 가능한 결과는 $A^c$ 내부에 있다. (ii) $A$ 와 $A^c$ 는 서로 겹치지 않는다. 속성 (i) 는 다음을 암시한다.

\begin{eqnarray}
P(A\text{ or }A^c) = 1
\label{complementSumTo1}
\end{eqnarray}

즉, 만약 결과가 $A$ 내부에 있지 않는다면, $A^c$ 내부에 나타내야만 된다. 특성 (ii) 을 적용하려면 서로 겹치지 않는 사건에 대한 가산 균칙을 사용한다:

\begin{eqnarray}
P(A\text{ or }A^c) = P(A) + P(A^c)
\label{complementDisjointEquation}
\end{eqnarray}

방정식~(\ref{complementSumTo1}) 와~(\ref{complementDisjointEquation}) 을 결합하면 사건 확률과 여사건 확률 사이에 매우 유용한 관계를 이끌어낸다.

\begin{termBox}{\tBoxTitle{여(Complement)}
$A$ 의 여사건은 $A^c$ 로 표기되고, $A^c$는 $A$ 에 없는 모든 결과를 나타낸다. $A$ 와 $A^c$ 는 수학적으로 다음과 같이 관련된다: \vspace{-2mm}
\begin{eqnarray}\label{complement}
P(A) + P(A^c) = 1, \quad\text{i.e.}\quad P(A) = 1-P(A^c)
\end{eqnarray}\vspace{-6.5mm}}
\end{termBox}

간단한 예로, $A$ 혹은 $A^c$ 을 계산하는 것은 몇 단계로 실행 가능하다. 하지만, 여(complement)를 사용하면 문제 복잡성이 늘어나면 상당한 시간을 절약할 수 있다.

\begin{exercise}

$A$ 가 주사위를 두번 굴려서 합이 \resp{12} 보다 적은 사건을 나타낸다고 하자. (a) 사건 $A^c$ 은 무엇을 나타낼가요? (b) ~\pageref{diceProb} 쪽 표~\ref{diceProb} 에서 $P(A^c)$ 을 알아내세요. (c) $P(A)$ 를 알아내시오.\footnote{(a)~$A$의 여: 합이 \resp{12} 와 같을 때. (b)~$P(A^c) = 1/36$. (c) $P(A^c) = 1/36$, (b) 에서 얻은 여확률과  방정식~(\ref{complement})을 사용: $P($ \resp{12} 보다 적음$) = 1 - P($\resp{12}$) = 1 - 1/36 = 35/36$.}
\end{exercise}

\begin{exercise} 주사위 두개 굴리기와 표~\ref{diceProb}에서 나온 확률을 다시 고려해보자. 다음 확률을 계산하시오: (a) 주사위 합이 6 이 \emph{아닐} 확률. (b) 합이 적어도 \resp{4} 일 확률. 즉, 
$B=\{$\resp{4}, \resp{5}, ..., \resp{12}$\}$ 일 사건이 일어날 확률을 알아낸다. (c) 합이 \resp{10} 보다 크지 않다. 즉, $D=\{$\resp{2}, \resp{3}, ..., \resp{10}$\}$ 일 사건이 일어날 확률을 알아낸다.\footnote{(a)~먼저 $P($\resp{6}$)=5/36$ 을 알아내고 나서, 여(complement)를 사용: $P($not \resp{6}$) = 1 - P($\resp{6}$) = 31/36$.

(b)~먼저 여(complement)를 알아내는데, 훨씬 적은 노력이 든다: $P($\resp{2} 혹은 \resp{3}$)=1/36+2/36=1/12$. 그리고 나서 $P(B) = 1-P(B^c) = 1-1/12 = 11/12$ 을 계산한다.

(c)~이전처럼, 여(complement)를 찾아내는 것이 $P(D)$ 을 알아내는 현명한 방법이다. 먼저 $P(D^c) = P($\resp{11} 혹은 \resp{12}$)=2/36 + 1/36=1/12$ 을 알아내고 나서, $P(D) = 1 - P(D^c) = 11/12$ 을 계산한다.}
\end{exercise}


\subsection{독립(Independence)}
\label{probabilityIndependence}

변수와 관측점이 독립이듯이, 확률과정도 또한 독립일 수 있다. 만약 한 확률과정 결과 정보를 아는 것이 다른 확률과정 결과에 대한 어떠한 정보도 제공하지 않는다면, 두 확률과정은 \term{독립}(independent)이다. 예를 들어, 동전 던지기와 주사위 던지기는 독립된 두 확률과정이다 -- 동전 앞면을 안다는 것이 주사위 굴리기 결과를 확인하는데 도움이 되지 못한다. 다른 한편으로, 주식 가격은 함께 올라가고 내려가서, 독립적이지 않다.

예제~\ref{probOf2Ones}에 독립된 두 확률과정 기본 예제가 나와있다: 주사위 두개 굴리기. 두 주사위 모두 \resp{1}이 나올 확률을 알고자 한다. 주사위 하나는 빨간색이고, 다른 하나는 하얀색으로 가정하자. 만약 빨간 주사위 결과가 \resp{1}이 나오면, 하얀색 주사위 결과에 대해서는 어떤 정보도 제공하지 못한다. 먼저,  예제~\ref{probOf2Ones} (페이지~\pageref{probOf2Ones})에 나온 동일한 문제와 마주한다. 다음 추론엔진을 사용해서 확률을 계산했다: 빨간 주사위가 \resp{1}나올 경우의 수는 $1/6^{th}$이고, 하얀색 주사위가 \resp{1}나올 경우의 수는 $1/6^{th}$이다. 그림~\ref{indepForRollingTwo1s}에 주사위 두개를 굴리는 사례를 도식화했다. 주사위는 서로 독립이기 때문에, 최종 결과값을 도출하는데 상응하는 결과 확률을 곱한다: $(1/6)\times(1/6)=1/36$. 이 사례를 많은 독립 확률과정으로 일반화할 수 있다.  

\begin{figure}[hht]
\centering
\includegraphics[width=0.6\textwidth]{ch_probability/figures/indepForRollingTwo1s/indepForRollingTwo1s}
\caption{첫번째 주사위 굴린 결과가 \resp{1}인 것은 $1/6^{th}$. 두번째 주사위 굴리기도 \resp{1}인 것도 $1/6^{th}$이다.}
\label{indepForRollingTwo1s}
\end{figure}

\textC{\newpage}

\begin{example}{기존 주사위와 다른 파란 주사위가 있다면 어떨까? 주사위 세개를 굴려서 모두 \resp{1}이 나올 확률은 얼마인가? }\label{threeDice}
예제~\ref{probOf2Ones}와 동일한 로직을 적용한다. 하얀색과 빨간색 주사위가 모두 \resp{1}일 경우 확률이 $1/36^{th}$이면, 파란색 주사위도 \resp{1}이 나올 확률도 $1/6^{th}$이다. 그래서 곱한다:

{\begin{align*}
P(\text{하얀색}=\text{\small\resp{1} and } \text{빨간색}=\text{\small\resp{1} and } \text{파란색}=\text{\small\resp{1}}) 
&= P(\text{하얀색}=\text{\small\resp{1}})\times P(\text{빨간색}=\text{\small\resp{1}})\times P(\text{파란색}=\text{\small\resp{1}}) \\
&= (1/6)\times (1/6)\times (1/6)
= 1/216
\end{align*}} \vspace{-7mm}
\end{example}

예제~\ref{threeDice}는 소위 독립 확률과정에 대한 곱셈정리(Multiplication Rule)를 보여주고 있다.

\begin{termBox}{\tBoxTitle{독립 확률과정에 대한 \term{곱셈정리}(Multiplication Rule)}
만약 $A$ 와 $B$가 서로 다른 두 독립 확률과정이라면, $A$ 와 $B$가 모두 발생할 확률은 각각의 확률 곱으로 계산된다: \vspace{-1.5mm}
\begin{eqnarray}\label{eqForIndependentEvents}
P(A \text{ and }B) = P(A) \times  P(B)
\end{eqnarray}

유사하게, 만약 $k$ 개 독립 확률과정에서 나온 $k$개 사건 $A_1$, ..., $A_k$가 있다면, 모든 사건이 일어날 확률은 다음과 같다. \vspace{-1.5mm}
\begin{eqnarray*}
P(A_1) \times  P(A_2)\times  \cdots \times  P(A_k)
\end{eqnarray*}\vspace{-6mm}}
\end{termBox}

\begin{exercise} \label{ex2Handedness}
사람의 약 9\%가 왼손잡이다. 미국 인구집단에서 무작위로 사람을 두명 뽑았다고 가정하자. 표본크기 2는 전체 모집단과 연관해서 매우 작기 때문에, 뽑힌 두 사람이 서로 독립이라고 가정해도 합리적이다. (a)~두 사람 모두 왼손잡이일 확률은 얼마인가? (b)~두 사람 모두 오른손잡이일 확률은 얼마인가?\footnote{(a) 첫번째 사람이 왼손잡이일 확률은 $0.09$이고, 두번째 사람도 동일하다. 독립 확률과정에 대한 곱셈정리를 적용해서 두사람이 모두 왼손잡이일 확률을 계산한다: $0.09\times 0.09 = 0.0081$

(b) 양손잡이인 인구비율이 거의 0으로 가정하는 것도 합리적으로, 오른손잡이 확률은 $P($right-handed$)=1-0.09=0.91$이 된다. (a)와 동일한 추론엔진을 사용해서 두사람이 모두 오른손잡이일 확률은 $0.91\times 0.91 = 0.8281$이 된다.}
\end{exercise}

\textC{\newpage}

\begin{exercise} \label{ex5Handedness}
무작위로 5명을 뽑았다고 가정하자.\footnote{
(a)~오른손잡이(right-handed)와 왼손잡이(left-handed)를 각각 \resp{RH}와 \resp{LH} 줄임말로 표현한다. 각각은 서로 독립이기 때문에, 독립 확률과정에 대한 곱셈정리를 적용한다: 
\begin{align*}
P(\text{5명 모두 \resp{RH}})
&= P(\text{첫번째 = \resp{RH}, 두번째 = \resp{RH}, ..., 다섯번째 = \resp{RH}}) \\
&= P(\text{첫번째 = \resp{RH}})\times P(\text{두번째 = \resp{RH}})\times  \dots \times P(\text{다섯번째 = \resp{RH}}) \\
&= 0.91\times 0.91\times 0.91\times 0.91\times 0.91 = 0.624
\end{align*}

(b)~(a)와 동일한 추론엔진을 사용해서, $0.09\times 0.09\times 0.09\times 0.09\times 0.09 = 0.0000059 이 된다.$

(c)~이 질문에 답하기 위해서 보수, $P($all five are \resp{RH}$)$, 를 사용한다:
\begin{align*}
P(\text{모두 \resp{RH}가 아님})
	= 1 - P(\text{모두 \resp{RH}})
	= 1 - 0.624 = 0.376
\end{align*}} \vspace{-1.5mm}
\begin{enumerate}
\setlength{\itemsep}{0mm}
\item[(a)] 모두 오른손잡이일 확률은 얼마인가?
\item[(b)] 모두 왼손잡이일 확률은 얼마인가?
\item[(c)] 모든 사람이 오른손잡이가 아닐 확률은 얼마인가?
\end{enumerate}
\end{exercise}

변수 \var{handedness}(잘쓰는 손), \var{gender}(성별)이 독립이라고 가정하자. 즉, 누군가의 \var{gender}(성별)을 안다는 것이 \var{handedness}(잘쓰는 손)에 대한 어떤 정보도 제공하지 못하고 반대의 경우도 그렇다. 그러면, 곱셈정리를 사용해서, 무작위로 고른 사람이 오른손잡이고, 동시에 여성일 확률을 계산할 수 있다.\footnote{미국 인구에서 실제 \resp{female}(여성) 비율이 약 50\%다. 그래서 여성 표본확률로 0.5를 사용한다. 하지만, 이 확률은 국가마다 다르다.}:

\begin{eqnarray*}
P(\text{오른손잡이 그리고 여성}) &=& P(\text{오른손잡이} \times  P(\text{여성}) \\
&=& 0.91 \times  0.50 = 0.455
\end{eqnarray*}


\begin{exercise}
무작위로 세명을 뽑았다고 가정하자.\footnote{간략한 정답은 다음과 같다. (a)~다음과 같이 확률표기법으로 적으면 $P($무작위로 뽑힌 사람이 남성이고 오른손잡이$)=0.455$. (b) 0.207. (c) 0.045. (d) 0.0093.} \vspace{-1.5mm}
\begin{enumerate}
\setlength{\itemsep}{0mm}
\item[(a)] 첫번째 사람이 남성이고 오른손잡이일 확률은 얼마인가?
\item[(b)] 첫 두사람이 남성이고 오른손잡이일 확률은 얼마인가?
\item[(c)] 세번째 사람이 여성이고 왼손잡이일 확률은 얼마인가?
\item[(d)] 첫 두사람이 남성이고 오른손잡이고, 세번째 사람이 여성이고 왼손잡이일 확률은 얼마인가?
\end{enumerate}
\end{exercise}

종종 한 결과가 또다른 결과에 관한 유용한 정보를 제공할 수 있는지 궁금하다. 지금 묻는 질문은, 출현된 두 사건은 독립인가? 수식~\eqref{eqForIndependentEvents}을 만족하면 두 사건 $A$ 와 $B$가 독립이라고 한다.

\begin{example}{만약 카드 한벌을 섞어 카드 하나를 뽑는다면, 카드가 하트가 나올 사건은 카드가 에이스가 나올 사건과 독립인가?}
카드가 하트일 확률은 $1/4$이고 카드가 에이스일 확률은 $1/13$이다. 카드가 하트 에이스일 확률은 $1/52$이다. 수식~\ref{eqForIndependentEvents}이 만족되는지 검사하자:

\begin{align*}
P({\color{redcards}\heartsuit})\times P(\text{에이스}) = \frac{1}{4}\times \frac{1}{13} = \frac{1}{52} 
					= P({\color{redcards}\heartsuit}\text{ and 에이스})
\end{align*}
방정식이 만족되기 때문에, 카드가 하트라는 사건과 카드가 에이스라는 사건은 서로 독립 사건이다.
\end{example}

%_________________
\section{조건부 확률 (특별 주제)}
\label{conditionalProbabilitySection}

\index{데이터!family\_college|(}

\data{family\_\hspace{0.3mm}college} 데이터셋은 \var{teen} 과 \var{parents} 변수 두개를 갖는 792 표본 사례로 구성되어 있다. 표~\ref{contTableOfParStCollege}에 데이터셋이 요약되어 있다.\footnote{\oiRedirect{textbook-student_parent_college_2001}{nces.ed.gov/pubs2001/2001126.pdf}에서 실제 모집단 요약정보에 기반한 모의시험 데이터셋} 
변수 \var{teen}은 \resp{college} 대학진학 혹은 \resp{not} 미진학으로, \var{college} 라벨은 고등학교 졸업 후에 십대자녀가 즉시 대학에 진학한 것을 의미한다. 십대자녀의 부모님 중 적어도 한명이 학사학위를 받았으면 변수 \var{parents}는 \resp{degree} 값을 갖는다.

\begin{table}[ht]
\centering
\begin{tabular}{ll rr r rr}
  && \multicolumn{2}{c}{\var{부모}} & \hspace{1cm} &  \\
  \cline{3-4}
	&& \resp{degree} & \resp{not} & 합계  \\
  \cline{2-5}
	& \resp{college}     & 231 & 214 & 445 \\
\raisebox{1.5ex}[0pt]{\var{십대 자녀}}	& \resp{not} \hspace{0.5cm} & 49 & 298 & 347   \\
  \cline{2-5}
	& 합계 & 280 & 512 & 792 \\
\end{tabular}
\caption{\data{family\_\hspace{0.3mm}college} 데이터셋을 요약하는 분할표}
\label{contTableOfParStCollege}
\end{table}
%set.seed(5); n <- 792
%parents <- sample(c("college", "not"), n, replace = TRUE, prob = c(0.35, 0.65))
%p <- ifelse(parents == "college", 0.82, 0.45)
%teen <- ifelse(runif(n) < p, "college", "not")
%table(teen, parents)
%table(teen, parents) / n

\begin{figure}[ht]
\centering
\includegraphics[width=0.7\textwidth]{ch_probability/figures/familyCollegeVenn/familyCollegeVenn}
\caption{\data{family\_\hspace{0.3mm}college} 데이터셋을 상자로 나타낸 벤다이어그램}
\label{familyCollegeVenn}
\end{figure}

\begin{example}{만약 적어도 십대자녀의 부모 중에 한명이 학사학위를 가졌다면, 고등학교 졸업 후에 바로 대학에 십대자녀가 들어갈 확률은 얼마인가?}
데이터를 사용해서 확률을 추정한다. 데이터셋에서 부모 \var{parents}가 학위 \resp{degree}를 갖는 280 사례중에서 변수 \var{teen}이 대학진학 \resp{college} 값을 갖는 사례가 231이다:

\begin{eqnarray*}
P(\text{\var{teen} \resp{college} given \var{parents} \resp{degree}}) = \frac{231}{280} = 0.825
\end{eqnarray*}
\end{example}

\begin{example}{표본에서 십대를 무작위로 뽑았는데 고등학교 졸업후에 곧바로 대학에 진학하지 않았다. 부모 중에 적어도 한명이 학사학위를 갖을 확률은 얼마인가?}\label{collegeProbOfParentsGivenStudentNot}
만약 십대가 대학에 진학하지 않았다면, 두번째 행에 십대 347명 중 한명이고, 적어도 부모 한명이 학사학위를 갖는 경우는 49명이다:
\begin{eqnarray*}
P(\text{\var{parents} \resp{degree} given \var{teen} \resp{not}}) = \frac{49}{347} = 0.141
\end{eqnarray*}
\end{example}

\subsection{주변확률과 결합확률}
\label{marginalAndJointProbabilities}

\index{주변확률|(}
\index{결합확률|(}

\data{family\_\hspace{0.3mm}college} 데이터셋에서 별도로 각 변수에 대한 행합계와 열합계가 표~\ref{contTableOfParStCollege}에 포함되어 있다. 이런 합계는 표본에 대한 \termsub{주변확률}{주변확률}(marginal probability)을 나타내는데 다른 변수에 관계없이 단일 변수에 대한 확률이다. 예를 들어, \var{teen} 십대 변수에 한정된 확률은 주변확률이다: 

\begin{align*}
P(\text{\var{teen} \resp{college}}) = \frac{445}{792} = 0.56
\end{align*}

두개 혹은 그 이상 변수에 대한 결과 확률을 \termsub{결합 \mbox{확률}}{결합확률}(joint probability)이라고 한다:
\begin{align*}
P(\text{\var{teen} \resp{college} and \var{parents} \resp{not}}) = \frac{214}{792} = 0.27
\end{align*}

두가지 표기법이 모두 통용되지만, ``and'' 자리를 콤마로 치환하는 것이 일반적이다. 즉,
\begin{center}
$P(\text{\var{teen} \resp{college}, \var{parents} \resp{not}})$ \\[2mm]
상기 표현식은 다음과 동일하다.\\[2mm]
$P(\text{\var{teen} \resp{college} and \var{parents} \resp{not}})$
\end{center}

\begin{termBox}{\tBoxTitle{주변확률과 결합확률}
만약 확률이 단일 변수에 기반하면,  \emph{\hiddenterm{주변확률}}(marginal probability)이다. 두개 혹은 그이상 변수 혹은 확률과정에 대한 결과 확률을 \emph{\hiddenterm{결합확률}}(joint probability)이라고 부른다.}
\end{termBox}

\data{family\_\hspace{0.3mm}college} 표본에 대한 결합확률을 요약하는데 \term{표비율}(table proportions)을 사용한다. 
Table~\ref{familyCollegeProbTable}


\term{표 비율(table proportions)}을 사용해서 \data{family\_\hspace{0.3mm}college} 표본에 대한 결합확률을 요약한다.
표~\ref{familyCollegeDistribution}에 나타난 비율을 얻으려면, 표 합계 792 로 표~\ref{contTableOfParStCollege}의 각 빈도수를 나누면 표 비율이 계산된다.
변수 \var{parents} 와 \var{teen}의 결합 확률분포가 표~\ref{familyCollegeDistribution}에 나와 있다.

\begin{table}[h]
\centering
\begin{tabular}{l rr r}
  \hline
& \var{parents}: \resp{degree} & \var{parents}: \resp{not} & Total  \\
  \hline
\var{teen}: \resp{college}     & 0.29 & 0.27 & 0.56 \\
\var{teen}: \resp{not} \hspace{0.5cm} & 0.06 & 0.38 & 0.44  \\
   \hline
Total & 0.35 & 0.65 & 1.00 \\
\hline
\end{tabular}
\caption{적어도 부모 중 한명이 학사학위가 있고, 십대 자녀가 대학에 입학했는지 요약하는 확률 표.}
\label{familyCollegeProbTable}
\end{table}

\begin{table}[h]
\centering
\begin{tabular}{l c}
  \hline
Joint outcome & Probability \\
  \hline
\var{parents} \resp{degree} and \var{teen} \resp{college} & 0.29 \\
\var{parents} \resp{degree} and \var{teen} \resp{not} & 0.06 \\
\var{parents} \resp{not} and \var{teen} \resp{college} & 0.27 \\
\var{parents} \resp{not} and \var{teen} \resp{not} & 0.38 \\
   \hline
Total & 1.00 \\
\hline
\end{tabular}
\caption{\data{family\_\hspace{0.3mm}college} 데이터셋에 대한 결합확률분포.}
\label{familyCollegeDistribution}
\end{table}

\begin{exercise}
표~\ref{familyCollegeDistribution}가 확률분포를 나타내는지 입증하라: 사건이 서로 겹치지 않고, 모든 확률값이 음수가 아니고, 확률합이 1이 된다.\footnote{각각의 네가지 결과 조합은 서로 겹치지 않고, 모든 확률은 실제로 음수가 아니고, 확률값을 모두 합하면 $0.29 + 0.06 + 0.27 + 0.38 = 1.00$.}
\end{exercise}

간단한 경우에 결합확률을 사용해서 주변확률을 계산할 수 있다. 예를 들어, 조사에서 십대 자녀가 대학에 진학한 확률은 변수 \var{teen}이 \resp{college} 값을 갖는 결과를 합치면 구할 수 있다:

We can compute marginal probabilities using joint probabilities in simple cases. For example, the probability a random teenager from the study went to college is found by summing the outcomes where \var{teen} takes value \resp{college}:\index{marginal probability|)}\index{joint probability|)}
\begin{align*}
P(\text{\underline{\color{black}\var{teen} \resp{college}}})
&=  P(\text{\var{parents} \resp{degree} and \underline{\color{black}\var{teen} \resp{college}}}) \\
& \quad \quad + P(\text{\var{parents} \resp{not} and \underline{\color{black}\var{teen} \resp{college}}}) \\
&= 0.28 + 0.27 \\
&= 0.56
\end{align*}


\subsection{조건부 확률 정의}

\index{조건부 확률|(}

부모의 학력수준과 십대 자녀의 학력수준 사이에 일부 연결이 있다: 부모의 학사학위가 십대 자녀의 대학입학과 연관되어 있다. 이번 절에서, 확률 추정을 향상시키는데 두 변수 사이에 연관된 정보를 사용하는 방법을 논의한다. 

조사에서 십대자녀가 대학에 입학할 확률은 0.56이다. 만약 십대자녀 부모 중 한명이 학사학위를 가졌는지 안다면 이 확률을 갱신할 수 있을까? 물론 그렇다.
이를 위해서, 부모가 학사학위를 갖고 있는 280개 사례로만 관점을 제한하고서 십대 자녀가 대학에 입학한 분율을 조사한다:

\begin{eqnarray*}
P(\text{\var{teen} \resp{college} given \var{parents} \resp{degree}}) = \frac{231}{280} = 0.825
\end{eqnarray*}

이것을 \term{조건부 확률}(conditional probability)이라고 부르는데 이유는 특정 조건(부모가 학사학위를 소유) 아래 확률을 계산했기 때문이다. 조건부 확률에는 두 부분으로 구성된다. 즉, \term{관심 결과}(outcome of interest)와 \term{조건}(condition)이다. 조건을 우리가 이미 사실로 알고 있는 정보로 간주하는 것이 유익하다. 대체로 이러한 정보는 알려진 결과 혹은 사건으로 기술될 수 있다.

확률 표기한 내부 텍스트를 관심 결과와 조건으로 구분한다:

\begin{eqnarray}
&& P(\text{\var{teen} \resp{college} given \var{parents} \resp{degree}}) \notag \\
&& = P(\text{\var{teen} \resp{college}}\ |\ \text{\var{parents} \resp{degree}}) = \frac{231}{280} = 0.825
\label{probStudentUsedIfParentsUsedInFormalNotation}
\end{eqnarray}
\marginpar[\raggedright\vspace{-10mm}

$P(A | B)$\vspace{1mm}\\\footnotesize $B$가 주어진\\결과 $A$가 나올\\확률]{\raggedright\vspace{-10mm}

$P(A | B)$\vspace{1mm}\\\footnotesize $B$가 주어진\\결과 $A$가 나올\\확률}

수직 막대 ``$|$''를 \emph{주어진}(given)으로 읽는다.


방정식~\eqref{probStudentUsedIfParentsUsedInFormalNotation}에서, 부모중 적어도 한명이 학사학위를 갖는 조건에서 십대자녀가 대학에 입학할 확률을 분수로 계산한다:\vspaceB{-1mm}
\begin{eqnarray}
&& P(\text{\var{teen} \resp{college}}\ |\ \text{\var{parents} \resp{degree}}) \notag \\
&&\quad = \frac{\text{\# \var{teen}변수가 \resp{college}값을 갖고 \var{parents} 변수가 \resp{degree} 값을 갖는 사례}}{\text{\# \var{parents}변수가 \resp{degree}값을 갖는 사례}} \label{ratioOfBothToRatioOfConditionalForParentsAndStudent} \\
&&\quad = \frac{231}{280} = 0.825 \notag
\end{eqnarray}

\var{parents} 변수와 \resp{degree} 값 조건을 만족하는 사례만 고려하고 나서, 십대 자녀가 대학에 진학하는지에 대한 관심 결과를 만족하는 사례의 비율을 계산했다.

종종, 주변확률과 결합확률이 데이터 빈도 대신에 제공된다. 예를 들어, 질병발병률은 빈도 형식보다는 백분율로 일반적으로 작성된다. 빈도 정보를 이용할 수 없을 때 조건부 확률을 계산하고자 한다면, 방정식~(\ref{ratioOfBothToRatioOfConditionalForParentsAndStudent})을 예제로 사용한다.

\var{parents} 변수와  \resp{degree} 값을 만족하는 조건 사례만 고려한다. 이 사례 중에서 조건부 확률은 관심 결과 \var{teen} 변수 \resp{college} 값 를 나타내는 분율이다. 
표~\ref{familyCollegeProbTable}에 정보만 제공된다고 가정하자, 즉, 확률 데이터. 그리고 나서, 만약 표본 1000 명을 뽑았다면, 약 35\% 혹은 $0.35\times 1000 = 350$ 명이 정보 조건(\var{parents} 변수 \resp{degree} 값)을 만족할 것을 예상한다. 마찬가지로, 29\% 혹은 $0.29\times 1000 = 290$ 명이 정보 조건을 만족하고 관심 결과를 나타낸다고 예상할 수 있다. 그리고 나면, 조건부 확률을 다음과 같이 계산할 수 있다.

\begin{align}
&P(\text{\var{teen} \resp{college}}\ |\ \text{\var{parents} \resp{degree}}) \notag \\
	&= \frac{\text{\# (\var{teen} \resp{college} and \var{parents} \resp{degree})}}{\text{\# (\var{parents} \resp{degree})}} \notag \\
	&= \frac{290}{350}
		= \frac{0.29}{0.35}
		= 0.829\quad\text{(반올림 오차로 인해 0.825와 다르다)}
\label{stUserPUsedHypSampSize}
\end{align}

방정식~(\ref{stUserPUsedHypSampSize})에서, 두 확률 0.29 와 0.35 분율을 정확하게 조사했고, 다음과 같이 쓸 수 있다.

\begin{align*}
P(\text{\var{teen} \resp{college} and \var{parents} \resp{degree}})
	\quad\text{and}\quad
	P(\text{\var{parents} \resp{degree}}).
\end{align*}

이러한 확률 분율이 조건부 확률에 대한 일반 공식에 대한 한가지 사례다.

\begin{termBox}{\tBoxTitle{Conditional probability}
조건 $B$가 주어진 관심 결과 $A$의 조건부 확률은 다음과 같이 계산된다:
\begin{align}
P(A | B) = \frac{P(A\text{ and }B)}{P(B)}
\label{condProbEq}
\end{align}}
\end{termBox}

\begin{exercise}\label{familyCollegeProbOfParentsEqualNotGivenTeen}
(a) 다음 문장을 조건부 확률 표기법으로 작성하시오: ``\emph{십대 자녀가 고등학교 졸업 후에 바로 대학에 진학하지 않았는 것을 안다면 부모 중 어느 누구도 학사학위가 없을 확률}''. 이제 조건이 부모가 아니라 십대 자녀가 기준인 것에 주목한다. \\[1mm]
(b)~(a)의 확률을 계산하라. 표~\vref{familyCollegeProbTable}가 도움이 될 수 있다.\footnote{(a) $P(\text{\var{parents} \resp{not}}\ |\ \text{\var{teen} \resp{not}})$. (b)
~조건부 확률에 대한 방정식~(\ref{condProbEq})이 나타내는 것은 먼저 $P(\text{\var{parents} \resp{not} 와 \var{teen} \resp{not}}) = 0.38$ and $P(\text{\var{teen} \resp{not}}) = 0.44$을 찾고나서, 비율이 조건부 확률이 됨을 나타낸다: $0.38 / 0.44 = 0.864$.}
\end{exercise}


\begin{exercise}\label{whyCondProbSumTo1}
(a)~만약 십대 자녀가 대학에 진학하지 않았다면 부모 중에 한명이 학사학위가 있을 확률을 알아내시오. \\[1mm]
(b)~(a)와 Guided Practice~\ref{familyCollegeProbOfParentsEqualNotGivenTeen}에서 나온 정답을 사용해서, \\[1mm]
\textC{{\color{white}.\hspace{5mm}}}$P(\text{\var{parents} \resp{degree}}\ |\ \text{\var{teen} \resp{not}})
\ + \ P(\text{\var{parents} \resp{not}}\ |\ \text{\var{teen} \resp{not}})$을 계산하시오.\\[1mm]
(c)~(b)의 합이 1이라는 것을 직관적으로 설명하시오.\footnote{(a)~확률은 $\frac{P(\text{\var{parents} \resp{degree}, \var{teen} \resp{not}})}{P(\text{\var{teen} \resp{not}})} = \frac{0.06}{0.44} = 0.136 이다.$. (b)~합은~1이다. (c)~십대자녀가 대학에 진학을 하지 않은 조건아래에서 부모는 학사 학위를 갖거나 갖지 못하거나 해야 된다. 확률이 같은 정보 조건이 부여된다면, 여사건은 조건부 확률에도 여전히 동작된다.}
\end{exercise}

\begin{exercise}
학사학위를 갖는 부모와 대학진학하는 십대자녀 사이에 연관이 있음을 데이터가 나타낸다. 이것이 부모 학사학위가 십대자녀의 대학진학에 \emph{인과관계}가 있음을 의미하는가?
\footnote{아니다. 연관은 있지만, 데이터는 관측된 것이다. 두가지 가능한 교락변수(confounding variable)에는 \var{소득}(income) and \var{종교}(region)가 포함된다. 다른 변수를 생각해 낼 수 있는가?}
\index{조건부 확률|)}
\index{데이터!family\_college|)}
\end{exercise}


\subsection{1721년 보스톤 천연두}

\index{데이터!smallpox|(}

\data{smallpox} 데이터셋에는 1721년부터 보스톤 지역에서 천연두에 노출된 6,224 명 표본이 담겨있다.\footnote{Fenner F. 1988. \emph{Smallpox and Its Eradication (History of International Public Health, No. 6)}. Geneva: World Health Organization. ISBN 92-4-156110-6.} 그 당시 의사는 제어된 형태로 질병에 사람을 노출시키는 예방접종(inoculation)이 사망우도를 줄일 수 있다고 믿었다.

두 변수를 갖는 사람이 각 사례를 나타낸다: \var{inoculated}(예방접종) 와 \var{result}(결과). 변수 \var{inoculated}(예방접종)은 두 수준을 갖는다: \resp{yes} 혹은 \resp{no}, 사람이 예방접종을 했는지 혹은 예방접종하지 않았는지를 나타낸다. 변수 \var{result}(결과)는 \resp{lived}(생존) 혹은 \resp{died}(사망) 결과를 나타낸다. 데이터가 표~\ref{smallpoxContingencyTable}와 표~\ref{smallpoxProbabilityTable}에 요약되어 있다.


\begin{table}[h]
\centering
\begin{tabular}{ll rr r}
& & \multicolumn{2}{c}{inoculated} & \\
\cline{3-4}
& & \resp{yes} & \resp{no} & Total  \\
\cline{2-5}
		& \resp{lived}     & 238 & 5136 & 5374 \\
\raisebox{1.5ex}[0pt]{\var{result}} &  \resp{died} \hspace{0.5cm} & 6 & 844 & 850  \\
\cline{2-5}
	& Total & 244 & 5980 & 6224 \\
\end{tabular}
\caption{\data{smallpox} 데이터에 대한 분할표.}
\label{smallpoxContingencyTable}
\end{table}

\begin{table}[h]
\centering
\begin{tabular}{ll rr r}
& & \multicolumn{2}{c}{inoculated} & \\
\cline{3-4}
& & \resp{yes} & \resp{no} & Total  \\
   \cline{2-5}
 & \resp{lived}     & 0.0382 & 0.8252 & 0.8634 \\
\raisebox{1.5ex}[0pt]{\var{result}} & \resp{died} \hspace{0.5cm} & 0.0010 & 0.1356  & 0.1366  \\
   \cline{2-5}
& Total & 0.0392 & 0.9608 & 1.0000 \\
\end{tabular}
\caption{전체 합, 6224로 각 빈도를 나누어서 계산된 \data{smallpox} 데이터에 대한 표 비율. \textC{\vspace{-2mm}}}
\label{smallpoxProbabilityTable}
\end{table}

%\textC{\newpage}

\begin{exercise} \label{probDiedIfNotInoculated}
예방접종을 받지 못한 사람이 천연두로 사망한 확률을 형식을 갖춘 표기법으로 작성하고, \mbox{확률}을 계산하시오.\footnote{$P($\var{결과} = \resp{사망} $|$ \var{예방접종} = \resp{미실시}$) = \frac{P(\text{\var{결과} = \resp{사망} and \var{예방접종} = \resp{미실시}})}{P(\text{\var{예방접종} = \resp{않함}})} = \frac{0.1356}{0.9608} = 0.1411$.}
\end{exercise}

\begin{exercise}
예방접종 받은 사람이 천연두로 사망한 확률을 알아내시오. 이 결과를 Guided Practice~\ref{probDiedIfNotInoculated}의 결과와 어떻게 비교되는가?\footnote{$P($\var{결과} = \resp{사망} $|$ \var{예방접종} = \resp{실시}$) = \frac{P(\text{\var{결과} = \resp{사망} and \var{예방접종} = \resp{실시}})}{P(\text{\var{예방접종} = \resp{실시}})} = \frac{0.0010}{0.0392} = 0.0255$. 예방접종을 받은 사람에 대한 사망률은 40 명중에 1명인 반면에 예방접종을 받은 사람에 대한 사망률은 7명 중 1명이다.}
\end{exercise}

\begin{exercise}\label{SmallpoxInoculationObsExpExercise}
예방접종을 받을지 받지 않을지는 보스톤 주민 본인 선택사항이다. (a) 본 조사는 관측조사인가 실험인가? (b) 데이터에서 어떤 인과관계를 추론할 수 있는가? (c) \resp{생존}(lived) 혹은 \resp{사망}(died)에 영향을 미칠 수 있고 또한 예방접종을 받는데 영향을 줄 수 있는 잠재적 중첩변수(confounding variable)는 무엇인가?\footnote{간략 해답: (a)~관측. (b)~아니요, 관측 조사에서 인과관계를 추론할 수 없다. (c)~최고 최신 의료에 접근성. (c)에 대해서 다른 타당한 해답도 있다.}
\end{exercise}

\subsection{일반곱셈정리}

\ref{probabilityIndependence}~절에서 독립 과정에 대한 곱셈정리를 소개했다. 독립이 아닐 수도 있는 사건에 대한 \term{일반곱셈정리}(General Multiplication Rule)가 다음에 있다.

\begin{termBox}{\tBoxTitle{일반곱셈정리}(General Multiplication Rule)
만약 $A$ 와 $B$가 두 결과 혹은 사건을 나타내면, \vspace{-1.5mm}
\begin{eqnarray*}
P(A\text{ and }B) = P(A | B)\times P(B)
\end{eqnarray*} \vspace{-6.5mm} \par
$A$를 관심 갖는 결과로 $B$를 조건으로 간주하는 것이 좋다.}
\end{termBox}
일반곱셈정리는 \pageref{condProbEq}~페이지 방정식~(\ref{condProbEq})에 나온 조건부 확률에 대한 정의를 단수히 재배열한 것이다.

\begin{example}{ \data{smallpox}(천연두) 데이터셋을 생각해보자. 단지 정보 두 조각만 주어졌다고 가정하자: 96.08\% 주민은 예방접종을 받지 않았고, 예방접종 받지 않은 85.88\% 주민만 생존했다. 주민인 예방접종받지 않고 생존할 확률을 어떻게 계산할 수 있을까?}
일반곱셈정리를 사용해서 해답을 계산하고 나서, 표~\ref{smallpoxProbabilityTable}를 사용해서 해답을 검증한다. 알아내고자 하는 것은 다음과 같다.

\begin{eqnarray*}
P(\text{\var{result} = \resp{lived} and \var{inoculated} = \resp{no}})
\end{eqnarray*}

그리고, 다음 정보가 주어졌다.

\begin{eqnarray*}
P(\text{\var{result} = \resp{lived} }|\text{ \var{inoculated} = \resp{no}})=0.8588 \\
P(\text{\var{inoculated} = \resp{no}})=0.9608
\end{eqnarray*}

예방접종을 받지 않은 주민 96.08\% 중에서, 85.88\% 주민이 생존했다:

\begin{eqnarray*}
P(\text{\var{result} = \resp{lived} and \var{inoculated} = \resp{no}}) = 0.8588\times 0.9608 = 0.8251
\end{eqnarray*}

상기식은 일반곱셈정리에 상응한다. 표~\ref{smallpoxProbabilityTable}에서 (일부 반올림 오차를 갖는) \resp{no} 와 \resp{lived} 교차점에 나타난 확률을 확인할 수 있다.
\end{example}

\begin{exercise}
$P($\var{inoculated} = \resp{yes}$) = 0.0392$ 와 $P($\var{result} = \resp{lived} $|$ \var{inoculated} = \resp{yes}$) = 0.9754$을 사용해서 예방접종하고 생존한 사람의 확률을 알아내시오.\footnote{정답은 0.0382이다. 표~\ref{smallpoxProbabilityTable}에서 검증할 수 있다.}
\end{exercise}

\begin{exercise}
예방접종 받은 97.45\% 사람이 생존했다면, 얼마의 예방접종 받은 사람이 사망해야만 하는가?\footnote{
단지 두가지 가능한 결과가 있다: \resp{생존}(lived) 혹은 \resp{사망}(died). 이것이 의미하는 바는 예방접종 받은 사람중에서 100\% - 97.45\% = 2.55\% 사람이 사망해야만 된다는 것이다.}
\end{exercise}

\begin{termBox}{\tBoxTitle{조건부 확률 합}
$A_1$, ..., $A_k$ 가 변수 혹은 과정에 대한 서로 겹치지 않는 모든 결과를 나타낸다고 두자. 그러면, $B$가 또 다른 변수 혹은 과정에 대한 사건이라면, 다음이 성립한다: \vspace{-1mm}
\begin{eqnarray*}
P(A_1|B)+\cdots+P(A_k|B) = 1
\end{eqnarray*}\vspace{-5.5mm} \par
해당 사건과 여사건이 동일한 정보에 대한 조건일 때, 여사건에 대한 규칙도 성립한다: \vspace{-1.5mm}
\begin{eqnarray*}
P(A | B) = 1 - P(A^c | B)
\end{eqnarray*}}
\end{termBox}

\begin{exercise}
위에서 계산된 확률에 기반해서, 천연두로부터 예방접종이 사망위험을 줄이는데 효과적인 것처럼 보이는가?\footnote{
``예방접종 받은'' 집단과 ``예방접종 받지않은'' 집단에 사망률에 표본차이가 상대적으로 크다. 그래서 \var{inoculated} 와 \var{outcome} 사이에 연관이 있어 보인다. 하지만, Guided Practice~\ref{SmallpoxInoculationObsExpExercise} 해답에서 주목했듯이, 관측조사로 인과관계가 있는지 확신할 수는 없다. (추후 연구를 통해서 예방접종이 사망률을 줄이는데 효과적이라는 것이 밝혀졌다.)}
\end{exercise}

\subsection{조건부 확률에서 독립 생각해보기}

만약 두 사건이 독립이라면, 하나의 결과를 알고 있다는 것이 또다른 것에 대한 정보를 제공하지 말아야 된다. 조건부 확률을 사용해서 수학적으로 참임을 보일 수 있다.

\begin{exercise} \label{condProbOfRollingA1AfterOne1}
$X$ 와 $Y$가 두 주사위를 굴려 나온 결과를 나타낸다고 두자.\footnote{간략 해답: (a) $1/6$. (b) $1/36$. (c)~$\frac{P(Y = \text{ \resp{1} and }X=\text{ \resp{1}})}{P(X=\text{ \resp{1}})} = \frac{1/36}{1/6} = 1/6$. 
(d)~확률은 (c)에서 나온 것과 같다: $P(Y=1)=1/6$. $X$에 관한 지식에 의해 $Y=1$ 확률은 변하지 않는다. $X$ 와 $Y$가 독립임으로 이해가 된다.}
\begin{enumerate}[(a)]
\item 첫번째 주사위, $X$이 \resp{1}이 될 확률은 얼마인가?
\item $X$ 와 $Y$ 모두가 \resp{1}이 될 확률은 얼마인가?
\item 조건부 확률에 대한 공식을 사용해서 $P(Y =$ \resp{1}$\ |\ X = $ \resp{1}$)$을 계산하시요.
\item $P(Y=1)$은 얼마인가? (c)에서 나온 정답과 다른가? 설명해보세요.
\end{enumerate}
\end{exercise}

\textC{\newpage}

Guided Practice~\ref{condProbOfRollingA1AfterOne1}(c)에서 독립 과정에 대해서 곱셈정리를 사용하는데 조건을 주는 정보가 어떠한 영향도 주지 않는 것을 보일 수 있다:
\begin{eqnarray*}
P(Y=\text{\resp{1}}\ |\ X=\text{\resp{1}})
	&=& \frac{P(Y=\text{\resp{1} and }X=\text{\resp{1}})}{P(X=\text{\resp{1}})} \\
	&=& \frac{P(Y=\text{\resp{1}})\times \color{oiGB}P(X=\text{\resp{1}})}{\color{oiGB}P(X=\text{\resp{1}})} \\
	&=& P(Y=\text{\resp{1}}) \\
\end{eqnarray*}

\begin{exercise}
정훈이는 카지노에서 룰렛 테이블을 지켜보고 있다가 마지막 다섯번 결과가 \resp{black}이 나온 것을 알아챘다. \resp{black}이 6번 연속해서 나올 확률은 매우 낮은 것(약 $1/64$)을 알아냈기 때문에 \resp{red}에 돈을 걸었다. 정훈이의 이러한 추론에 대해 잘못된 점은 무엇인가?\footnote{다음번 룰렛이 이전 룰렛과 독립이라는 것을 잊었다. 카지노에서는 이런 관행을 적극 이용한다; 
이상한 낌새를 알아채지 못하는 도박사를 속여서 배당률이 본인에게 유리하게 믿도록 많은 내기 게임에서 나온 마지막 일부 결과를 카지노에서 게시한다. 이것을 \term{도박사의 오류}(gambler's fallacy)라고 부른다.}
\end{exercise}


\subsection{수형도(Tree diagrams)}

\index{데이터!smallpox|)}
\index{수형도|(}

\termsub{수형도}{수형도}(tree diagram)는 데이터 구조 주위로 결과와 확률을 조직화하는 도구다.
두개 혹은 그 이상 과정이 연이어 발생하고 각 과정이 이전 과정에 조건부로 일어날 때 가장 유용하게 사용된다.

\data{smallpox} 데이터가 이런 기술에 적합한다. 모집단을 \var{inoculation} 예방접종 변수로 쪼갠다: \resp{yes} 와 \resp{no}.
쪼개진 것을 따라서, 생존률이 각 집단에 대해 관측된다. 그림~\ref{smallpoxTreeDiagram}에 나타난 \term{수형도}에 이러한 구조가 반영되어 있다. \var{inoculation} 변수 첫번째 분지를 \term{최초} 분지(Primary branch)라고 하고, 다른 분지를 \term{이차} 분지 (secondary branch)라고 부른다. 

\begin{figure}[ht]
\centering
\includegraphics[width=0.93\textwidth]{ch_probability/figures/smallpoxTreeDiagram/smallpoxTreeDiagram}
\caption{\data{smallpox} 데이터셋에 대한 수형도}
\label{smallpoxTreeDiagram}
\end{figure}

그림~\ref{smallpoxTreeDiagram}에서 처럼, 수형도는 주변확률과 조건확률로 주석(annotation)이 달려 있다. 수형도는 천연두 데이터를 \var{inoculation} 예방접종 변수를 각각 주변확률 0.0392 와 0.9608을 갖는 \resp{yes} 와 \resp{no} 집단으로 쪼갠다. 두번째 분지는 첫번째에 조건부로 되어 있어서, 조건부확률을 해당 분지에 할당한다. 예를 들어, 그림~\ref{smallpoxTreeDiagram}의 최상단 분지는 \var{inoculated} = \resp{yes} 예방접종 받은 조건부로 \var{result} = \resp{lived} 생존한 확률인다. 일반적으로 수형도 왼쪽에 오른쪽으로 옮겨오면서 거쳐온 숫자를 곱해서 분지 각각의 끝부분에 결합확률을 계산해낸다. 이런 방식의 결합확률은 일반곱셈정리를 사용해서 계산한다:

\begin{eqnarray*}
&& P(\text{\var{inoculated} = \resp{yes} and \var{result} = \resp{lived}}) \\
	&&\quad = P(\text{\var{inoculated} = \resp{yes}})\times P(\text{\var{result} = \resp{lived}}|\text{\var{inoculated} = \resp{yes}}) \\
	&&\quad = 0.0392\times 0.9754=0.0382
\end{eqnarray*}

\begin{example}{통계학 수업에 중간고사와 기말고사 점수를 생각해보자. 중간고사에서 통계학 수강생 중 13\%만 \resp{A}를 받았다고 가정하자. 중간시험에서 \resp{A}를 받은 학생중 47\%가 기말시험에서 \resp{A} 학점을 받았고, \resp{A} 보다 낮은 학점을 받은 학생중에 11\% 학생이 기말시험에서 \resp{A}를 받았다. 무작위로 기말고사에서 학생을 뽑았고, 이 학생이 \resp{A} 학점을 받은 것을 알아냈다. 이 학생이 중간고사에서 \resp{A} 학점을 받을 확률은 얼마인가?} \label{exerciseForTreeDiagramOfStudentGettingAOnMidtermGivenThatSheGotAOnFinal}

최종 목적지는 $P(\text{\var{midterm} = \resp{A}} | \text{\var{final} = \resp{A}})$ 을 알아내는 것이다. 이 조건부확률을 계산하기 위해서는 다음 확률값이 필요하다:

\begin{eqnarray*}
P(\text{\var{midterm} = \resp{A} and \var{final} = \resp{A}}) \qquad\text{and}\qquad
P(\text{\var{final} = \resp{A}})
\end{eqnarray*}

하지만, 상기 정보는 제공되지 않아서, 즉각적으로 확률을 계산하는 것이 명확하지는 않다. 어떻게 풀어나갈지 확실하지 않기 때문에, 그림~\ref{testTree}에서 보여주듯이, 수형도로 정보를 구조화하는 것이 유용하다. 수형도를 만들어 나갈 때, 주변확률을 갖는 변수가 종종 수형도의 최초분지를 만드는데 사용된다; 이 경우에는 주변확률이 중간고사로 제공되었다. 제공된 조건부 확률에 상응하는 기말고사는 이차 분지로 나타난다.

\begin{figure}[ht]
\centering
\includegraphics[width=0.9\textwidth]{ch_probability/figures/testTree/testTree}
\caption{중간고사 \var{midterm} 과 기말고사 \var{final} 변수를 기술하는 수형도.}
\label{testTree}
\end{figure}

수형도를 만들고서 문제의 확률을 계산한다:

\begin{eqnarray*}
&&P(\text{\var{midterm} = \resp{A} and \var{final} = \resp{A}}) = 0.0611 \\
&&P(\text{\underline{\color{black}\var{final} = \resp{A}}})  \\
&& \quad= P(\text{\var{midterm} = \resp{other} and \underline{\color{black}\var{final} = \resp{A}}}) + P(\text{\var{midterm} = \resp{A} and \underline{\color{black}\var{final} = \resp{A}}}) \\
&& \quad= 0.0957 + 0.0611  = 0.1568
\end{eqnarray*}

주변확률, $P($\var{final} = \resp{A}$)$은 \var{final} = \resp{A}에 상응하는 수형도 우측의 모든 결합확률을 더해서 계산된다.
이제 마지막으로 두 확률 비율을 취해서 계산한다:

\begin{eqnarray*}
P(\text{\var{midterm} = \resp{A}} | \text{\var{final} = \resp{A}}) &=& \frac{P(\text{\var{midterm} = \resp{A} and \var{final} = \resp{A}})}{P(\text{\var{final} = \resp{A}})} \\
&=& \frac{0.0611}{0.1568} = 0.3897
\end{eqnarray*}
중간고사에서 \resp{A} 학점을 받을 확률은 약 0.39다.
\end{example}

\begin{exercise}
통계학 개론 수업 후에, 78\% 학생이 성공적으로 수형도를 만들어 낼 수 있게 되었다. 수형도를 만들 수 있는 학생 중에서 97\% 학생이 통계학 개론 수업을 과락없이 통과(pass)한 반면에 수형도를 만들 수 없는 학생중에서는 57\% 만 통과했다. 
(a) 상기 정보를 사용해서 수형도를 만들어 보세요. (b) 임의로 뽑은 학생이 과락없이 통계학 수업을 이수할 확률은 얼마인가?
(c) 만약 한 학생이 통계학 수업을 이수한 것을 알고 있다면 이 학생이 수형도를 만들어 낼 수 있는 확률을 계산하시오.\footnote{\begin{minipage}[t]{0.47\linewidth}
(a) 수형도는 우측에 나와 있다. 
(b)~수업을 이수한 학생을 나타내는 결합확률 두개를 식별해내고, 둘을 더한다: $P($passed$) = 0.7566+0.1254= 0.8820$. 
(c)~$P($construct tree diagram $|$ passed$) = \frac{0.7566}{0.8820} = 0.8578$. \vspace{15mm} \\\ 
\end{minipage}
\begin{minipage}[c]{0.5\linewidth}
\includegraphics[width=\textwidth]{ch_probability/figures/treeDiagramAndPass/treeDiagramAndPass} \vspace{-25mm}
\end{minipage}}
\end{exercise}


\subsection{베이즈 정리(Bayes' Theorem)}
\label{bayesTheoremSubsection}

\index{베이즈 정리|(}

많은 경우에, 다음 형태로 조건부 확률이 주어진다.
\begin{align*}
P(\text{변수 1에 관한 문장} | \text{ 변수 2에 관한 문장})
\end{align*}
하지만, 정말 알고 있은 것은 반전된 조건부 확률이다.
\begin{align*}
P(\text{변수 2에 관한 문장 } | \text{ 변수 1에 관한 문장})
\end{align*}

수형도를 사용해서 첫번째 조건부 확률이 주어졌을 때 두번째 조건부 확률을 구한다. 하지만, 종종 수형도로 이러한 시나리오를 그릴 수 없는 경우가 있다. 이와 같은 경우에 매우 유용하고 일반적인 공식을 적용할 수 있다: 베이즈 정리.
먼저 조건부 확률을 반전하는 예제를 살펴보는데, 여전히 수형도를 적용할 수 있다.

\begin{example}{


In Canada, about 0.35\% of women over 40 will develop breast cancer in any given year. A common screening test for cancer is the mammogram, but this test is not perfect. In about 11\% of patients with breast cancer, the test gives a \term{false negative}: it indicates a woman does not have breast cancer when she does have breast cancer. Similarly, the test gives a \term{false positive} in 7\% of patients who do not have breast cancer: it indicates these patients have breast cancer when they actually do not.\footnote{The probabilities reported here were obtained using studies reported at \oiRedirect{textbook-breastCancerDotOrg_20090831b}{www.breastcancer.org} and \oiRedirect{textbook-ncbi_nih_breast_cancer}{www.ncbi.nlm.nih.gov/pmc/articles/PMC1173421}.} If we tested a random woman over 40 for breast cancer using a mammogram and the test came back positive -- that is, the test suggested the patient has cancer -- what is the probability that the patient actually has breast cancer?} 

\label{probabilityOfBreastCancerGivenPositiveTestExample}

\begin{figure}[h]
\centering
\includegraphics[width=0.93\textwidth]{ch_probability/figures/BreastCancerTreeDiagram/BreastCancerTreeDiagram}
\caption{Tree diagram for Example~\ref{probabilityOfBreastCancerGivenPositiveTestExample}, computing the probability a random patient who tests positive on a mammogram actually has breast cancer.}
\label{BreastCancerTreeDiagram}
\end{figure}

Notice that we are given sufficient information to quickly compute the probability of testing positive if a woman has breast cancer ($1.00-0.11=0.89$). However, we seek the inverted probability of cancer given a positive test result. (Watch out for the non-intuitive medical language: a~\emph{positive} test result suggests the possible presence of cancer in a mammogram screening.) This inverted probability may be broken into two pieces:
\begin{align*}
P(\text{has BC } | \text{ mammogram$^+$}) = \frac{P(\text{has BC and mammogram$^+$})}{P(\text{mammogram$^+$})}
\end{align*}
where ``has BC'' is an abbreviation for the patient actually having breast cancer and ``mammogram$^+$'' means the mammogram screening was positive. A tree diagram is useful for identifying each probability and is shown in Figure~\ref{BreastCancerTreeDiagram}. The probability the patient has breast cancer and the mammogram is positive is
\begin{align*}
P(\text{has BC and mammogram$^+$}) &= P(\text{mammogram$^+$ } | \text{ has BC})P(\text{has BC}) \\
	&= 0.89\times 0.0035 = 0.00312
\end{align*}
The probability of a positive test result is the sum of the two corresponding scenarios:
\begin{align*}
P(\text{\underline{\color{black}mammogram$^+$}}) &= P(\text{\underline{\color{black}mammogram$^+$} and has BC}) + P(\text{\underline{\color{black}mammogram$^+$} and no BC}) \\
	&= P(\text{has BC})P(\text{mammogram$^+$ } | \text{ has BC}) \\
	&\qquad\qquad	+ P(\text{no BC})P(\text{mammogram$^+$ } | \text{ no BC}) \\
	&= 0.0035\times 0.89 + 0.9965\times 0.07 = 0.07288
\end{align*}
Then if the mammogram screening is positive for a patient, the probability the patient has breast cancer is
\begin{align*}
P(\text{has BC } | \text{ mammogram$^+$})
	&= \frac{P(\text{has BC and mammogram$^+$})}{P(\text{mammogram$^+$})}\\
	&= \frac{0.00312}{0.07288} \approx 0.0428
\end{align*}
That is, even if a patient has a positive mammogram screening, there is still only a~4\%~chance that she has breast cancer.
\end{example}

Example~\ref{probabilityOfBreastCancerGivenPositiveTestExample} highlights why doctors often run more tests regardless of a first positive test result. When a medical condition is rare, a single positive test isn't generally definitive.

Consider again the last equation of Example~\ref{probabilityOfBreastCancerGivenPositiveTestExample}.
Using the tree diagram, we can see that the numerator (the top of the fraction) is equal to the following product:
\begin{align*}
P(\text{has BC and mammogram$^+$}) = P(\text{mammogram$^+$ } | \text{ has BC})P(\text{has BC})
\end{align*}
The denominator -- the probability the screening was positive -- is equal to the sum of probabilities for each positive screening scenario:
\begin{align*}
P(\text{\underline{\color{black}mammogram$^+$}})
	&= P(\text{\underline{\color{black}mammogram$^+$} and no BC})
		+ P(\text{\underline{\color{black}mammogram$^+$} and has BC})
\end{align*}
In the example, each of the probabilities on the right side was broken down into a product of a conditional probability and marginal probability using the tree diagram.
\begin{align*}
P(\text{mammogram$^+$})
	&= P(\text{mammogram$^+$ and no BC}) + P(\text{mammogram$^+$ and has BC}) \\
	&= P(\text{mammogram$^+$ } | \text{ no BC})P(\text{no BC}) \\
			   &\qquad\qquad + P(\text{mammogram$^+$ } | \text{ has BC})P(\text{has BC})
\end{align*}
We can see an application of Bayes' Theorem by substituting the resulting probability expressions into the numerator and denominator of the original conditional probability.
\begin{align*}
& P(\text{has BC } | \text{ mammogram$^+$})  \\
& \qquad= \frac{P(\text{mammogram$^+$ } | \text{ has BC})P(\text{has BC})}
	{P(\text{mammogram$^+$ } | \text{ no BC})P(\text{no BC}) + P(\text{mammogram$^+$ } | \text{ has BC})P(\text{has BC})}
\end{align*}

\begin{termBox}{\tBoxTitle{Bayes' Theorem: inverting probabilities}
Consider the following conditional probability for variable 1 and variable 2:\vspace{-1.5mm}
\begin{align*}
P(\text{outcome $A_1$ of variable 1 } | \text{ outcome $B$ of variable 2})
\end{align*}
Bayes' Theorem states that this conditional probability can be identified as the following fraction:\vspace{-1.5mm}
\begin{align}
\frac{P(B | A_1) P(A_1)}
	{P(B | A_1) P(A_1) + P(B | A_2) P(A_2) + \cdots + P(B | A_k) P(A_k)}
	\label{equationOfBayesTheorem}
\end{align}
where $A_2$, $A_3$, ..., and $A_k$ represent all other possible outcomes of the first variable.}\index{Bayes' Theorem|textbf}
\end{termBox}

Bayes' Theorem is just a generalization of what we have done using tree diagrams. The numerator identifies the probability of getting both $A_1$ and $B$. The denominator is the marginal probability of getting $B$. This bottom component of the fraction appears long and complicated since we have to add up probabilities from all of the different ways to get $B$. We always completed this step when using tree diagrams. However, we usually did it in a separate step so it didn't seem as complex.

To apply Bayes' Theorem correctly, there are two preparatory steps:
\begin{enumerate}
\setlength{\itemsep}{0mm}
\item[(1)] First identify the marginal probabilities of each possible outcome of the first variable: $P(A_1)$, $P(A_2)$, ..., $P(A_k)$.
\item[(2)] Then identify the probability of the outcome $B$, conditioned on each possible scenario for the first variable: $P(B | A_1)$, $P(B | A_2)$, ..., $P(B | A_k)$.
\end{enumerate}
Once each of these probabilities are identified, they can be applied directly within the formula.

\begin{tipBox}{\tipBoxTitle{Only use Bayes' Theorem when tree diagrams are difficult}
Drawing a tree diagram makes it easier to understand how two variables are connected. Use Bayes' Theorem only when there are so many scenarios that drawing a tree diagram would be complex.}
\end{tipBox}

\textC{\newpage}

\begin{exercise} \label{exerciseForParkingLotOnCampusBeingFullAndWhetherOrNotThereIsASportingEvent}
Jose visits campus every Thursday evening. However, some days the parking garage is full, often due to college events. There are academic events on 35\% of evenings, sporting events on 20\% of evenings, and no events on 45\% of evenings. When there is an academic event, the garage fills up about 25\% of the time, and it fills up 70\% of evenings with sporting events. On evenings when there are no events, it only fills up about 5\% of the time. If Jose comes to campus and finds the garage full, what is the probability that there is a sporting event? Use a tree diagram to solve this problem.\footnote{\begin{minipage}[t]{0.47\linewidth}
The tree diagram, with three primary branches, is shown to the right. Next, we identify two probabilities from the tree diagram. (1) The probability that there is a sporting event and the garage is full: 0.14. (2) The probability the garage is full: $0.0875 + 0.14 + 0.0225 = 0.25$. Then the solution is the ratio of these probabilities: $\frac{0.14}{0.25} = 0.56$. If the garage is full, there is a 56\% probability that there is a sporting event. \vspace{0.1mm} \\\ 
\end{minipage}
\begin{minipage}[c]{0.5\linewidth}
\includegraphics[width=\textwidth]{ch_probability/figures/treeDiagramGarage/treeDiagramGarage}\vspace{-30mm}
\end{minipage}}
\end{exercise}

\begin{example}{Here we solve the same problem presented in Guided Practice~\ref{exerciseForParkingLotOnCampusBeingFullAndWhetherOrNotThereIsASportingEvent}, except this time we use Bayes' Theorem.}
The outcome of interest is whether there is a sporting event (call this $A_1$), and the condition is that the lot is full ($B$). Let $A_2$ represent an academic event and $A_3$ represent there being no event on campus. Then the given probabilities can be written as
\begin{align*}
&P(A_1) = 0.2 &&P(A_2) = 0.35 &&P(A_3) = 0.45 \\
&P(B | A_1) = 0.7 &&P(B | A_2) = 0.25 &&P(B | A_3) = 0.05
\end{align*}
Bayes' Theorem can be used to compute the probability of a sporting event ($A_1$) under the condition that the parking lot is full ($B$):
\begin{align*}
P(A_1 | B) &= \frac{P(B | A_1) P(A_1)}{P(B | A_1) P(A_1) + P(B | A_2) P(A_2) + P(B | A_3) P(A_3)} \\
		&= \frac{(0.7)(0.2)}{(0.7)(0.2) + (0.25)(0.35) + (0.05)(0.45)} \\
		&= 0.56 
\end{align*}
Based on the information that the garage is full, there is a 56\% probability that a sporting event is being held on campus that evening.
\end{example}

\begin{exercise} \label{exerciseForParkingLotOnCampusBeingFullAndWhetherOrNotThereIsAnAcademicEvent}
Use the information in the previous exercise and example to verify the probability that there is an academic event conditioned on the parking lot being full is 0.35.\footnote{Short answer:
\begin{align*}
P(A_2 | B) &= \frac{P(B | A_2) P(A_2)}{P(B | A_1) P(A_1) + P(B | A_2) P(A_2) + P(B | A_3) P(A_3)} \\
		&= \frac{(0.25)(0.35)}{(0.7)(0.2) + (0.25)(0.35) + (0.05)(0.45)} \\
		&= 0.35
\end{align*}}
\end{exercise}

\begin{exercise} \label{exerciseForParkingLotOnCampusBeingFullAndWhetherOrNotThereIsNoEvent}
In Guided Practice~\ref{exerciseForParkingLotOnCampusBeingFullAndWhetherOrNotThereIsASportingEvent} and~\ref{exerciseForParkingLotOnCampusBeingFullAndWhetherOrNotThereIsAnAcademicEvent}, you found that if the parking lot is full, the probability there is a sporting event is 0.56 and the probability there is an academic event is 0.35. Using this information, compute $P($no event $|$ the lot is full$)$.\footnote{Each probability is conditioned on the same information that the garage is full, so the complement may be used: $1.00 - 0.56 - 0.35 = 0.09$.}
\end{exercise}

The last several exercises offered a way to update our belief about whether there is a sporting event, academic event, or no event going on at the school based on the information that the parking lot was full. This strategy of \emph{updating beliefs} using Bayes' Theorem is actually the foundation of an entire section of statistics called \term{Bayesian statistics}. While Bayesian statistics is very important and useful, we will not have time to cover much more of it in this book.

\index{Bayes' Theorem|)}
\index{tree diagram|)}
\index{conditional probability|)}
\index{probability|)}



%_________________
\section{Sampling from a small population (special topic)}
\label{smallPop}

\begin{example}{Professors sometimes select a student at random to answer a question. If each student has an equal chance of being selected and there are 15 people in your class, what is the chance that she will pick you for the next question?}
If there are 15 people to ask and none are skipping class, then the probability is $1/15$, or about $0.067$.
\end{example}

\begin{example}{If the professor asks 3 questions, what is the probability that you will not be selected? Assume that she will not pick the same person twice in a given lecture.}\label{3woRep}
For the first question, she will pick someone else with probability $14/15$. When she asks the second question, she only has 14 people who have not yet been asked. Thus, if you were not picked on the first question, the probability you are again not picked is $13/14$. Similarly, the probability you are again not picked on the third question is $12/13$, and the probability of not being picked for any of the three questions is
\begin{eqnarray*}
&&P(\text{not picked in 3 questions}) \\
&&\quad = P(\text{\var{Q1}} = \text{\resp{not\_\hspace{0.3mm}picked}, }\text{\var{Q2}} = \text{\resp{not\_\hspace{0.3mm}picked}, }\text{\var{Q3}} = \text{\resp{not\_\hspace{0.3mm}picked}.}) \\
&&\quad = \frac{14}{15}\times\frac{13}{14}\times\frac{12}{13} = \frac{12}{15} = 0.80
\end{eqnarray*}
\end{example}

\begin{exercise}
What rule permitted us to multiply the probabilities in Example~\ref{3woRep}?\footnote{The three probabilities we computed were actually one marginal probability, $P($\var{Q1}$ = $\resp{not\_\hspace{0.3mm}picked}$)$, and two conditional probabilities:
\begin{eqnarray*}
&&P(\text{\var{Q2}} =  \text{\resp{not\_\hspace{0.3mm}picked} }|\text{ \var{Q1}} = \text{\resp{not\_\hspace{0.3mm}picked}}) \\
&&P(\text{\var{Q3}} =  \text{\resp{not\_\hspace{0.3mm}picked} }|\text{ \var{Q1}} = \text{\resp{not\_\hspace{0.3mm}picked}, }\text{\var{Q2}} = \text{\resp{not\_\hspace{0.3mm}picked}})
\end{eqnarray*}
Using the General Multiplication Rule, the product of these three probabilities is the probability of not being picked in 3 questions.}
\end{exercise}

\textC{\newpage}

\begin{example}{Suppose the professor randomly picks without regard to who she already selected, i.e. students can be picked more than once. What is the probability that you will not be picked for any of the three questions?}\label{3wRep}
Each pick is independent, and the probability of not being picked for any individual question is $14/15$. Thus, we can use the Multiplication Rule for independent processes.
\begin{eqnarray*}
&&P(\text{not picked in 3 questions}) \\
&&\quad = P(\text{\var{Q1}} = \text{\resp{not\_\hspace{0.3mm}picked}, }\text{\var{Q2}} = \text{\resp{not\_\hspace{0.3mm}picked}, }\text{\var{Q3}} = \text{\resp{not\_\hspace{0.3mm}picked}.}) \\
&&\quad = \frac{14}{15}\times\frac{14}{15}\times\frac{14}{15} = 0.813
\end{eqnarray*}
You have a slightly higher chance of not being picked compared to when she picked a new person for each question. However, you now may be picked more than once.
\end{example}

\begin{exercise}
Under the setup of Example~\ref{3wRep}, what is the probability of being picked to answer all three questions?\footnote{$P($being picked to answer all three questions$) = \left(\frac{1}{15}\right)^3 = 0.00030$.}
\end{exercise}

If we sample from a small population \term{without replacement}, we no longer have independence between our observations. In Example~\ref{3woRep}, the probability of not being picked for the second question was conditioned on the event that you were not picked for the first question. In Example~\ref{3wRep}, the professor sampled her students \term{with replacement}: she repeatedly sampled the entire class without regard to who she already picked. 

\begin{exercise} \label{raffleOf30TicketsWWOReplacement}
Your department is holding a raffle. They sell 30 tickets and offer seven prizes. (a) They place the tickets in a hat and draw one for each prize. The tickets are sampled without replacement, i.e. the selected tickets are not placed back in the hat. What is the probability of winning a prize if you buy one ticket? (b)~What if the tickets are sampled with replacement?\footnote{(a) First determine the probability of not winning. The tickets are sampled without replacement, which means the probability you do not win on the first draw is $29/30$, $28/29$ for the second, ..., and $23/24$ for the seventh. The probability you win no prize is the product of these separate probabilities: $23/30$. That is, the probability of winning a prize is $1 - 23/30 = 7/30 = 0.233$. (b)~When the tickets are sampled with replacement, there are seven independent draws. Again we first find the probability of not winning a prize: $(29/30)^7 = 0.789$. Thus, the probability of winning (at least) one prize when drawing with replacement is 0.211.}
\end{exercise}

\begin{exercise} \label{followUpToRaffleOf30TicketsWWOReplacement}
Compare your answers in Guided Practice~\ref{raffleOf30TicketsWWOReplacement}. How much influence does the sampling method have on your chances of winning a prize?\footnote{There is about a 10\% larger chance of winning a prize when using sampling without replacement. However, at most one prize may be won under this sampling procedure.}
\end{exercise}

Had we repeated Guided Practice~\ref{raffleOf30TicketsWWOReplacement} with 300 tickets instead of 30, we would have found something interesting: the results would be nearly identical. The probability would be 0.0233 without replacement and 0.0231 with replacement. When the sample size is only a small fraction of the population (under 10\%), observations are nearly independent even when sampling without replacement.



\textC{\newpage}



%_________________
\section{Random variables (special topic)}
\label{randomVariablesSection}

\index{random variable|(}

\begin{example}{Two books are assigned for a statistics class: a textbook and its corresponding study guide. The university bookstore determined 20\% of enrolled students do not buy either book, 55\% buy the textbook only, and 25\% buy both books, and these percentages are relatively constant from one term to another. If~there are 100 students enrolled, how many books should the bookstore expect to sell to this class?}\label{bookStoreSales}
Around 20 students will not buy either book (0 books total), about 55 will buy one book (55 books total), and approximately 25 will buy two books (totaling 50 books for these 25 students). The bookstore should expect to sell about 105 books for this class.
\end{example}

\begin{exercise}
Would you be surprised if the bookstore sold slightly more or less than 105 books?\footnote{If they sell a little more or a little less, this should not be a surprise. Hopefully Chapter~\ref{introductionToData} helped make clear that there is natural variability in observed data. For example, if we would flip a coin 100 times, it will not usually come up heads exactly half the time, but it will probably be close.}
\end{exercise}

\begin{example}{The textbook costs \$137 and the study guide \$33. How much revenue should the bookstore expect from this class of 100 students?}\label{bookStoreRev}
About 55 students will just buy a textbook, providing revenue of
\begin{eqnarray*}
\$137 \times  55 = \$7,535
\end{eqnarray*}
The roughly 25 students who buy both the textbook and the study guide would pay a total of
\begin{eqnarray*}
(\$137 + \$33) \times  25 = \$170 \times  25 = \$4,250
\end{eqnarray*}
Thus, the bookstore should expect to generate about $\$7,535 + \$4,250 = \$11,785$ from these 100 students for this one class. However, there might be some \emph{sampling variability} so the actual amount may differ by a little bit.
\end{example}

\begin{figure}[h]
\centering
\includegraphics[width=0.65\textwidth]{ch_probability/figures/bookCostDist/bookCostDist}
\caption{Probability distribution for the bookstore's revenue from a single student. The distribution balances on a triangle representing the average revenue per student.}
\label{bookCostDist}
\end{figure}

\begin{example}{What is the average revenue per student for this course?}\label{revFromStudent}
The expected total revenue is \$11,785, and there are 100 students. Therefore the expected revenue per student is $\$11,785/100 =  \$117.85$.
\end{example}

\subsection{Expectation}

\index{expectation|(}

We call a variable or process with a numerical outcome a \term{random variable}, and we usually represent this random variable with a capital letter such as $X$, $Y$, or $Z$. The amount of money a single student will spend on her statistics books is a random variable, and we represent it by $X$.

\begin{termBox}{\tBoxTitle{Random variable}
A random process or variable with a numerical outcome.}
\end{termBox}

The possible outcomes of $X$ are labeled with a corresponding lower case letter $x$ and subscripts. For example, we write $x_1=\$0$, $x_2=\$137$, and $x_3=\$170$, which occur with probabilities $0.20$, $0.55$, and $0.25$. The distribution of $X$ is summarized in Figure~\ref{bookCostDist} and Table~\ref{statSpendDist}.

\begin{table}[h]
\centering
\begin{tabular}{l ccc r}
\hline
$i$	  & 1 & 2 & 3  & Total\\
\hline
$x_i$ & \$0 & \$137 & \$170 & --\\
$P(X=x_i)$ & 0.20 & 0.55 & 0.25 & 1.00 \\
\hline
\end{tabular}
\caption{The probability distribution for the random variable $X$, representing the bookstore's revenue from a single student.}
\label{statSpendDist}
\end{table}

We computed the average outcome of $X$ as \$117.85 in Example~\ref{revFromStudent}. We call this average the \term{expected value} of $X$, denoted by $E(X)$\index{EX@$E(X)$}\marginpar[\raggedright\vspace{-3mm}

$E(X)$\vspace{1mm}\\\footnotesize Expected\\value of $X$]{\raggedright\vspace{-3mm}

$E(X)$\vspace{1mm}\\\footnotesize Expected\\value of $X$}. The expected value of a random variable is computed by adding each outcome weighted by its probability:
\begin{align*}
E(X) &= 0 \times  P(X=0) + 137 \times  P(X=137) + 170 \times  P(X=170) \\
	&= 0 \times  0.20 + 137 \times  0.55 + 170 \times  0.25 = 117.85
\end{align*}

\begin{termBox}{\tBoxTitle{Expected value of a Discrete Random Variable}
If $X$ takes outcomes $x_1$, ..., $x_k$ with probabilities $P(X=x_1)$, ..., $P(X=x_k)$, the expected value of $X$ is the sum of each outcome multiplied by its corresponding probability:
\begin{align}
E(X) 	&= x_1\times P(X=x_1) + \cdots + x_k\times P(X=x_k) \notag \\
	&= \sum_{i=1}^{k}x_iP(X=x_i)
\end{align}
The Greek letter $\mu$\index{Greek!mu ($\mu$)} may be used in place of the notation $E(X)$.}
\end{termBox}

The expected value for a random variable represents the average outcome. For example, $E(X)=117.85$ represents the average amount the bookstore expects to make from a single student, which we could also write as $\mu=117.85$.

It is also possible to compute the expected value of a continuous random variable (see Section~\ref{contDist}). However, it requires a little calculus and we save it for a later class.\footnote{$\mu = \int xf(x)dx$ where $f(x)$ represents a function for the density curve.}

In physics, the expectation holds the same meaning as the center of gravity. The distribution can be represented by a series of weights at each outcome, and the mean represents the balancing point. This is represented in Figures~\ref{bookCostDist} and~\ref{bookWts}. The idea of a center of gravity also expands to continuous probability distributions. Figure~\ref{contBalance} shows a continuous probability distribution balanced atop a wedge placed at the mean.

\begin{figure}
\centering
\includegraphics[width=0.72\textwidth]{ch_probability/figures/bookWts/bookWts}
\caption{A weight system representing the probability distribution for $X$. The string holds the distribution at the mean to keep the system balanced.}
\label{bookWts}
\end{figure}

\begin{figure}
\centering
\includegraphics[width=0.68\textwidth]{ch_probability/figures/contBalance/contBalance}
\caption{A continuous distribution can also be balanced at its mean.}
\label{contBalance}
\end{figure}

\index{expectation|)}


\subsection{Variability in random variables}

Suppose you ran the university bookstore. Besides how much revenue you expect to generate, you might also want to know the volatility (variability) in your revenue. 

The \indexthis{variance}{variance} and \indexthis{standard deviation}{standard deviation} can be used to describe the variability of a random variable. Section~\ref{variability}
introduced a method for finding the variance and standard deviation for a data set. We first computed deviations from the mean ($x_i - \mu$), squared those deviations, and took an average to get the variance. In the case of a random variable, we again compute squared deviations. However, we take their sum weighted by their corresponding probabilities, just like we did for the expectation. This weighted sum of squared deviations equals the variance, and we calculate the standard deviation by taking the square root of the variance, just as we did in Section~\ref{variability}.

\begin{termBox}{\tBoxTitle{General variance formula}
If $X$ takes outcomes $x_1$, ..., $x_k$ with probabilities $P(X=x_1)$, ..., $P(X=x_k)$ and expected value $\mu=E(X)$, then the variance of $X$, denoted by $Var(X)$ or the symbol $\sigma^2$, is
\begin{align}
\sigma^2 &= (x_1-\mu)^2\times P(X=x_1) + \cdots \notag \\
	& \qquad\quad\cdots+ (x_k-\mu)^2\times P(X=x_k) \notag \\
	&= \sum_{j=1}^{k} (x_j - \mu)^2 P(X=x_j)
\end{align}
The standard deviation of $X$, labeled $\sigma$\index{Greek!sigma ($\sigma$)}, is the square root of the variance.}
\end{termBox}
\marginpar[\raggedright\vspace{-47mm}

$Var(X)$\vspace{1mm}\\\footnotesize Variance\\of $X$]{\raggedright\vspace{-47mm}

$Var(X)$\vspace{1mm}\\\footnotesize Variance\\of $X$}

\begin{example}{Compute the expected value, variance, and standard deviation of $X$, the revenue of a single statistics student for the bookstore.}
It is useful to construct a table that holds computations for each outcome separately, then add up the results.
\begin{center}
\begin{tabular}{l rrr r}
\hline
$i$ & 1 & 2 & 3 & Total \\
\hline
$x_i$ & \$0 & \$137 & \$170 &  \\
$P(X=x_i)$ & 0.20 & 0.55 & 0.25 &  \\
$x_i \times  P(X=x_i)$ & 0 & 75.35 & 42.50 & 117.85 \\
\hline
\end{tabular}
\end{center}
Thus, the expected value is $\mu=117.85$, which we computed earlier. The variance can be constructed by extending this table:
\begin{center}
\begin{tabular}{l rrr r}
\hline
$i$ & 1 & 2 & 3 & Total \\
\hline
$x_i$ & \$0 & \$137 & \$170 &  \\
$P(X=x_i)$ & 0.20 & 0.55 & 0.25 &  \\
$x_i \times  P(X=x_i)$ & 0 & 75.35 & 42.50 & 117.85 \\
$x_i - \mu$ & -117.85 & 19.15 & 52.15 &  \\
$(x_i-\mu)^2$ & 13888.62 &  366.72 & 2719.62 &  \\
$(x_i-\mu)^2\times P(X=x_i)$ & 2777.7 & 201.7 & 679.9 & 3659.3 \\
\hline
\end{tabular}
\end{center}
The variance of $X$ is $\sigma^2 = 3659.3$, which means the standard deviation is $\sigma = \sqrt{3659.3} = \$60.49$.
\end{example}

\begin{exercise}
The bookstore also offers a chemistry textbook for \$159 and a book supplement for \$41. From past experience, they know about 25\% of chemistry students just buy the textbook while 60\% buy both the textbook and supplement.\footnote{(a) 100\% - 25\% - 60\% = 15\% of students do not buy any books for the class. Part~(b) is represented by the first two lines in the table below. The expectation for part~(c) is given as the total on the line $y_i\times P(Y=y_i)$. The result of part~(d) is the square-root of the variance listed on in the total on the last line: $\sigma = \sqrt{Var(Y)} = \$69.28$.
\begin{center}
\begin{tabular}{rrrrr}
  \hline
$i$ (scenario) & 1 (\resp{noBook}) & 2 (\resp{textbook}) & 3 (\resp{both}) & Total \\
  \hline
$y_i$ & 0.00 & 159.00 & 200.00 &  \\
$P(Y=y_i)$ & 0.15 & 0.25 & 0.60 & \\
$y_i\times P(Y=y_i)$ & 0.00 & 39.75 & 120.00 & $E(Y) = 159.75$\\
$y_i-E(Y)$ & -159.75 & -0.75 & 40.25 & \\
$(y_i-E(Y))^2$ & 25520.06 & 0.56 & 1620.06 & \\
$(y_i-E(Y))^2\times P(Y)$ & 3828.0 & 0.1 & 972.0 & $Var(Y) \approx 4800$ \\
   \hline
\end{tabular}
\end{center}}
\begin{enumerate}
\item[(a)] What proportion of students don't buy either book? Assume no students buy the supplement without the textbook.
\item[(b)] Let $Y$ represent the revenue from a single student. Write out the probability distribution of $Y$, i.e. a table for each outcome and its associated probability.
\item[(c)] Compute the expected revenue from a single chemistry student. 
\item[(d)] Find the standard deviation to describe the variability associated with the revenue from a single student.
\end{enumerate}
\end{exercise}

\subsection{Linear combinations of random variables}

So far, we have thought of each variable as being a complete story in and of itself. Sometimes it is more appropriate to use a combination of variables. For instance, the amount of time a person spends commuting to work each week can be broken down into several daily commutes. Similarly, the total gain or loss in a stock portfolio is the sum of the gains and losses in its components.

\begin{example}{John travels to work five days a week. We will use $X_1$ to represent his travel time on Monday, $X_2$ to represent his travel time on Tuesday, and so on. Write an equation using $X_1$, ..., $X_5$ that represents his travel time for the week, denoted by $W$.}
His total weekly travel time is the sum of the five daily values:
$$ W = X_1 + X_2 + X_3 + X_4 + X_5 $$
Breaking the weekly travel time $W$ into pieces provides a framework for understanding each source of randomness and is useful for modeling $W$.
\end{example}

\begin{example}{It takes John an average of 18 minutes each day to commute to work. What would you expect his average commute time to be for the week?}
We were told that the average (i.e. expected value) of the commute time is 18 minutes per day: $E(X_i) = 18$. To get the expected time for the sum of the five days, we can add up the expected time for each individual day:
\begin{align*}
E(W) &= E(X_1 + X_2 + X_3 + X_4 + X_5) \\
	&= E(X_1) + E(X_2) + E(X_3) + E(X_4) + E(X_5) \\
	&= 18 + 18 + 18 + 18 + 18 = 90\text{ minutes}
\end{align*}
The expectation of the total time is equal to the sum of the expected individual times. More generally, the expectation of a sum of random variables is always the sum of the expectation for each random variable.
\end{example}

\begin{exercise} \label{elenaIsSellingATVAndBuyingAToasterOvenAtAnAuction}
Elena is selling a TV at a cash auction and also intends to buy a toaster oven in the auction. If $X$ represents the profit for selling the TV and $Y$ represents the cost of the toaster oven, write an equation that represents the net change in Elena's cash.\footnote{She will make $X$ dollars on the TV but spend $Y$ dollars on the toaster oven: $X-Y$.}
\end{exercise}

\begin{exercise}
Based on past auctions, Elena figures she should expect to make about \$175 on the TV and pay about \$23 for the toaster oven. In total, how much should she expect to make or spend?\footnote{$E(X-Y) = E(X) - E(Y) = 175 - 23 = \$152$. She should expect to make about \$152.}
\end{exercise}

\begin{exercise} \label{explainWhyThereIsUncertaintyInTheSum}
Would you be surprised if John's weekly commute wasn't exactly 90 minutes or if Elena didn't make exactly \$152? Explain.\footnote{No, since there is probably some variability. For example, the traffic will vary from one day to next, and auction prices will vary depending on the quality of the merchandise and the interest of the attendees.}
\end{exercise}

Two important concepts concerning combinations of random variables have so far been introduced. First, a final value can sometimes be described as the sum of its parts in an equation. Second, intuition suggests that putting the individual average values into this equation gives the average value we would expect in total. This second point needs clarification -- it is guaranteed to be true in what are called \emph{linear combinations of random variables}.

A \term{linear combination} of two random variables $X$ and $Y$ is a fancy phrase to describe a combination
$$ aX + bY$$
where $a$ and $b$ are some fixed and known numbers. For John's commute time, there were five random variables -- one for each work day -- and each random variable could be written as having a fixed coefficient of 1:
$$ 1X_1 + 1 X_2 + 1 X_3 + 1 X_4 + 1 X_5 $$
For Elena's net gain or loss, the $X$ random variable had a coefficient of +1 and the $Y$ random variable had a coefficient of -1.

When considering the average of a linear combination of random variables, it is safe to plug in the mean of each random variable and then compute the final result. For a few examples of nonlinear combinations of random variables -- cases where we cannot simply plug in the means -- see the footnote.\footnote{If $X$ and $Y$ are random variables, consider the following combinations: $X^{1+Y}$, $X\times Y$, $X/Y$. In such cases, plugging in the average value for each random variable and computing the result will not generally lead to an accurate average value for the end result.}

\begin{termBox}{\tBoxTitle{Linear combinations of random variables and the average result}
If $X$ and $Y$ are random variables, then a linear combination of the random variables is given by
\begin{align}\label{linComboOfRandomVariablesXAndY}
aX + bY
\end{align}
where $a$ and $b$ are some fixed numbers. To compute the average value of a linear combination of random variables, plug in the average of each individual random variable and compute the result:
\begin{align*}
a\times E(X) + b\times E(Y)
\end{align*}
Recall that the expected value is the same as the mean, e.g. $E(X) = \mu_X$.}
\end{termBox}

\begin{example}{Leonard has invested \$6000 in Google Inc. (stock ticker: GOOG) and \$2000 in Exxon Mobil Corp. (XOM). If $X$ represents the change in Google's stock next month and $Y$ represents the change in Exxon Mobil stock next month, write an equation that describes how much money will be made or lost in Leonard's stocks for the month.}
For simplicity, we will suppose $X$ and $Y$ are not in percents but are in decimal form (e.g. if Google's stock increases 1\%, then $X=0.01$; or if it loses 1\%, then $X=-0.01$). Then we can write an equation for Leonard's gain as
\begin{align*}
\$6000\times X + \$2000\times Y
\end{align*}
If we plug in the change in the stock value for $X$ and $Y$, this equation gives the change in value of Leonard's stock portfolio for the month. A positive value represents a gain, and a negative value represents a loss.
\end{example}

\begin{exercise}\label{expectedChangeInLeonardsStockPortfolio}
Suppose Google and Exxon Mobil stocks have recently been rising 2.1\% and 0.4\% per month, respectively. Compute the expected change in Leonard's stock portfolio for next month.\footnote{$E(\$6000\times X + \$2000\times Y) = \$6000\times 0.021 + \$2000\times 0.004 = \$134$.}
% library(stockPortfolio); gr <- getReturns(c("GOOG", "XOM"), start="2006-01-01"); gr
\end{exercise}

\begin{exercise}
You should have found that Leonard expects a positive gain in Guided Practice~\ref{expectedChangeInLeonardsStockPortfolio}. However, would you be surprised if he actually had a loss this month?\footnote{No. While stocks tend to rise over time, they are often volatile in the short term.}
\end{exercise}

\subsection{Variability in linear combinations of random variables}

Quantifying the average outcome from a linear combination of random variables is helpful, but it is also important to have some sense of the uncertainty associated with the total outcome of that combination of random variables. The expected net gain or loss of Leonard's stock portfolio was considered in Guided Practice~\ref{expectedChangeInLeonardsStockPortfolio}. However, there was no quantitative discussion of the volatility of this portfolio. For instance, while the average monthly gain might be about \$134 according to the data, that gain is not guaranteed. Figure~\ref{changeInLeonardsStockPortfolioFor36Months} shows the monthly changes in a portfolio like Leonard's during the 36 months from 2009 to 2011. The gains and losses vary widely, and quantifying these fluctuations is important when investing in stocks.

\begin{figure}[ht]
\centering
\includegraphics[width=0.65\textwidth]{ch_probability/figures/changeInLeonardsStockPortfolioFor36Months/changeInLeonardsStockPortfolioFor36Months}
\caption{The change in a portfolio like Leonard's for the 36 months from 2009 to 2011, where \$6000 is in Google's stock and \$2000 is in Exxon Mobil's.}
\label{changeInLeonardsStockPortfolioFor36Months}
\end{figure}

Just as we have done in many previous cases, we use the variance and standard deviation to describe the uncertainty associated with Leonard's monthly returns. To do so, the variances of each stock's monthly return will be useful, and these are shown in Table~\ref{sumStatOfGOOGXOM}. The stocks' returns are nearly independent.

\begin{table}
\centering
\begin{tabular}{lrrr}
\hline
	& Mean ($\bar{x}$) & Standard deviation ($s$) & Variance ($s^2$) \\
\hline
GOOG & 0.0210	& 0.0846					&	0.0072	\\
XOM & 0.0038		& 0.0519					&	0.0027	\\
\hline
\end{tabular}
\caption{The mean, standard deviation, and variance of the GOOG and XOM stocks. These statistics were estimated from historical stock data, so notation used for sample statistics has been used.}
\label{sumStatOfGOOGXOM}
\end{table}

Here we use an equation from probability theory to describe the uncertainty of Leonard's monthly returns; we leave the proof of this method to a dedicated probability course. The variance of a linear combination of random variables can be computed by plugging in the variances of the individual random variables and squaring the coefficients of the random variables:
\begin{align*}
Var(aX + bY) = a^2\times Var(X) + b^2\times Var(Y)
\end{align*}
It is important to note that this equality assumes the random variables are independent; if independence doesn't hold, then more advanced methods are necessary. This equation can be used to compute the variance of Leonard's monthly return:
\begin{align*}
Var(6000\times X + 2000\times Y)
	&= 6000^2\times Var(X) + 2000^2\times Var(Y) \\
	&= 36,000,000\times 0.0072 + 4,000,000\times 0.0027 \\
	&= 270,000
\end{align*}
The standard deviation is computed as the square root of the variance: $\sqrt{270,000} = \$520$. While an average monthly return of \$134 on an \$8000 investment is nothing to scoff at, the monthly returns are so volatile that Leonard should not expect this income to be very stable.

\begin{termBox}{\tBoxTitle{Variability of linear combinations of random variables}
The variance of a linear combination of random variables may be computed by squaring the constants, substituting in the variances for the random variables, and computing the result:
\begin{align*}
Var(aX + bY) = a^2\times Var(X) + b^2\times Var(Y)
\end{align*}
This equation is valid as long as the random variables are independent of each other. The standard deviation of the linear combination may be found by taking the square root of the variance.}
\end{termBox}

\begin{example}{Suppose John's daily commute has a standard deviation of 4 minutes. What is the uncertainty in his total commute time for the week?} \label{sdOfJohnsCommuteWeeklyTime}
The expression for John's commute time was
\begin{align*}
X_1 + X_2 + X_3 + X_4 + X_5
\end{align*}
Each coefficient is 1, and the variance of each day's time is $4^2=16$. Thus, the variance of the total weekly commute time is
\begin{align*}
&\text{variance }= 1^2 \times  16 + 1^2 \times  16 + 1^2 \times  16 + 1^2 \times  16 + 1^2 \times  16 = 5\times 16 = 80 \\
&\text{standard deviation } = \sqrt{\text{variance}} = \sqrt{80} = 8.94
\end{align*}
The standard deviation for John's weekly work commute time is about 9 minutes.
\end{example}

\begin{exercise}
The computation in Example~\ref{sdOfJohnsCommuteWeeklyTime} relied on an important assumption: the commute time for each day is independent of the time on other days of that week. Do you think this is valid? Explain.\footnote{One concern is whether traffic patterns tend to have a weekly cycle (e.g. Fridays may be worse than other days). If that is the case, and John drives, then the assumption is probably not reasonable. However, if John walks to work, then his commute is probably not affected by any weekly traffic cycle.}
\end{exercise}

\begin{exercise}\label{elenaIsSellingATVAndBuyingAToasterOvenAtAnAuctionVariability}
Consider Elena's two auctions from Guided Practice~\ref{elenaIsSellingATVAndBuyingAToasterOvenAtAnAuction} on page~\pageref{elenaIsSellingATVAndBuyingAToasterOvenAtAnAuction}. Suppose these auctions are approximately independent and the variability in auction prices associated with the TV and toaster oven can be described using standard deviations of \$25 and \$8. Compute the standard deviation of Elena's net gain.\footnote{The equation for Elena can be written as
\begin{align*}
(1)\times X + (-1)\times Y
\end{align*}
The variances of $X$ and $Y$ are 625 and 64. We square the coefficients and plug in the variances:
\begin{align*}
(1)^2\times Var(X) + (-1)^2\times Var(Y) = 1\times 625 + 1\times 64 = 689
\end{align*}
The variance of the linear combination is 689, and the standard deviation is the square root of 689: about \$26.25.}
\end{exercise}

Consider again Guided Practice~\ref{elenaIsSellingATVAndBuyingAToasterOvenAtAnAuctionVariability}. The negative coefficient for $Y$ in the linear combination was eliminated when we squared the coefficients. This generally holds true: negatives in a linear combination will have no impact on the variability computed for a linear combination, but they do impact the expected value computations.

\index{random variable|)}

%_________________
\section{Continuous distributions (special topic)}
\label{contDist}

\index{data!FCID|(}
\index{hollow histogram|(}
\begin{example}{Figure~\ref{fdicHistograms} shows a few different hollow histograms of the variable \var{height} for 3 million US adults from the mid-90's.\footnote{This sample can be considered a simple random sample from the US population. It relies on the USDA Food Commodity Intake Database.} How does changing the number of bins allow you to make different interpretations of the data?}\label{usHeights}
Adding more bins provides greater detail. This sample is extremely large, which is why much smaller bins still work well. Usually we do not use so many bins with smaller sample sizes since small counts per bin mean the bin heights are very volatile.
\end{example}

\begin{figure}[ht]
\centering
\includegraphics[width=0.9\textwidth]{ch_probability/figures/fdicHistograms/fdicHistograms}
\caption{Four hollow histograms of US adults heights with varying bin widths.}
\label{fdicHistograms}
\end{figure}

\begin{example}{What proportion of the sample is between \resp{180} cm and \resp{185} cm tall (about 5'11" to 6'1")?}\label{contDistProb}
We can add up the heights of the bins in the range \resp{180} cm and \resp{185} and divide by the sample size. For instance, this can be done with the two shaded bins shown in Figure~\ref{usHeightsHist180185}. The two bins in this region have counts of 195,307 and 156,239 people, resulting in the following estimate of the probability:
\begin{eqnarray*}
\frac{195307+156239}{\text{3,000,000}} = 0.1172
\end{eqnarray*}
This fraction is the same as the proportion of the histogram's area that falls in the range \resp{180} to \resp{185} cm.
\end{example}

\begin{figure}
\centering
\includegraphics[width=0.9\textwidth]{ch_probability/figures/usHeightsHist180185/usHeightsHist180185}
\caption{A histogram with bin sizes of 2.5 cm. The shaded region represents individuals with heights between \resp{180} and \resp{185} cm. }
\label{usHeightsHist180185}
\end{figure}

\subsection{From histograms to continuous distributions}

Examine the transition from a boxy hollow histogram in the top-left of Figure~\ref{fdicHistograms} to the much smoother plot in the lower-right. In this last plot, the bins are so slim that the hollow histogram is starting to resemble a smooth curve. This suggests the population height as a \emph{continuous} numerical variable might best be explained by a curve that represents the outline of extremely slim bins.

This smooth curve represents a \term{probability density function} (also called a \term{density} or \term{distribution}), and such a curve is shown in Figure~\ref{fdicHeightContDist} overlaid on a histogram of the sample. A density has a special property: the total area under the density's curve is 1. 

\begin{figure}[tbh]
\centering
\includegraphics[width=0.9\textwidth]{ch_probability/figures/fdicHeightContDist/fdicHeightContDist}
\caption{The continuous probability distribution of heights for US adults.}
\label{fdicHeightContDist}
\end{figure}

\index{hollow histogram|)}

\subsection{Probabilities from continuous distributions}

We computed the proportion of individuals with heights \resp{180} to \resp{185} cm in Example~\ref{contDistProb} as a fraction:
\begin{eqnarray*}
\frac{\text{number of people between \resp{180} and \resp{185}}}{\text{total sample size}}
\end{eqnarray*}
We found the number of people with heights between \resp{180} and \resp{185} cm by determining the fraction of the histogram's area in this region. Similarly, we can use the area in the shaded region under the curve to find a probability (with the help of a computer):
\begin{eqnarray*}
P(\text{\var{height} between \resp{180} and \resp{185}})
	= \text{area between \resp{180} and \resp{185}}
	= 0.1157
\end{eqnarray*}
The probability that a randomly selected person is between \resp{180} and \resp{185} cm is 0.1157. This is very close to the estimate from Example~\ref{contDistProb}: 0.1172. 

\begin{figure}
\centering
\includegraphics[width=0.8\textwidth]{ch_probability/figures/fdicHeightContDistFilled/fdicHeightContDistFilled}
\caption{Density for heights in the US adult population with the area between 180 and 185 cm shaded. Compare this plot with Figure~\ref{usHeightsHist180185}.}
\label{fdicHeightContDistFilled}
\end{figure}

\begin{exercise}
Three US adults are randomly selected. The probability a single adult is between \resp{180} and \resp{185} cm is 0.1157.\footnote{Brief answers: (a) $0.1157 \times 0.1157 \times 0.1157 = 0.0015$. (b) $(1-0.1157)^3 = 0.692$} \vspace{-1.5mm}
\begin{enumerate}
\setlength{\itemsep}{0mm}
\item[(a)] What is the probability that all three are between \resp{180} and \resp{185} cm tall?
\item[(b)] What is the probability that none are between \resp{180} and \resp{185} cm?
\end{enumerate}
\end{exercise}

\begin{example}{What is the probability that a randomly selected person is \textbf{exactly} \resp{180}~cm? Assume you can measure perfectly.}
\label{probabilityOfExactly180cm}
This probability is zero. A person might be close to \resp{180} cm, but not exactly \resp{180} cm tall. This also makes sense with the definition of probability as area; there is no area captured between \resp{180}~cm and \resp{180}~cm.
\end{example}

\begin{exercise}
Suppose a person's height is rounded to the nearest centimeter. Is there a chance that a random person's \textbf{measured} height will be \resp{180} cm?\footnote{This has positive probability. Anyone between \resp{179.5} cm and \resp{180.5} cm will have a \emph{measured} height of \resp{180} cm. This is probably a more realistic scenario to encounter in practice versus Example~\ref{probabilityOfExactly180cm}.}
\end{exercise}

\index{data!FCID|)}



