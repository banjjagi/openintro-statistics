\chapter{확률 변수 분포}
\label{modeling}

\index{분포!정규|(}

%_________________
\section{정규 분포}
\label{normalDist}

실무에서 보게되는 모든 분포 중에서, 분포 하나가 가장 일반적이다. 좌우대칭, 단봉, 종모양은 통계학에서 어디서나 산재한다. 사실, 그림~\ref{simpleNormal}에 나왔듯이, 너무나 일반적이라 사람들이 종종 \term{정규곡선}(normal curve) 혹은 \termsub{정규 분포}{분포!정규}로 알고 있다.
\footnote{수학 표현식으로 형식화한 첫번째 사람, 가우스(Frederic Gauss)을 따라 가우스 분포(Gaussian distribution)로도 소개된다.}
SAT 점수와 미국 성인 신장 같은 변수는 정규분포를 가깝게 따른다.

\begin{figure}
\centering
\includegraphics[width=0.7\textwidth]{ch_distributions/figures/simpleNormal/simpleNormal}
\caption{정규 곡선}
\label{simpleNormal}
\end{figure}

\begin{termBox}{\tBoxTitle{정규분포에 대한 사실}
많은 변수가 거의 정규분포지만, 어떤 것도 정확하게 정규분포는 아니다. 따라서, 어떤 단일 문제 대해 완벽하지는 않지만 정규분포가 다양한 문제에 매우 유용하다. 자료 탐색과 통계학에서 중요한 문제를 푸는데 정규분포를 사용한다.\vspace{0.7mm}}
\end{termBox}


\subsection{정규 분포 모형}

항상 좌우대칭, 단봉, 종모양 곡선으로 정규분포모형이 기술된다. 하지만, 모형 세부사항에 따라 정규분포곡선은 달라 보인다. 구체적으로, 정규분포모형은 매개변수 두개로 조정될 수 있다: 평균과 표준편차. 아마도 추측했듯이, 평균을 바꾸면 종모양 곡선이 왼쪽 혹은 오른쪽으로 이동하고, 표준편차를 바꾸면 곡선을 늘이고 줄인다. 그림~\ref{twoSampleNormals}에서 왼쪽 패널에 평균 0과 표준편차 1, 오른쪽 패널에 평균 19, 표준편차 4를 갖는 정규분포가 나와있다. 그림~\ref{twoSampleNormalsStacked}에 동일한 축에 두 분포를 나타냈다.

\begin{figure}[hht]
\centering
\includegraphics[width=0.85\textwidth]{ch_distributions/figures/twoSampleNormals/twoSampleNormals}
\caption{두 곡선은 정규분포를 나타낸다. 하지만, 중심과 퍼짐에서 차이가 있다. 평균 0과 표준편차 1을 갖는 정규분포를 \term{표준정규분포}(standard normal distribution)라고 부른다.}
\label{twoSampleNormals}
\end{figure}

\begin{figure}[hht]
\centering
\includegraphics[width=0.6\textwidth]{ch_distributions/figures/twoSampleNormalsStacked/twoSampleNormalsStacked}
\caption{그림~\ref{twoSampleNormals}에 나온 정규모형, 하지만 동일한 척도로 함께 도식화했다.}
\label{twoSampleNormalsStacked}
\end{figure}

만약 정규분포가 평균 $\mu$ 와 표준편차 $\sigma$ 를 갖는다면, 정규분포를 $N(\mu, \sigma)$ 작성한다.\marginpar[\raggedright\vspace{-5mm}

$N(\mu, \sigma)$\vspace{1mm}\\\footnotesize 평균 $\mu$ \& \\표준편차 $\sigma$를 \\갖는 정규분포]{\raggedright\vspace{-5mm}

$N(\mu, \sigma)$\vspace{1mm}\\\footnotesize Normal dist.\\with mean $\mu$\\\& st. dev. $\sigma$}.

그림~\ref{twoSampleNormalsStacked}에 나온 두 분포를 다음과 같이 작성한다.
\begin{align*}
N(\mu=0,\sigma=1)\quad\text{and}\quad N(\mu=19,\sigma=4)
\end{align*}

평균과 표준편차가 정규분포를 정확하게 기술하기 때문에, 평균과 표준편차를 \termsub{모수}{모수}라고 부른다.

\begin{exercise}

다음을 모수를 갖는 정규분포를 간략하게 작성하시오.\footnote{(a)~$N(\mu=5,\sigma=3)$. (b)~$N(\mu=-100, \sigma=10)$. (c)~$N(\mu=2, \sigma=9)$.}
\begin{parts}
\item 평균~5, 표준편차~3,
\item 평균~-100, 표준편차~10, 
\item 평균~2, 표준편차~9.
\end{parts}
\end{exercise}

\subsection{Z-점수 표준화}

\begin{example}{
표~\vref{satACTstats}에 SAT와 ACT 총점에 대한 평균과 표준편차가 나와있다.
SAT와 ACT 점수 분포는 둘다 거의 정규분포다. 영아가 SAT에서 1800점 받았고, 정훈이가 ACT에서 24점 받았다. 누가 더 좋은 성적은 냈는가?}\label{actSAT}
참고로 표준편차를 사용한다. 영아는 SAT 평균에 보다 1 표준편차 높다: $1500 + 300=1800$. 정훈이는 ACT 평균보다 1 표준편차 높다: $21+0.6\times 5=24$. 그림~\ref{satActNormals}에서, 영아가 정훈이보다 더 좋은 경향이 있음을 볼 수 있다. 그래서 영아의 점수가 더 낫다.
\end{example}

\begin{table}
\centering
\begin{tabular}{l r r}
  \hline
  & SAT & ACT \\
  \hline
평균 \hspace{0.3cm} & 1500 & 21 \\
표준편차 & 300 & 5 \\
   \hline
\end{tabular}
\caption{SAT 와 ACT에 대한 평균과 표준편차.}
\label{satACTstats}
\end{table}

\begin{figure}
\centering
\includegraphics[width=65mm]{ch_distributions/figures/satActNormals/satActNormals}
\caption{SAT와 ACT 점수 분포와 함께 영아와 정훈이 점수가 나와 있다.}
\label{satActNormals}
\end{figure}


예제~\ref{actSAT}는 Z-점수라는 표준화기법을 사용했다. 거의 정규분포 관측점에 가장 흔히 사용되는 방법이지만, 임의 분포에도 사용될 수 있다. 관측점 \term{Z-점수}(Z-score)\marginpar[\raggedright\vspace{-3mm}

$Z$\vspace{1mm}\\\footnotesize Z-score, the\\standardized\\observation]{\raggedright\vspace{-3mm}

$Z$\vspace{1mm}\\\footnotesize Z-점수, \\표준화된\\관측점}\index{Z@$Z$}는 관측점이 평균 위 혹은 아래 위치한 표준편차 숫자로 정의된다. 
만약 관측점이 평균보다 1 표준편차 위라면, Z-점수는 1 이다. 만약 평균보다 1.5 표준편차 \emph{아래} 위치한다면, Z-점수는 -1.5가 된다. 만약 $x$가 분포 $N(\mu, \sigma)$에서 나온 관측점이라면, Z-점수를 수학적으로 다음과 같이 정의한다.

\begin{eqnarray*}
Z = \frac{x-\mu}{\sigma}
\end{eqnarray*}

$\mu_{SAT}=1500$, $\sigma_{SAT}=300$, $x_{영아}=1800$ 정보를 사용해서, 영아 Z-점수를 찾는다:

\begin{eqnarray*}
Z_{영아} = \frac{x_{영아} - \mu_{SAT}}{\sigma_{SAT}} = \frac{1800-1500}{300} = 1
\end{eqnarray*}

\begin{termBox}{\tBoxTitle{Z-점수(Z-score)}
관측점 Z-점수는 관측점이 평균 위 혹은 아래 위치한 표준편차 숫자다. 다음 식을 사용해서 모수로 평균 $\mu$와 표준편차 $\sigma$를 갖는 분포를 따르는 관측점 $x$에 대한 Z-점수를 계산한다.
\begin{eqnarray*}
Z = \frac{x-\mu}{\sigma}
\end{eqnarray*}}
\end{termBox}

\begin{exercise}
ACT 평균과 표준편차와 함께 정훈이 ACT 점수 24를 사용해서 Z-점수를 계산하라.
\footnote{$Z_{Tom} = \frac{x_{Tom} - \mu_{ACT}}{\sigma_{ACT}} = \frac{24 - 21}{5} = 0.6$}
\end{exercise}

평균 이상 관측점은 항상 양의 Z-점수를 갖고, 평균 아래 관측점은 음의 Z-점수를 갖는다. 만약 관측점이 평균과 동일하다면 (예를 들어, SAT 점수 1500), Z-점수는 $0$ 이다.

\begin{exercise}
$X$가 $N(\mu=3, \sigma=2)$에서 나온 확률변수로 놓고, $x=5.19$를 관측했다고 가정하자.
(a) $x$에 대한 Z-점수를 찾아라. (b) Z-점수를 사용해서 $x$ 가 평균보다 얼마나 표준편차 위 혹은 아래에 위치하는지 알아내라.
\footnote{(a) Z-점수는 $Z = \frac{x-\mu}{\sigma} = \frac{5.19 - 3}{2} = 2.19/2 = 1.095$ 가 된다. 
(b) $x$ 관측점은 평균보다 1.095 표준편차 \emph{위}에 위치한다. $Z$가 양수기 때문에 평균보다 위에 있어야 된다는 것을 안다.}
\end{exercise}

\begin{exercise} \label{headLZScore}

주머니쥐(brushtail possums) 머리둘레 길이는 평균 92.6 mm 와 표준편차 3.6 mm 를 갖고 거의 정규분포를 따른다. 머리둘레 길이 95.4 mm와  85.8 mm을 갖는 주머니쥐에 대한 Z-점수를 계산하시오. 
\footnote{머리둘레 길이 $x_1=95.4$ mm 갖는 주머니쥐: $Z_1 = \frac{x_1 - \mu}{\sigma} = \frac{95.4 - 92.6}{3.6} = 0.78$. 머리둘레 길이 $x_2=85.8$ mm 갖는 주머니쥐: $Z_2 = \frac{85.8 - 92.6}{3.6} = -1.89$.}
\end{exercise}

Z-점수를 사용해서 어느 관측점이 다른 관측점 보다 더 특이한 식별할 수 있다. 만약 한 관측점 $x_1$의 Z-점수 절대값이 다른 관측점 $x_2$의 Z-점수 절대값보다 크다면 $x_1$ 관측점이 다른 관측점 $x_2$보다 더 특이하다고 한다: $|Z_1| > |Z_2|$. 분포가 좌우대칭일 때 이러한 기법이 특히 통찰력있다.

\begin{exercise}
Guided Practice~\ref{headLZScore}에 나온 어느 관측점이 더 특이한가?
\footnote{
두번째 관측점 Z-점수 \emph{절대값}이 첫번째보다 더 크기 때문에, 두번째 관측점이 더 특이한 머리둘레 길이를 갖는다.}
\end{exercise}

\subsection{정규확률표(Normal probability table)}

\begin{example}{
예제~\ref{actSAT}에서 영아는 SAT 시험에서 1800 점을 얻었는데 $Z=1$에 상응한다. SAT 수험생중에 몇분위수에 위치하는지 알고싶다.}
영아 \term{백분위수}(percentile)는 영아보다 낮은 SAT 점수를 받은 학생 백분율이다. 그림~\ref{satBelow1800}에서 영아보다 낮은 점수를 받은 사람을 나타내는 영역을 음영처리했다. 정규곡선 아래 전체 면적은 항상 1 이다. SAT 시점에서 영아보다 낮은 점수를 받은 학생 비율은 그림~\ref{satBelow1800}에 음영처리된 \emph{면적}과 같다: 0.8413. 다른 말로, 영아는 SAT 수험생 중에서 $84^{th}$ 백분위수에 있다.
\end{example}

\begin{figure}[htb]
   \centering
   \includegraphics[width=0.6\textwidth]{ch_distributions/figures/satBelow1800/satBelow1800}
   \caption{SAT 점수에 대한 정규 모형으로 영아보다 낮은 점수를 받은 수험생 영역을 음영처리했다.}
   \label{satBelow1800}
\end{figure}

정규모형을 사용해서 백분위수를 찾을 수도 있다. Z-점수와 상응하는 백분위수를 나열한 \term{정규확률표}(normal probability table)를 사용해서 Z-점수에 기반해서 백분위수를 식별할 수도 있고 역으로도 가능하다. 통계 소프트웨어를 사용할 수도 있다.

정규확률표가 부록~\vref{normalProbabilityTable}에 나와 있고, 표~\ref{zTableShort}에 일부 나와있다. 이 표를 사용해서 특정 Z-점수에 대응되는 백분위수를 식별한다. 예를 들어, $Z=0.43$ 백분위수는 표~\ref{zTableShort}에서 행 $0.4$, 열 $0.03$에 나와 있다: 0.6664, 즉 $66.64^{th}$ 백분위수다. 일반적으로 $Z$를 소수점 아래 두자리에서 반올림한다. 정규확률표에서 첫번째 소수점 아래 자리에 적절한 행을 식별하고 나서, 두번째 소수점 아래 값을 나타내는 열을 알아낸다. 행과 열의 교차점이 관측점의 백분위수가 된다.

\begin{figure}
\centering
\includegraphics[width=0.9\textwidth]{ch_distributions/figures/normalTails/normalTails}
\caption{ $Z$의 좌측 면적이 관측점 백분위수를 나타낸다.}
\label{normalTails}
\end{figure}

\begin{table}
\centering
\begin{tabular}{c | rrrrr | rrrrr |}
  \cline{2-11}
&&&& \multicolumn{4}{c}{ $Z$의 두번째 소수점 자리} &&& \\
  \cline{2-11}
$Z$ & 0.00 & 0.01 & 0.02 & \highlightT{0.03} & \highlightO{0.04} & 0.05 & 0.06 & 0.07 & 0.08 & 0.09 \\
  \hline
  \hline
0.0 & \scriptsize{0.5000} & \scriptsize{0.5040} & \scriptsize{0.5080} & \scriptsize{0.5120} & \scriptsize{0.5160} & \scriptsize{0.5199} & \scriptsize{0.5239} & \scriptsize{0.5279} & \scriptsize{0.5319} & \scriptsize{0.5359} \\
  0.1 & \scriptsize{0.5398} & \scriptsize{0.5438} & \scriptsize{0.5478} & \scriptsize{0.5517} & \scriptsize{0.5557} & \scriptsize{0.5596} & \scriptsize{0.5636} & \scriptsize{0.5675} & \scriptsize{0.5714} & \scriptsize{0.5753} \\
  0.2 & \scriptsize{0.5793} & \scriptsize{0.5832} & \scriptsize{0.5871} & \scriptsize{0.5910} & \scriptsize{0.5948} & \scriptsize{0.5987} & \scriptsize{0.6026} & \scriptsize{0.6064} & \scriptsize{0.6103} & \scriptsize{0.6141} \\
%  May comment out 0.0-0.2 to make extra space. Then insert the following line:
%  $\vdots$ &   $\vdots$ &   $\vdots$ &   $\vdots$ &   $\vdots$ &   $\vdots$ &   $\vdots$ &   $\vdots$ &   $\vdots$ &   $\vdots$ &   $\vdots$ \\
  0.3 & \scriptsize{0.6179} & \scriptsize{0.6217} & \scriptsize{0.6255} & \scriptsize{0.6293} & \scriptsize{0.6331} & \scriptsize{0.6368} & \scriptsize{0.6406} & \scriptsize{0.6443} & \scriptsize{0.6480} & \scriptsize{0.6517} \\
\highlightT{0.4} & \scriptsize{0.6554} & \scriptsize{0.6591} & \scriptsize{0.6628} & \highlightT{\scriptsize{0.6664}} & \scriptsize{0.6700} & \scriptsize{0.6736} & \scriptsize{0.6772} & \scriptsize{0.6808} & \scriptsize{0.6844} & \scriptsize{0.6879} \\
  \hline
  0.5 & \scriptsize{0.6915} & \scriptsize{0.6950} & \scriptsize{0.6985} & \scriptsize{0.7019} & \scriptsize{0.7054} & \scriptsize{0.7088} & \scriptsize{0.7123} & \scriptsize{0.7157} & \scriptsize{0.7190} & \scriptsize{0.7224} \\
  0.6 & \scriptsize{0.7257} & \scriptsize{0.7291} & \scriptsize{0.7324} & \scriptsize{0.7357} & \scriptsize{0.7389} & \scriptsize{0.7422} & \scriptsize{0.7454} & \scriptsize{0.7486} & \scriptsize{0.7517} & \scriptsize{0.7549} \\
  0.7 & \scriptsize{0.7580} & \scriptsize{0.7611} & \scriptsize{0.7642} & \scriptsize{0.7673} & \scriptsize{0.7704} & \scriptsize{0.7734} & \scriptsize{0.7764} & \scriptsize{0.7794} & \scriptsize{0.7823} & \scriptsize{0.7852} \\
\highlightO{0.8} & \scriptsize{0.7881} & \scriptsize{0.7910} & \scriptsize{0.7939} & \scriptsize{0.7967} & \highlightO{\scriptsize{0.7995}} & \scriptsize{0.8023} & \scriptsize{0.8051} & \scriptsize{0.8078} & \scriptsize{0.8106} & \scriptsize{0.8133} \\
  0.9 & \scriptsize{0.8159} & \scriptsize{0.8186} & \scriptsize{0.8212} & \scriptsize{0.8238} & \scriptsize{0.8264} & \scriptsize{0.8289} & \scriptsize{0.8315} & \scriptsize{0.8340} & \scriptsize{0.8365} & \scriptsize{0.8389} \\
  \hline
  \hline
  1.0 & \scriptsize{0.8413} & \scriptsize{0.8438} & \scriptsize{0.8461} & \scriptsize{0.8485} & \scriptsize{0.8508} & \scriptsize{0.8531} & \scriptsize{0.8554} & \scriptsize{0.8577} & \scriptsize{0.8599} & \scriptsize{0.8621} \\
  1.1 & \scriptsize{0.8643} & \scriptsize{0.8665} & \scriptsize{0.8686} & \scriptsize{0.8708} & \scriptsize{0.8729} & \scriptsize{0.8749} & \scriptsize{0.8770} & \scriptsize{0.8790} & \scriptsize{0.8810} & \scriptsize{0.8830} \\
  $\vdots$ &   $\vdots$ &   $\vdots$ &   $\vdots$ &   $\vdots$ &   $\vdots$ &   $\vdots$ &   $\vdots$ &   $\vdots$ &   $\vdots$ &   $\vdots$ \\
   \hline
\end{tabular}
\caption{정규확률표 일부. $Z=0.43$을 갖는 정규확률변수에 대한 백분위수가 \highlightT{강조}되었다. 0.8000에 가장 가까운 백분위수도 \highlightO{강조}되었다.}
\label{zTableShort}
\end{table}

백분위수에 연관된 Z-점수도 찾을 수 있다. 예를 들어, $80^{th}$ 백분위수에 대한 $Z$ 값을 식별하려면, 정규확률표 중간에서 0.8000에 가장 가까운 값을 찾는다: 0.7995. 행과 열을 조합해서 $80^{th}$ 백분위수에 대한 Z-점수를 알아낸다: 0.84.

\begin{exercise}
SAT 시험에서 영아보다 더 높은 점수를 받은 SAT 수험생 비율을 알아내시오.
\footnote{만약 영아보다 84\% 학생이 더 낮은 점수를 받았다면, 더 높은 점수를 받은 학생 숫자는 16\%가 되어야만 된다. (일반적으로, 정규모형 혹은 다른 어떤 연속분포가 사용될 때 등순위는 무시된다.}
\end{exercise}


\textC{\newpage}


\subsection{정규확률 예제}

누적 SAT 점수는 정규모형($N(\mu=1500, \sigma=300)$)으로 근사된다.

\begin{example}{새논은 무작위로 선택된 SAT(Scholastic Aptitude Test, 대학입학 학습 능력 적성 시험) 수험생이고, 새논의 SAT에 관해서는 어떤것도 알려진 것이 없다. 새논이 적어도 SAT 점수 1630 점을 얻을 확률은 얼마인가?}\label{satAbove1630Exam}
먼저, 항상 정규분포 그림을 그리고 라벨을 붙여라. (그림이 정확할 필요는 없다.) 새논이 1630 점 이상을 얻을 가능성에 관심이 있어서 위쪽 꼬리를 음영처리한다:
\begin{center}
\includegraphics[width=0.45\textwidth]{ch_distributions/figures/satAbove1630/satAbove1630}
\end{center}
상기 그림에는 평균과 평균 상하 2 표준편차이 나와있다. 곡선 아래 음영처리된 지역을 찾는 가장 간단한 방식은 Z-점수 경계값을 사용하는 것이다. $\mu=1500$, $\sigma=300$, 경계값 $x=1630$ 에서, Z-점수는 다음과 같이 계산된다:

\begin{eqnarray*}
Z = \frac{x - \mu}{\sigma} = \frac{1630 - 1500}{300} = \frac{130}{300} = 0.43
\end{eqnarray*}

표~\ref{zTableShort}, 혹은 부록~\vref{normalProbabilityTable}에 나온 정규확률표에서 $Z=0.43$ 백분위수를 찾으면 0.6664가 나온다. 하지만, 백분위수는 0.43보다 \emph{낮은} Z-점수를 기술하고 있다. $Z=0.43$ 보다 \emph{높은} 영역을 알아내려면, 1 에서 아래꼬리 영역을 빼서 계산한다:

\begin{center}
\includegraphics[width=0.5\textwidth]{ch_distributions/figures/subtractingArea/subtractingArea}
\end{center}
새논이 SAT에서 적어도 1630점 받을 확률은 0.3336 이다.
\end{example}

\begin{tipBox}{\tipBoxTitle{항상 그림을 먼저 그리고 나서 두번째로 Z-점수를 알아낸다}
임의 정규확률 상황에서 \emph{항상 항상 항상} 정규곡선을 그리고 라벨을 붙이고, 먼저 관심 영역을 음영처리하라. 그림이 확률 추정치를 줄 것이다. \vspace{3mm}

상황을 나타내는 그림을 도식화한 후에, 관심있는 관측점에 대한 Z-점수를 식별하라.\vspace{1mm}}
\end{tipBox}

\begin{exercise}
만약 새논이 적어도 1630 점을 받을 확률이 0.3336 이라면, 새논이 1630 점보다 낮은 점수를 받을 확률은 얼마인가? 이 연습문제를 나타내는 정규곡선을 그리시오. 단, 위쪽 대신에 아래쪽 영역을 음영처리한다. 
\footnote{예제~\ref{satAbove1630Exam}에서 확률을 알아냈다: 0.6664. 예제~\ref{satAbove1630Exam}에서 ``0.6664'' 아래 음영처리된 영역이 이번 예제에 대한 그림이 된다.}
\end{exercise}

\begin{example}{에드워드는 SAT에서 1400 점을 받았다. 에드워드 백분위수는 어떻게 될까?} \label{edwardSatBelow1400}
먼저, 그림이 필요하다. 에드워드 백분위수는 1400 점 만큼 높지 않은 수험생 비율이다. 1400점 기준 좌측 점수가 해당 영역이다.
\begin{center}
\includegraphics[width=0.4\textwidth]{ch_distributions/figures/satBelow1400/satBelow1400}
\end{center}

평균 $\mu=1500$, 표준편차 $\sigma=300$, 경계값 $x=1400$ 으로 필요한 정보를 식별하면 Z-점수를 계산하기 쉽다:

\begin{eqnarray*}
Z = \frac{x - \mu}{\sigma} = \frac{1400 - 1500}{300} = -0.33
\end{eqnarray*}
정규확률표를 사용해서, 행에서 $-0.3$, 열에서 $0.03$을 식별하면 대응되는 확률은 $0.3707$이 된다. 에드워드는 $37^{th}$ 번째 백분위수 위치에 있다.
\end{example}

\begin{exercise}
예제~\ref{edwardSatBelow1400} 결과를 사용해서 에드워드보다 더 좋은 점수를 받은 SAT 수험생 비율을 계산하시오. 그림도 새로 그리시오.
\footnote{만약 에드워드가 SAT 수험생 37\% 보다 더 점수가 높다면, 약 63\% 수험생이 에드워드 보다 더 점수가 높아야 된다.\\
\includegraphics[height=12mm]{ch_distributions/figures/satBelow1400/satAbove1400}}
\end{exercise}

\begin{tipBox}{\tipBoxTitle{우측 면적}
대부분 교과서에서 정규확률표는 좌측 면적을 제시한다. 만약 우측 면적이 필요하다면, 먼저 좌측 면적을 알아내고 나서 1 에서 뺀다.}
\end{tipBox}

\begin{exercise}
스튜어트는 SAT 2100 점을 받았다. 각 부분에 대해서 그림을 그리시오. (a) 스튜어트 백분위수는 얼마인가? (b) 스튜어트 보다 더 높은 점수를 받은 수험생은 몇 퍼센트인가?
\footnote{수치 정답: (a) 0.9772. (b) 0.0228.}
\end{exercise}

표본 100 명에 기초하여\footnote{표본은 USDA 식료품 섭취 데이터베이스에서 추출함}, 나이 20세에서 62세 사이 미국성인 남성 신장은 평균 70.0'', 표준편차 3.3'' 거의 정규분포를 따른다.

\begin{exercise}
마이크(Mike)는 5'7'', 짐(Jim)은 6'4'' 신장을 갖는다. (a) 마이크의 신장 백분위수는 어떻게 될까? (b) 짐의 신장 백분위수는 어떻게 될까? 각각에 대해서 도식화도 하시오.
\footnote{먼저 신장을 인치로 환산한다: 67과 76 인치. 도식화된 그림은 아래 있다. (a) $Z_{마이크} = \frac{67 - 70}{3.3} = -0.91\ \to\ 0.1814$. (b) $Z_{짐} = \frac{76 - 70}{3.3} = 1.82\ \to\ 0.9656$. \\\includegraphics[height=12mm]{ch_distributions/figures/mikeAndJimPercentiles/mikeAndJimPercentiles}}
\end{exercise}

마지막 몇 문제는 특정 관측점에 대한 확률과 백분위수를 알아내는데 집중했다. 특정 백분위수에 대응되는 관측점을 알고자 하면 어떨까?

\begin{example}{에릭의 키는 $40^{th}$ 번째 백분위수에 위치해 있다. 에릭의 신장은 얼마나 될까?}\label{normalExam40Perc}
항상 그렇듯이, 도식화를 먼저 해라.\vspace{-1mm}
\begin{center}
\includegraphics[width=0.35\textwidth]{ch_distributions/figures/height40Perc/height40Perc}\vspace{-1mm}
\end{center}
이 경우에, 아래쪽 꼬리 확률이 (0.40) 로 알려졌다. 그림에서 음영처리했다. 알아내고자 하는 것은 이 확률값에 상응하는 관측점이다. 첫번째 단계로, $40^{th}$ 번째 백분위수와 연관된 Z-점수를 알아낸다.

백분위수가 50\% 아래에 있기 때문에, $Z$ 값은 음수라는 것을 알고 있다. 정규확률표 음수부분을 살펴보고, 표 \emph{안쪽}에서 0.4000 에 가장 가까운 확률을 찾는다. 0.4000 값이 행에서 -0.2, 열에서 0.05와 0.06 사이에 위치한 것을 찾아낸다. 0.05에 좀더 가깝기 때문에 이 값을 취한다: $Z=-0.25$.

$Z_{에릭}=-0.25$ 와 모집단 모수 $\mu=70$ 와 $\sigma=3.3$ 인치라는 것을 알게 되서, Z-점수 공식을 만들어서 에릭의 신장을 알아낸다. 미지수를 $x_{Erik}$로 라벨 붙인다:

\begin{eqnarray*}
-0.25 = Z_{Erik} = \frac{x_{Erik} - \mu}{\sigma} = \frac{x_{Erik} - 70}{3.3}
\end{eqnarray*}

$x_{Erik}$에 대해서 풀면 신장은 69.18 인치가 나온다. 즉, 에릭은 약 5'9'' (5-피트, 9-인치에 대한 표기법).
\end{example}

\begin{example}{$82^{nd}$ 번째 백분위수에 성인 남성 신장은 얼마인가?}
다시 도식화를 먼저한다. \textC{\vspace{-3mm}}
\begin{center}
\includegraphics[width=0.28\textwidth]{ch_distributions/figures/height82Perc/height82Perc}\textC{\vspace{-1mm}}
\end{center}
다음으로, $82^{nd}$ 번째 백분위수에 Z-점수를 알아내는데 양수다. Z-표를 보고 행에서 0.9, 가장 가까운 열은 0.02, 즉 $Z=0.92$에서 $Z$ 값을 찾는다. 마지막으로 평균 $\mu$, 표준편차 $\sigma$, Z-점수 $Z=0.92$를 Z-점수 공식에 꽂아넣어 신장 $x$를 구한다:

\begin{eqnarray*}
0.92 = Z = \frac{x-\mu}{\sigma} = \frac{x - 70}{3.3}
\end{eqnarray*}
상기 방정식을 풀면 $82^{nd}$ 번째 백분위수 신장은 73.04 인치 즉, 약 6'1'' 이 나온다.
\end{example}

\begin{exercise}
(a) SAT 점수에서 $95^{th}$ 번째 백분위수는 어떻게 되는가? (b) 남성 신장에 대한 $97.5^{th}$ 번째 백분위수는 어떻게 되는가? 항시 정규확률 문제 풀듯이, 도식화를 먼저 하라. \footnote{기억할 점: 도식화를 먼저하고 나서, Z-점수를 구한다. (도식화는 독자에게 남겨둔다.) Z-점수는 백분위수와 정규확률표를 사용해서 구한다. (a) 정규확률표 (중간부) 확률 부분에서 0.95를 찾는데, 행에서 1.6 열에서 (약) 0.05, 즉, $Z_{95}=1.65$. $Z_{95}=1.65$, $\mu = 1500$, $\sigma = 300$ 을 알면, Z-점수 공식을 세운다: $1.65 = \frac{x_{95} - 1500}{300}$. $x_{95}$에 대해 해를 구한다: $x_{95} = 1995$. (b) 유사하게 $Z_{97.5} = 1.96$을 구하고 나서, 신장에 대한 Z-점수 공식을 세우고, 계산하면 $x_{97.5} = 76.5$이 된다.}
\end{exercise}

\begin{exercise}\label{more74Less69}
(a)~무작위로 선택한 성인 남성 신장이 적어도 5'9'' (69 인치)가 될 확률은 얼마인가? (b) 성인 남성이 5'9'' (69 인치) 보다 작을 확률은 얼마인가?\footnote{수치 해답: (a) 0.1131. (b) 0.3821.}
\end{exercise}

\begin{example}{무작위로 고른 성인 남성의 신장이 5'9'' -- 6'2'' 사이에 위치할 확률은 얼마인가?}
신장을 인치로 환산하면 69 인치에서 74 인치가 된다. 먼저 도식화를 한다. 관심있는 영역은 위쪽 꼬리도, 아래쪽 꼬리도 아니다.\textC{\vspace{-2mm}}
\begin{center}
\includegraphics[width=0.35\textwidth]{ch_distributions/figures/between59And62/between59And62}\textC{\vspace{-2mm}}
\end{center}
곡선 아래 전체 면적은 1 이다. 만약 음영처리되지 않는 두 꼬리 영역을 알아내면(Guided Practice~\ref{more74Less69}에서 해당 영역 면적은 각각 $0.3821$ 와 $0.1131$ 이다), 중간 영역을 알 수 있다: \textC{\vspace{-2mm}}
\begin{center}
\includegraphics[width=0.6\textwidth]{ch_distributions/figures/subtracting2Areas/subtracting2Areas}\textC{\vspace{-2mm}}
\end{center}
즉, 5'9'' -- 6'2'' 사이 확률은 0.5048 이 된다.
\end{example}

\begin{exercise}
SAT 점수가 1500 -- 2000 사이 수험생은 몇 퍼센트인가? \footnote{축약한 해답이다. (도식화를 확실히 하라!) 
먼저 1500 점 이하를 받는 수험생을 퍼센트를 구하고, 2000 점 이상 받는 수험생을 구한다: $Z_{1500} = 0.00 \to 0.5000$ (이하 영역) $Z_{2000} = 1.67 \to 0.0475$ (이상 영역). 최종 정답: $1.0000-0.5000 - 0.0475 = 0.4525$.}
\end{exercise}

\begin{exercise}
성인 남성 신장이 5'5'' -- 5'7'' 사이가 몇 퍼센트인가? \footnote{5'5'' 는 65 인치. 5'7'' 는 67 인치. 수치 해답: $1.000 - 0.0649 - 0.8183 = 0.1168$, 즉, 11.68\%.}
\end{exercise}

\CalculatorVideos{calculations for the normal distribution}


\subsection{68-95-99.7 규칙}
정규분포에서 평균의 표준편차 1,2,3 안에 포진할 확률에 대한 유용한 경험규칙을 제시한다. Z-표나 계산기를 사용하지 않고, 빠른 추정을 시행할 때, 이 방법이 폭넓게 실무에서 사용된다.

\begin{figure}[hht]
\centering
\includegraphics[height=1.9in]{ch_distributions/figures/6895997/6895997}
\caption{정규분포에서 평균의 표준편차 1,2,3 안에 포진할 확률.}
\label{6895997}
\end{figure}

\begin{exercise}
Z-표를 사용해서 약 68\%, 95\%, 99.7\% 관측점이 정규분포에서 평균의 표준편차 1,2,3 안에 포진하는 것을 검증하라. 예를 들어, 먼저 $Z=-1$ 와 $Z=1$ 사이 포진될 영역을 먼저 찾는다. 약 0.68 면적을 차지한다. 유사하게 $Z=-2$ 와 $Z=2$ 사이는 약 0.95 면적이 되어야 한다.
\footnote{
도식화를 먼저한다. $Z=-1$ and $Z=1$ 사이 영역을 알아내기 위해서, 정규확률표를 사용해서 $Z=-1$ 아래 영역을 알아낸다. 다음으로, $Z=-1$ -- $Z=1$ 사이 면적이 약 0.68 이 됨을 검증한다. $Z=-2$ -- $Z=2$ 에 대해서도 이 과정을 반복한다. $Z=-3$ -- $Z=3$ 에 대해서도 마찬가지다.}
\end{exercise}

정규확률변수가 평균으로부터 표준편차 4, 5 혹은 더 큰 값에 속하는 것도 가능하다. 하지만, 데이터가 거의 정규분포를 따른다면 이러한 출현은 매우 드물다. 평균에서 4 표준편차 보다 더 멀리 있을 확률은 약 30,000 가지 사례 중 하나가 된다. 표준편차 5, 6에 대해서 각각 약 3백5십만 중에 하나, 10억중에 하나가 된다.

\begin{exercise}
SAT 점수는 평균 $\mu = 1500$ 와 표준편차  $\sigma = 300$을 모수로 근사하게 정규분포를 따른다. (a) 900 -- 2100 사이 점수를 받는 수험생은 몇 퍼센트인가? (b) 1500 -- 2100 사이 점수를 받는 수험생은 몇 퍼센트인가?
\footnote{900 과 2100 점은 평균 위 아래 2 표준편차를 나타낸다. 이것이 의미하는 것은 약 수험생 95\%가 900 -- 2100 사이 점수를 받는 것이다. (b) 정규모형은 좌우대칭이기 때문에, (a) (수험생의 $\frac{95\%}{2} = 47.5\%$) 의 수험생 절반이 900 -- 1500 점을 받을 것인 반면에, 
수험생 47.5\%는 1500 -- 2100 점을 받을 것이다.}
\end{exercise}

%_________________
\section{정규분포 근사를 평가하기}
\label{assessingNormal}

많은 확률과정을 정규분포로 잘 근사할 수 있다. 이미 좋은 예제 두개를 살펴봤다: SAT 점수, 미국성인 남자 신장. 정규모형을 사용하는 것이 극단적으로 편리하고 도움이 되지만, 정규성(normality)은 항상 근사(approximation)라는 것을 기억하는 것이 중요하다. 정규분포 가정의 적합성을 시험하는 것이 많은 자료분석에서 중요한 단계다.

\index{정규확률그림|(}

예제~\ref{normalExam40Perc}는 미국 남성 신장 분포가 정규모형으로 잘 근사됨을 제시한다. 데이터가 정규분포를 따른다는 가정아래에서 작업을 추진하는데 관심이 있지만, 데이터 정규성 가정이 일리가 있는지 먼저 점검해야만 된다.

정규성 가정을 점검하는 시각적 방법이 두가지 있는데 모두 신속하게 구현되고 해석될 수 있다. 첫번째 방법은 그림~\ref{fcidMHeights} 왼쪽 패널에 나와있듯이, 단순 히스토그램과 데이터를 가장 잘 적합하는 정규곡선을 겹치는 것이다. 표본평균 $\bar{x}$ 와 표준편차 $s$를 정규곡선을 가장 잘 적합하는 모수로 사용한다. 곡선이 히스토그램과 더 가까울수록 정규모형가정에 더 타당성이 간다. 또다른 좀더 일반적인 방법은 그림~\ref{fcidMHeights} 오른쪽 패널에 나온 \term{정규확률그림}(normal probability plot)을 조사하는 것이다.\footnote{\term{분위수-분위수 그림}(quantile-quantile plot)으로도 흔히 불린다.} 점들이 직선에 더 가까이 위치할수록, 데이터가 정규모형을 따른다는 확신을 갖을 수 있다.

\begin{figure}
\centering
\includegraphics[width=0.8\textwidth]{ch_distributions/figures/fcidMHeights/fcidMHeights}
\caption{남성 100명 신장 표본. 관측점은 가장가까운 정수 인치로 반올림했다. 따라서, 정규확률그림에서 점들이 증가할 때 점프하는 것처럼 보이는 것이 설명된다.}
\label{fcidMHeights}
\end{figure}

\textC{\newpage}

\begin{example}{정규분포에서 모의실험을 위해 표본 40, 100, 400을 추출해서 데이터셋 3개를 만들었다. 표본 데이터셋에 대한 히스토그램과 정규확률그림이 그림~\ref{normalExamples}에 나와있다. 실제 데이터를 그림으로 나타낼 때 이것이 좋은 벤치마크가 될 것이다.} \label{normalExamplesExample}

\begin{figure}
\centering
\includegraphics[width=\textwidth]{ch_distributions/figures/normalExamples/normalExamples}
\caption{정규분포에서 모의실험한 3가지 데이터셋에 대한 히스토그램과 정규확률그림; $n=40$ (왼쪽), $n=100$ (가운데), $n=400$ (우측).}
\label{normalExamples}
\end{figure}

관측점 40개로 구성된 모의실험 데이터셋에 대한 히스토그램(위쪽)과 정규확률그림(아래쪽)이 왼쪽 패널에 나와있다. 데이터셋이 너무 작어서 히스토그램에서 명확한 구조를 볼 수는 없다. 정규확률그림도 이런 점을 반영하고 있다. 직선에서 일부 편차가 보인다. 이러한 작은 데이터셋에 대해 이러한 편차는 예상된다.

관측점 100개로 구성된 모의실험 데이터셋에 대한 진단 그림이 중간 패널에 나와있다. 히스토그램이 정규성을 좀더 보여주고 있고, 정규확률그림도 더 좋은 적합을 보여주고 있다. 직선에서 눈에 띄는 편차를 보는 관측점이 일부 있지만, 이런 관측점이 특별히 극단적이지는 않다. 

관측점 400개로 구성된 데이터셋은 정규분포와 매우 닮은 히스토그램을 보여주고, 정규확률그림도 거의 완벽한 직선이다. 정규확률그림에서 직선에서 약간 편차를 보이는 (가장 큰) 관측점이 하나 있다. 만약 관측점이 직선에서 떨어져 3배 편차를 갖는다면, 실제 데이터셋에서 매우 중요하다. 명백한 이상점이 정규분포된 데이터에서 발생할 수 있지만 드물다.

히스토그램은 표본크기가 증가함에 따라 좀더 정규분포를 닮아가고, 정규확률그림도 더 직선에 가까워지고 안정된 모습이 되어감에 주목한다.

\end{example}

\textC{\pagebreak}

\begin{example}{NBA 농구선수 신장은 정규분포를 따르는가? 그림~\ref{nbaNormal}에 나타난 2008-9 시즌 NBA 농구선수 435 명을 생각해보자.\footnote{데이터 출처 \oiRedirect{textbook-nba_com}{www.nba.com}.}}
먼저 NBA 농구선수 신장에 대한 히스토그램과 정규확률그림을 생성한다. 왼쪽 패널에 히스토그램은 다소 왼쪽으로 치우쳤는데, 좌우대칭인 정규분포와 대조된다. 정규확률그림에 점들이 직선에 가까이 위치하지 않고 ``물결(wave)''이 있는 것처럼 보인다. 이런 특성과 예제~\ref{normalExamplesExample}의 정규분포에서 나온 400개 표본 관측점과 비교하면 좀더 강하게 정규모형에서 벗어난 것을 보게된다. NBA 농구선수신장은 정규분포에서 나온 것 같지 않다.
\end{example}

\begin{figure}
\centering
\includegraphics[width=\textwidth]{ch_distributions/figures/nbaNormal/nbaNormal}
\caption{2008-9 시즌 NBA 농구선수 신장에 대한 히스토그램과 정규확률그림.}
\label{nbaNormal}
\end{figure}

\begin{example}{
정규분포로 포커 승리를 근사할 수 있을까? 50일에 걸치 한 포커 승부사 승리를 생각해보자. 이 데이터에 대한 히스토그램과 정규확률그림이 그림~\ref{pokerNormal}에 나와있다.}
히스토그램에 데이터는 강하게 우측으로 기울어져 있는데, 정규확률그림의 우측 상단에 강하게 벗어난 것에 상응한다. 만약 예제~\ref{normalExamplesExample}에서 나온 정규분포 40개 표본 관측점과 비교한다면, 포커승리 데이터는 매우 강하게 정규모형에서 벗어난 것처럼 보인다.
\end{example}

\begin{figure}
\centering
\includegraphics[width=\textwidth]{ch_distributions/figures/pokerNormal/pokerNormal}
\caption{포커 데이터에 대한 최고 적합 정규 곡선을 갖는 히스토그램과 정규확률그림.}
\label{pokerNormal}
\end{figure}

\begin{exercise}\label{normalQuantileExercise}
그림~\ref{normalQuantileExer}에서 어느 데이터셋이 그럴듯하게 거의 정규분포에서 나온 것 같은지 알아내시오. 여러분은 본인이 내린 결론에 대해 모두 확신을 갖나요? 네개 그림에서는 각각 100 (좌측 상단), 50 (우측 상단), 500 (좌측 하단), 15 (좌측 하단)개 점이 있다.
\footnote{정답은 약간 다를 수 있다. 
좌측상단 그림은 데이터셋에 가장 작은 값에서 일부 이탈이 보인다; 구체적으로 데이터셋 왼쪽 꼬리는 유의해야 되는 이상점이 있다. 우측상단과 좌측하단 그림은 각각의 표본크기에 대한 직선에서 명백하거나 극단적인 이탈이 보여지지 않는다. 따라서 정규모형이 데이터셋에 대해서 타당해보인다. 우측하단 그림에 일관된 곡률(curvature)이 있는데 정규분포에서 나오지 않음을 제시한다. 만약 관측점 수직좌표만 면밀히 살펴본다면, -20 -- 0 사이 데이터가 몰려있고, 0 -- 70 사이 관측점이 5개가 흩어져 있음을 보게 된다. 이런 사실이 우측으로 매우 강하게 기울어진 분포임을 기술하고 있다.}
\end{exercise}

\begin{figure}
\centering
\includegraphics[width=0.9\textwidth]{ch_distributions/figures/normalQuantileExer/normalQuantileExer}
\caption{Guided Practice~\ref{normalQuantileExercise}에 대한 정규확률그림 4개.}
\label{normalQuantileExer}
\end{figure}

%When observations spike downwards on the left side of a normal probability plot, it means the data have more outliers in the left tail than we'd expect under a normal distribution. When observations spike upwards on the right side, it means the data have more outliers in the right tail than what we'd expect under the normal distribution.

\begin{exercise} \label{normalQuantileExerciseAdditional}
그림~\ref{normalQuantileExerAdditional}에 기울어진 분포 두개에 대한 정규확률그림이 나와있다. 한 분포는 아래쪽(좌측으로 기움)으로 다른 분포는 위쪽(우측으로 기움)으로 기울어졌다. 어느 분포가 어는 것인가?
\footnote{
점들이 수직축을 따라 어떻게 위치하는지 면밀히 조사한다. 첫번째 그림에서, 대부분의 점은 아래쪽에 위쪽에 적은 관측점이 산재되어 있다; 위쪽으로 기울어진 분포를 기술한다. 두번째 그림은 정반대 면을 보여주고 있다. 분포가 아래쪽으로 기울어져 있다.}
\end{exercise}

\begin{figure}
\centering
\includegraphics[width=0.9\textwidth]{ch_distributions/figures/normalQuantileExer/normalQuantileExerAdditional}
\caption{Guided Practice~\ref{normalQuantileExerciseAdditional}에 대한 정규확률그림.}
\label{normalQuantileExerAdditional}
\end{figure}


\index{정규확률그림|)}
\index{분포!정규|)}


%_________________
\section{기하분포 (특별 주제)}
\label{geomDist}

동전 \resp{앞면}이 나올 때까지 얼마나 오랜동안 동전을 던져야 될까? 혹은 주사위 \resp{1} 면이 나올 때까지 얼마나 오랜동안 주사위를 굴려야 될까? 이러한 질문에 기하분포를 사용해서 답을 구할 수 있다. 베르누이 분포를 사용해서 먼저 각 시도(동전 하나 던지기 혹은 주사위 굴리기)를 공식화한다. 그리고 나서 기하분포를 만드는데 확률(\ref{probability}~장)에서 학습한 도구와 조합한다.

\subsection{베르누이 분포}
\label{bernoulli}

\index{분포!베르누이|(}

스탠리 밀그램(Stanley Milgram)\index{밀그램, 스탠리}은 1963년에 일련의 실험을 시작해서 권위에 복종해서 이방인에게 심각한 충격을 주는 사람 비율을 추정했다. 약 65\% 사람이 권위에 복종해서 심한 충격을 주는 것을 밀그램이 알아냈다. 수년에 걸친 추가적인 연구도 근사적으로 공동체와 시간에 걸쳐 이 숫자가 일관성이 있는 것으로 제시했다.
\footnote{밀그램 실험에 대한 추가 정보는 다음 웹사이트를 참조한다.\par \ \ \hspace{0.2mm}\ \oiRedirect{textbook-milgram}{www.cnr.berkeley.edu/ucce50/ag-labor/7article/article35.htm}.}

밀그램 실험에서 사람 각각은 \term{시행}(trial)으로 간주될 수 있다. 최악의 충격을 가하는 것을 거부하면 \term{성공}(success) 라벨을 사람에게 붙인다. 만약 최악의 충격을 가하면 \term{실패}(failure) 라벨을 사람에게 붙인다. 35\% 사람만 가장 심각한 충격을 가하는 것을 거부했기 때문에, \term{성공확률}(probability of a success)을 $p=0.35$ 로 표기한다. 실패확률은 $q=1-p$로 표기한다.

따라서, \resp{성공} 혹은 \resp{실패}를 연구조사에서 각 사람에 대해 기록한다. 각 개인에 대한 시행이 단지 두가지 가능한 결과만 가질 때, 이를 \termsub{베르누이 확률변수}{분포!베르누이}라고 부른다.

\begin{termBox}{\tBoxTitle{베르누이 확률변수, 기술적 표현}
베르누이 확률변수는 정확하게 두가지 가능한 결과를 갖는다. 전형적으로 두가지 결과 중에 하나를 ``성공'' 그리고 다른 하나를 ``실패''라는 표식을 한다. 성공을 \resp{1} 그리고 실패를 \resp{0}으로 표기할 수도 있다.}
\end{termBox}

\begin{tipBox}{\tipBoxTitle{``성공''이 긍정적인 무언가일 필요는 없다.}
가장 심한 충격을 가하기 거부한 사람에게 ``성공'' 표식을 했고 그밖의 다른 사람은 ``실패''로 표식했다. 하지만, 아주 쉽게 이러한 표식을 정반대로 할 수도 있다. 수학적 얼개(mathematical framework)는 일관성만 있기만 하면, 어느 결과를 성공으로 혹은 실패로 표식을 하던 표식에 의존하지 않는다.}
\end{tipBox}

베르누이 확률변수를 종종 성공에 대해서는 \resp{1}, 실패에 대해서는 \resp{0} 으로 표기한다. 데이터 입력의 편리성에 더해서, 수학적으로도 다루기 편하다. 10번의 시행을 관측했다고 가정하자:

\begin{center}
\resp{0} \resp{1} \resp{1} \resp{1} \resp{1} \resp{0} \resp{1} \resp{1} \resp{0} \resp{0}
\end{center}
그러면 \term{표본 비율}(sample proportion), $\hat{p}$ 은 상기 관측점에 대한 표본 평균이다:
\begin{eqnarray*}
\hat{p} = \frac{\text{ 성공 횟수 (\#) }}{\text{시행 횟수 (\#) }} = \frac{0+1+1+1+1+0+1+1+0+0}{10} = 0.6
\end{eqnarray*}%

베르누이 확률변수에 대한 수학적 탐구를 좀더 연장할 수 있다. \resp{0} 과 \resp{1} 이 수치 결과라서, 베르누이 확률변수에 대한 평균과 표준편차를 정의할 수 있다.
\footnote{만약 ${p}$ 가 성공에 대한 진정한 확률이라면, 베르누이 확률변수 $X$에 대한 평균은 다음과 같이 주어진다.
\begin{align*}
\mu = E[X] &= P(X=0)\times0 + P(X=1)\times1 \\
	&= (1-p)\times0 + p\times 1 = 0+p = p
\end{align*}
마찬가지로 $X$에 대한 분산도 다음과 같이 계산될 수 있다:
\begin{align*}
\sigma^2 &= {P(X=0)(0-p)^2 + P(X=1)(1-p)^2} \\
	&= {(1-p)p^2 + p(1-p)^2} = {p(1-p)}
\end{align*}
표준편차는 $\sigma=\sqrt{p(1-p)}$ 이 된다.}

\begin{termBox}{\tBoxTitle{베르누이 확률변수, 수학적}
만약 $X$가 성공확률 $p$로 1 값과 $1-p$ 확률로 0 값을 갖는 확률변수라면, $X$는 다음 평균과 표준편차를 갖는 베르누이 확률변수다:
\begin{align*}
\mu &= p
	&\sigma&= \sqrt{p(1-p)}
\end{align*}}
\end{termBox}

일반적으로 베르누이 확률변수를 단지 두가지 결과(성공 혹은 실패)만 갖는 확률과정으로 생각하는 것이 유용하다. 그러면 성공과 실패에 대해 각각 \resp{1} 과 \resp{0} 수치표식을 사용하는 수학적 얼개를 만들 수 있다.

\index{분포!베르누이|)}

\subsection{기하분포}

\index{분포!기하|(}

\begin{example}{스미스 박사는 밀그램 실험을 반복하고 싶지만 가장 심한 충격을 가하지 않는 사람을 발견할 때까지만 사람을 표본추출하고자 한다. \footnote{이것은 가상이다. 왜냐하면, 현실에서 이러한 종류의 연구는 아마도 현행 윤리규범 아래에서 허락되지 않을 것이기 때문이다.}
만약 가장 심각한 충격을 가하지 \emph{않을} 확률이 여전히 0.35 이며 피험자가 독립이라면, 첫번째 피험자에서 연구를 멈출 가능성은 얼마나 되는가? 두번째 피험자? 세번째 피험자? 첫번째 성공을 찾기 전에 가장 심한 충격을 가하는 $n-1$ 피험자를 갖게 되면 어떨까? 즉, 첫번째 성공이 $n^{th}$ 번째 피험자에서 나온다. (만약 첫번째 성공이 다섯번째 피험자에서 나온다면, $n=5$가 된다.)} \label{waitForShocker}
첫번째 피험자 다음에 멈출 확률은 첫번째 피험자가 가장 심한 충격을 가하지 않을 가능성이 된다: $1-0.65=0.35$. 두번째 피험자 확률은 다음과 같다.
\begin{eqnarray*}
&&P(\text{두번째 피험자가 가장 심한 충격을 가하지 않을 첫번째 피험자}) \\
&&\quad = P(\text{첫번째 피험자 가함, 두번째 피험자 가하지 않음}) = (0.65)(0.35) = 0.228
\end{eqnarray*}
마찬가지로, 세번째 피험자 확률은 $(0.65)(0.65)(0.35) = 0.148$ 이 된다.

만약 첫번째 성공이 $n^{th}$ 번째 피험자라면, $n-1$ 실패와 최종적 성공 하나가 있는데 상응하는 확률은 $(0.65)^{n-1}(0.35)$ 이 된다. 이것은 $(1-0.35)^{n-1}(0.35)$ 와 같다.
\end{example}

예제~\ref{waitForShocker}는 기하분포를 예시하고 있는데 \term{독립이며 같은 분포를 따르는(iid)}(independent and identically distributed, iid) 베르누이 확률변수에 대해서 성공까지 대기시간을 기술한다. 이 경우에, \emph{독립}(independence) 측면은 예제에 피험자가 서로 영향을 주지 않는다를 의미하고 \emph{같은 분포}(identical)는 각각이 동일한 성공확률을 갖는다는 것을 의미한다.

예제~\ref{waitForShocker}에 나온 기하분포가 그림~\ref{geometricDist35}으로 나와있다. 일반적ㅇ르ㅗ, 기하분포 확률은 \term{기하급수적}으로 빠르게 감소한다.

\begin{figure}
\centering
\includegraphics[width=0.8\textwidth]{ch_distributions/figures/geometricDist35/geometricDist35}
\caption{성공확률이 $p=0.35$ 일 때, 기하분포.}
\label{geometricDist35}
\end{figure}

교과서에서 이 분포에 대한 첫번째 성공에 필요한 평균 (기대) 시행 횟수, 표준편차, 분산에 대한 공식을 도출하지 않지만, 각각에 대한 일반 공식이 다음에 나와 있다.

\begin{termBox}{\tBoxTitle{기하분포\index{분포!기하|textbf}}
한번 시행에 대한 성공확률이 $p$ 이고, 실패확률이 $1-p$ 라면, $n^{th}$ 번째 시행에서 첫번째 성공을 나올 확률은 다음으로 주어진다.\vspace{-1.5mm}
\begin{eqnarray}
(1-p)^{n-1}p
\end{eqnarray}
대기 시간에 대한 평균 (즉, 기대값), 분산, 표준편차는 다음으로 주어진다.\vspace{-2.5mm}
\begin{align}
\mu &= \frac{1}{p}
	&\sigma^2&=\frac{1-p}{p^2}
	&\sigma &= \sqrt{\frac{1-p}{p^2}}
\label{geomFormulas}
\end{align}}
\end{termBox}

평균과 기대값 모두에 대해서 기호 $\mu$를 사용하는 것이 우연은 아니다. 평균과 기대값은 다름 아닌 동일한 것이다.

왼쪽편 방정식~(\ref{geomFormulas})에서 말하는 것은 평균적으로 성공을 얻는데 $1/p$ 시험이 걸린다. 수학적 결과는 직관적으로 예상한 것과 일치한다. 만약 성공 확률이 높다면 (예를 들어, 0.8), 성공하는데 오래 걸리지 않을 것이다: 평균적으로 $1/0.8 = 1.25$ 시행. 만약 성공확률이 낮다면 (예를 들어, 0.1), 성공을 보기까지 많은 시행을 볼 것으로 예상된다: $1/0.1 = 10$ 시행.

\begin{exercise}
가장 심한 충격을 가하는 것을 거부할 피험자 확률이 약 0.35 다. 만약 충격을 가하지 않을 피험자가 나올때까지 피험자를 조사한다면, 얼마나 많은 피험자를 검사해야 되는가? 방정식~(\ref{geomFormulas})에 나온 첫번째 표현식이 도움이 될 수 있다.
\footnote{첫번째 성공을 찾아내는데 약 $1/0.35 = 2.86$ 피험자를 볼 것으로 예상된다.}
\end{exercise}

\begin{example}{
스미스 박사가 첫 네번째 피험자 중에서 첫 성공이 나올 확률은 얼마인가?} \label{marglimFirstSuccessIn4}
첫번째 성공이 첫째 피험자($n=1$), 두번째 피험자($n=2$), 세번째 피험자($n=3$), 네번째 피험자($n=4$)에서 나올 확률로 네 사건은 서로 겹치지 않는 결과다. 표본에서 피험자는 매우 큰 모집단에서 무작위로 뽑았기 때문에, 피험자 표본은 독립이다. 각 경우에 대해서 확률을 계산하고 개별 결과를 합한다:
\begin{eqnarray*}
&&P(n=1, 2, 3,\text{ or }4) \\
	&& \quad = P(n=1)+P(n=2)+P(n=3)+P(n=4) \\
	&& \quad = (0.65)^{1-1}(0.35) + (0.65)^{2-1}(0.35) + (0.65)^{3-1}(0.35) + (0.65)^{4-1}(0.35) \\
	&& \quad = 0.82
\end{eqnarray*}
피험자 네명 안에서 연구실험이 끝날 가능성은 82\%가 나온다.
\end{example}

\begin{exercise}

예제~\ref{marglimFirstSuccessIn4}를 푸는 좀더 똑똑한 방법을 알아내시오. 동일한 결과임도 보이시오.
\footnote{먼저 여사건 확률을 알아낸다: $P($첫 4번 시행에서 미성공$) = 0.65^4 = 0.18$. 다음으로, 1 에서 계산한 확률을 뺀다: $1-P($첫 4번 시행에서 미성공$) = 1-0.18 = 0.82$.}
\end{exercise}

\begin{example}{어떤 지역에서 가장 심한 충격을 가하는 피험자 비율이 ``단지'' 55\%만 된다고 가정하자. 만약 해당 지역에서 피험자를 무작위로 뽑는다면, 성공으로 간주되는 피험자를 찾기 전에 검사해야되는 피험자 예상 숫자는 얼마인가? 대기시간에 대한 표준편차는 얼마인가?} \label{onlyShocking55PercOfTheTimeExample}
성공은 피험자가 가장 심한 충격을 가하지 \textbf{않을} 때로, 해당 지역에 대해 $p=1-0.55=0.45$ 확률을 갖는다. 검사해야 되는 피험자 숫자는 $1/p = 1/0.45 = 2.22$ 이고, 표준편차는 $\sqrt{(1-p)/p^2} = 1.65$ 이 된다.
\end{example}

\begin{exercise}
예제~\ref{onlyShocking55PercOfTheTimeExample}에서 나온 $\mu = 2.22$ 와 $\sigma = 1.65$ 결과를 사용해서 3 혹은 적은 시행으로 끝날 실험 비율을 찾는데 정규모형을 사용하는 것이 적절한가?\footnote{아니다. 기하분포는 항상 우측으로 기울어져있고, 결코 정규분포로 잘 근사될 수 없다.}
\end{exercise}

독립 가정이 기하분포의 정확한 시나리오 기술에 매우 중요하다. 수학적으로, $n^{th}$ 번째 시행에 성공확률을 만들어내는데 독립 확률과정에 대한 곱셈정리를 사용해야 됐다. 의존성이 있는 시행에 대한 기하모형을 일반화하는 것은 단순한 작업이 아니다.

%\begin{exercise}
%The independence assumption is crucial to the geometric distribution's accurate description of a scenario. Why?\footnote{Independence simplified the situation. Mathematically, we can see that to construct the probability of the success on the $n^{th}$ trial, we had to use the Multiplication Rule for Independent processes. It is no simple task to generalize the geometric model for dependent trials.}
%\end{exercise}
\index{분포!기하|)}


\section{이항 분포 (특별 주제)}
\label{binomialModel}

\index{분포!이항|(}

\begin{example}{
``충격'' 연구에 참가할 피험자를 4명 무작위로 선정한다고 가정하자. 네명중 정확하게 한명이 성공할 가능성은 얼마인가? 편의상 네명을 알렌 ($A$), 브리트니 ($B$), 캐롤 ($C$), 대미안 ($D$) 으로 부르자. 이전 예제에서와 마찬가지로 35\% 피험자가 성공했다고 가정한다.}\label{oneRefuser}

한명이 충격을 가하는 것을 거절하는 시나리오를 생각해보자:
\begin{eqnarray*}
&&P(A=\text{\resp{거절}},\text{ }B=\text{\resp{충격}},\text{ }C=\text{\resp{충격}},\text{ }D=\text{\resp{충격}}) \\
 &&\quad =  P(A=\text{\resp{거절}})\ P(B=\text{\resp{충격}})\ P(C=\text{\resp{충격}})\ P(D=\text{\resp{충격}}) \\
 &&\quad =  (0.35)  (0.65)  (0.65)  (0.65) = (0.35)^1 (0.65)^3 = 0.096
\end{eqnarray*}
하지만 다른 시나리오 세개가 더 있다: 브리트니, 캐롤, 대미안이 충격을 가하는 것을 거절할 수 있는 피험자가 될 수 있다. 각각의 경우에 확률은 $(0.35)^1(0.65)^3$이 된다. 이 네가지 시나리오가 피험자 네명중 정확하게 한명만 가장 심한 충격을 가하는 것을 거절하는 모든 가능한 방법을 샅샅이 찾아 나온 것이다. 그래서 전체 확률은 $4\times(0.35)^1(0.65)^3 = 0.38$이 된다.
\end{example}

\begin{exercise}
가장 심한 충격을 가하는 것을 거절할 유일한 피험자가 브리트니인 시나리오가 확률 $(0.35)^1(0.65)^3$이 될 것을 확증하시오.~
\footnote{$P(A=\text{\resp{충격}},\text{ }B=\text{\resp{거절}},\text{ }C=\text{\resp{충격}},\text{ }D=\text{\resp{충격}}) = (0.65)(0.35)(0.65)(0.65) = (0.35)^1(0.65)^3$.}
\end{exercise}

\textC{\newpage}


\subsection{이항 분포}

예제~\ref{oneRefuser}에서 개요로 서술된 시나리오는 이항분포로 불리는 특별한 경우다. \termsub{이항 분포}(binomial distribution){분포!이항}는 성공확률 $p$를 갖는 독립적인 $n$개 베르누이 시행에서 정확하게 $k$개 성공을 갖는 확률을 기술한다 (예제~\ref{oneRefuser}에서는 $n=4$, $k=1$, $p=0.35$). 좀더 일반적으로 이항분포와 연관된 확률을 알아내고자 한다. 즉, 확률을 얻는데 $n$, $k$, $p$ 를 사용하는 공식을 원한다. 이를 위해서, 예제 각 부분을 다시 면밀히 살펴보자.

충격 가하는 것을 거부하는 일인이 될 수 있는 피험자 네명 있고, 네가지 시나리오 각각은 동일한 확률을 갖는다. 따라서, 최종 확률을 다음과 같이 식별할 수 있다.

\begin{eqnarray}
[\text{시나리오 갯수 (\#)}] \times P(\text{단일 시나리오})
\label{genBinomialFormula}
\end{eqnarray}
상기 방정식의 첫번째 구성요소가 $n=4$ 번 시행 중에서 성공 $k=1$을 정리하는 방법의 갯수다. 두번째 구성요소는 네가지 (동일하게 가능한) 시나리오의 어느 것에 대한 확률이다.

$n$번 시행에 $k$번 성공과 $n-k$번 실패의 일반적인 경우 아래에서 $P($단일 시나리오$)$를 생각해보자. 이런 시나리오에서, 독립 사건에 대한 곱셈 정리를 적용한다:

\begin{eqnarray*}
p^k(1-p)^{n-k}
\end{eqnarray*}

이것은 $P($단일 시나리오$)$에 대한 일반 공식이다.

둘째로, $n$번 시행에서 $k$번 성공을 선택하는 방법의 수에 대한 일반 공식을 도입한다:

\begin{eqnarray*}
{n\choose k} = \frac{n!}{k!(n-k)!}
\end{eqnarray*}

${n\choose k}$는 \term{n choose k} 라고 읽는다. \footnote{$n$ choose $k$ 에 대한 다른 표기법에는 $_nC_k$, $C_n^k$, $C(n,k)$이 있다.} 느낌표 표기법(즉, $k!$)은 \term{계승}(factorial)\label{factorialDefinitionInTheBinomialSection} 표현식을 나타낸다.

\begin{eqnarray*}
&& 0! = 1 \label{zeroFactorial} \\
&& 1! = 1 \\
&& 2! = 2\times1 = 2 \\
&& 3! = 3\times2\times1 = 6 \\
&& 4! = 4\times3\times2\times1 = 24 \\
&& \vdots \\
&& n! = n\times(n-1)\times...\times3\times2\times1
\end{eqnarray*}

공식을 사용해서, $n=4$번 시행에서 $k=1$ 성공을 선택하는 방법의 수를 계산할 수 있다:  

\begin{eqnarray*}
{4 \choose 1} = \frac{4!}{1!(4-1)!} =  \frac{4!}{1!3!} 
	= \frac{4\times3\times2\times1}{(1)(3\times2\times1)} = 4
\end{eqnarray*}
상기 결과는 예제~\ref{oneRefuser}에서 가능한 각 시나리오를 주의깊이 생각해서 정확하게 찾아낸 것과 같다.

방정식~(\ref{genBinomialFormula})에 시나리오 수에 대해서는 ``$n$ choose $k$'', 단일 시나리오 확률에 대해서는 $p^k(1-p)^{n-k}$으로 대체하면 일반 이항 공식이 산출된다.

\begin{termBox}{\tBoxTitle{이항 분포} 
단일 시행 성공확률을 $p$ 라고 가정하자. 그러면, $n$번 독립적인 시행에 정확하게 $k$번 성공을 관측하는 확률은 다음과 같이 주어진다.\vspace{-1mm}
\begin{eqnarray}
{n\choose k}p^k(1-p)^{n-k} = \frac{n!}{k!(n-k)!}p^k(1-p)^{n-k}
\label{binomialFormula}
\end{eqnarray}
부가적으로, 관측된 성공 횟수에 대한 평균, 분산, 표준편차는 다음과 같다. \vspace{-2mm}
\begin{align}
\mu &= np
	&\sigma^2 &= np(1-p)
	&\sigma &= \sqrt{np(1-p)}
\label{binomialStats}
\end{align}}
\end{termBox}

\begin{tipBox}{\tipBoxTitle{이항 분포인가? 검사할 4가지 조건.\label{isItBinomialTipBox}}
(1) 시행이 독립이다. \\
(2) 시행 횟수 $n$ 이 고정된다. \\
(3) 시행 결과 각각은 \emph{성공} 혹은 \emph{실패}로 분류될 수 있다. \\
(4) 성공 확률 $p$는 시행 각각에 대해 같다.}
\end{tipBox}

\begin{example}{무작위로 뽑은 피험자 학생 8명 중에서 3명이 가장 심한 충격을 가하는 것을 거절할 확률은 얼마인가? 즉, 8명 중 5명은 충격을 가할 확률은 얼마인가?}
이항모형을 적용하고자 한다. 그래서 조건을 검사한다. 시행 횟수는 고정 ($n=8$) (조건 2). 시행 결과 각각은 성공과 실패로 분류될 수 있다 (조건 3). 표본이 무작위이기 때문에, 시행은 독립이다 (조건 1). 성공확률은 각 시행에 대해 같다 (조건 4).

관심있는 결과에는 $n=8$번 시행에 $k=3$번 성공이 있다. 성공 확률은 $p=0.35$이다. 그래서 8명중 3명이 거절할 확률은 다음과 같이 주어진다.
\begin{eqnarray*}
{ 8 \choose 3}(0.35)^3(1-0.35)^{8-3}
	&=& \frac{8!}{3!(8-3)!}(0.35)^3(1-0.35)^{8-3} \\
	&=& \frac{8!}{3!5!}(0.35)^3(0.65)^5
\end{eqnarray*}
누승부분은 다음과 같이 처리한다:
\begin{eqnarray*}
\frac{8!}{3!5!} = \frac{8\times7\times6\times5\times4\times3\times2\times1}{(3\times2\times1)(5\times4\times3\times2\times1)} = \frac{8\times7\times6}{3\times2\times1} = 56
\end{eqnarray*}

$(0.35)^3(0.65)^5 \approx 0.005$을 사용해서, 최종 확률은 약 $56*0.005 = 0.28$이 나온다.
\end{example}

\begin{tipBox}{\tipBoxTitle{이항확률 계산하기}
이항모형을 사용하는 첫단계는 모형이 적절한지 점검하는 것이다. 두번째 단계는 $n$, $p$, $k$ 값을 식별한다. 마지막 단계는 공식을 적용하고 결과를 해석한다.}
\end{tipBox}

\begin{tipBox}{\tipBoxTitle{``$n$ choose $k$'' 계산}
일반적으로, 누승에 있는 소거를 즉시 실행하는 것이 유용하다. 대안으로, 많은 컴퓨터 프로그램과 계산기에 내장된 함수를 사용해서 ``$n$ choose $k$'', 누승, 심지어 전체 이항확률을 계산한다.}
\end{tipBox}

\begin{exercise}
만약 연구를 실행하고 무작위로 피험자 학생 40명을 뽑았다면, 얼마나 많은 학생이 가장 심한 충격을 가하는 것을 거절할 것으로 예상하는가? 거절할 피험자 학생 숫자에 대한 표준편차는 얼마인가? 방정식~(\ref{binomialStats})이 유용할 것이다. 
\footnote{질문은 기대값(평균)과 표준편차를 구하는 것으로 둘다 방정식~(\ref{binomialStats})에 나온 공식으로 직접 계산할 수 있다: $\mu=np = 40\times 0.35 = 14$ 와 $\sigma = \sqrt{np(1-p)} = \sqrt{40\times 0.35\times 0.65} = 3.02$. 
대략 관측점 95\%가 평균에 대해 표준편차 2 안에 포진하기 때문에 (\ref{variability}~절 참조), 충격을 가하길 거부할 피험자는 적어도 8명 하지만 20명보다는 적은 수를 아마도 관측할 것이다.}
\end{exercise}

\begin{exercise}
무작위로 선택된 흡연자가 일생동안 중증 폐질환으로 발전할 확률은 약 0.3 이다. 만약 흡연하는 친구 4명이 있다면, 이항모형을 적용할 조건이 만족되는가? \footnote{가능한 정답: 만약 친구가 서로 잘 알고 있다면, 독립성 가정은 아마도 충족되지 못한다. 예를 들어, 친분이 있으면 유사한 흡연 습관을 갖을 수도 있다.}
\end{exercise}

\begin{exercise}
\label{noMoreThanOneFriendWSevereLungCondition}%
친구 네명은 서로를 알지 못하고, 모집단에서 나온 무작위 표본인 것처럼 처리한다고 가정하자. 이항 모형이 적절한가? (a) 친구 중 어느 누구도 중증 폐질환으로 발전한지 않을 확률은 얼마인가? (b) 친구 한명이 중증 폐질환으로 발전할 확률은 얼마인가? (c) 1명 이상 심각한 폐질환으로 발전하지 않을 확률은 얼마인가?
\footnote{이항모형이 적절한지 점검하려면, 다음 조건이 만족되어야 한다. (i) 친구를 무작위 표본으로 처리한다고 가정했기 때문에, 표본은 독립이다. (ii) 고정된 시행 횟수가 있다($n=4$). (iii) 결과 각각은 성공 혹은 실패다. (iv) 성공 확률은 각 시행마다 같다. 왜냐하면 각 개인은 무작위 표본과 같다 (``성공''이 누군가 폐질환을 얻는 병리학적 선택이면 $p=0.3$). 방정식~\eqref{binomialFormula}에 나온 이항공식에서 (a)와 (b)를 계산한다: $P(0) =  {4 \choose 0} (0.3)^0 (0.7)^4 = 1\times1\times0.7^4 = 0.2401$, $P(1) = {4 \choose 1} (0.3)^1(0.7)^{3} = 0.4116$. 주목: $0!=1$. (c)는 (a)와 (b) 합으로 계산될 수 있다: $P(0) + P(1) = 0.2401 + 0.4116 = 0.6517$. 즉, 네명의 흡연하는 친구 중 한명이상 중증 폐질환으로 발전하지 않을 확률이 약 65\%가 된다.}
\end{exercise}

\begin{exercise}
친구 네명 중에서 적어도 2명이 일생동안 중증 폐질환으로 발전할 확률은 얼마인가?
\footnote{Guided Practice~\ref{noMoreThanOneFriendWSevereLungCondition}에서 0.6517로 계산된 값의 여확률 (한명이상 중증 폐질환으로 발전하지 않는다). 그래서 $1 - 0.6517$을 계산한다: 0.3483.}
\end{exercise}

\begin{exercise}
흡연하는 친구 7명이 있는데 무작위 흡연자 표본으로 처리할 수 있다고 가정하자. (a) 중증 폐질환으로 발전할 것으로 예상되는 친구는 몇명인가? 즉, 평균이 얼마인가? (b) 7명 친구중에 많아봐야 2명이 중증 폐질환으로 발전할 확률은 얼마인가?\footnote{(a)~$\mu=0.3\times7 = 2.1$. (b)~$P($친구 0명, 혹은  1명, 혹은 2명이 중증 폐질환으로 발전 $) = P(k=0) + P(k=1)+P(k=2) = 0.6471$.}
\end{exercise}

다음으로, 이항확률에서 첫번째 항을 고려해보자. 첫번째 항은 특정 시나리오 아래에서 ``$n$ choose $k$''다.

\begin{exercise}
임의 숫자 $n$에 대해서 ${n \choose 0}=1$ 와 ${n \choose n}=1$이 왜 참인가?
\footnote{수학 표현식을 말로 표현한다. $n$번 시행에서 0번 성공하고 $n$번 실패하는 다른 방법이 얼마나 될까? (1 방법) $n$번 시행에서 $n$번 성공하고 0번 실패하는 다른 방법이 얼마나 될까? (1 방법) }
\end{exercise}

\begin{exercise}
$n$번 시행에서 1번 성공하고 $n-1$번 실패하는 방법을 얼마나 마련할 수 있을까? $n$번 시행에서 $n-1$번 성공하고 1번 실패하는 방법을 얼마나 마련할 수 있을까?\footnote{1번 성공과 $n-1$번 실패: 
성공을 놓을 수 있는 유일한 장소가 정확하게 $n$개 있다. 그래서 1번 성공 $n-1$번 실패를 마련하는 방법은 $n$번 있다. 두번째 질문에도 유사한 논리전개를 펼 수 있다. 수학적으로, 다음 두 방정식을 확인하고 결과를 제시한다:
\begin{eqnarray*}
{n \choose 1} = n, \qquad {n \choose n-1} = n
\end{eqnarray*}}
\end{exercise}

\CalculatorVideos{calculations for the binomial coefficient and binomial formula}


\subsection{이항분포에 대한 정규 근사}

\index{분포!이항!정규 근사|(}

The binomial formula is cumbersome when the sample size ($n$) is large, particularly when we consider a range of observations. In some cases we may use the normal distribution as an easier and faster way to estimate binomial probabilities.

\begin{example}{Approximately 20\% of the US population smokes cigarettes. A local government believed their community had a lower smoker rate and commissioned a survey of 400 randomly selected individuals. The survey found that only 59 of the 400 participants smoke cigarettes. If the true proportion of smokers in the community was really 20\%, what is the probability of observing 59 or fewer smokers in a sample of 400 people?}\label{exactBinomialForN400P20SmokerExample}
We leave the usual verification that the four conditions for the binomial model are valid as an exercise.

The question posed is equivalent to asking, what is the probability of observing $k=0$, 1, ..., 58, or 59 smokers in a sample of $n=400$ when $p=0.20$? We can compute these 60 different probabilities and add them together to find the answer:
\begin{align*}
&P(k=0\text{ or }k=1\text{ or }\cdots\text{ or } k=59) \\
	&\qquad= P(k=0) + P(k=1) + \cdots + P(k=59) \\
	&\qquad=0.0041
\end{align*}
If the true proportion of smokers in the community is $p=0.20$, then the probability of observing 59 or fewer smokers in a sample of $n=400$ is less than 0.0041.
\end{example}

The computations in Example~\ref{exactBinomialForN400P20SmokerExample} are tedious and long. In general, we should avoid such work if an alternative method exists that is faster, easier, and still accurate. Recall that calculating probabilities of a range of values is much easier in the normal model. We might wonder, is it reasonable to use the normal model in place of the binomial distribution? Surprisingly, yes, if certain conditions are met.

\begin{exercise}
Here we consider the binomial model when the probability of a success is $p=0.10$. Figure~\ref{fourBinomialModelsShowingApproxToNormal} shows four hollow histograms for simulated samples from the binomial distribution using four different sample sizes: $n=10, 30, 100, 300$. What happens to the shape of the distributions as the sample size increases? What distribution does the last hollow histogram resemble?\footnote{The distribution is transformed from a blocky and skewed distribution into one that rather resembles the normal distribution in last hollow histogram}
\end{exercise}

\begin{figure}[h]
\centering
\includegraphics[width=0.92\textwidth]{ch_distributions/figures/fourBinomialModelsShowingApproxToNormal/fourBinomialModelsShowingApproxToNormal}
\caption{Hollow histograms of samples from the binomial model when $p=0.10$. The sample sizes for the four plots are $n=10$, 30, 100, and 300, respectively.}
\label{fourBinomialModelsShowingApproxToNormal}
\end{figure}

\begin{termBox}{\tBoxTitle{Normal approximation of the binomial distribution}
The binomial distribution with probability of success $p$ is nearly normal when the sample size $n$ is sufficiently large that $np$ and $n(1-p)$ are both at least 10. The approximate normal distribution has parameters corresponding to the mean and standard deviation of the binomial distribution:\vspace{-1.5mm}
\begin{align*}
\mu &= np
&&\sigma= \sqrt{np(1-p)}
\end{align*}}
\end{termBox}

The normal approximation may be used when computing the range of many possible successes. For instance, we may apply the normal distribution to the setting of Example~\ref{exactBinomialForN400P20SmokerExample}.

\begin{example}{How can we use the normal approximation to estimate the probability of observing 59 or fewer smokers in a sample of 400, if the true proportion of smokers is $p=0.20$?} \label{approxBinomialForN400P20SmokerExample}
Showing that the binomial model is reasonable was a suggested exercise in Example~\ref{exactBinomialForN400P20SmokerExample}. We also verify that both $np$ and $n(1-p)$ are at least 10:
\begin{align*}
np&=400\times 0.20=80
&n(1-p)&=400\times 0.8=320
\end{align*}
With these conditions checked, we may use the normal approximation in place of the binomial distribution using the mean and standard deviation from the binomial model:
\begin{align*}
\mu &= np = 80
&\sigma &= \sqrt{np(1-p)} = 8
\end{align*}
We want to find the probability of observing fewer than 59 smokers using this model.
\end{example}

\begin{exercise}
Use the normal model $N(\mu=80, \sigma=8)$ to estimate the probability of observing fewer than 59 smokers. Your answer should be approximately equal to the solution of Example~\ref{exactBinomialForN400P20SmokerExample}: 0.0041.\footnote{Compute the Z-score first: $Z=\frac{59 - 80}{8} = -2.63$. The corresponding left tail area is 0.0043.}
\end{exercise}


\subsection{The normal approximation breaks down on small intervals}

\begin{caution}
{The normal approximation may fail on small intervals}
{The normal approximation to the binomial distribution tends to perform poorly when estimating the probability of a small range of counts, even when the conditions are met.}
\end{caution}

Suppose we wanted to compute the probability of observing 69, 70, or 71 smokers in 400 when $p=0.20$. With such a large sample, we might be tempted to apply the normal approximation and use the range 69 to 71. However, we would find that the binomial solution and the normal approximation notably differ:
\begin{align*}
\text{Binomial:}&\ 0.0703
&\text{Normal:}&\ 0.0476
\end{align*}
We can identify the cause of this discrepancy using Figure~\ref{normApproxToBinomFail}, which shows the areas representing the binomial probability (outlined) and normal approximation (shaded). Notice that the width of the area under the normal distribution is 0.5 units too slim on both sides of the interval.

\begin{figure}[h]
\centering
\includegraphics[width=\textwidth]{ch_distributions/figures/normApproxToBinomFail/normApproxToBinomFail}
\caption{A normal curve with the area between 69 and 71 shaded. The outlined area represents the exact binomial probability.}
\label{normApproxToBinomFail}
\end{figure}

\begin{tipBox}{\tipBoxTitle{Improving the accuracy of the normal approximation to the binomial distribution}
The normal approximation to the binomial distribution for intervals of values is usually improved if cutoff values are modified slightly. The cutoff values for the lower end of a shaded region should be reduced by 0.5, and the cutoff value for the upper end should be increased by 0.5.}
\end{tipBox}

The tip to add extra area when applying the normal approximation is most often useful when examining a range of observations. While it is possible to apply it when computing a tail area, the benefit of the modification usually disappears since the total interval is typically quite wide.

\index{distribution!binomial!normal approximation|)}
\index{distribution!binomial|)}



%_________________
\section{More discrete distributions (special topic)}
\label{discreteModels}

\subsection{Negative binomial distribution}
\label{negativeBinomial}

\index{distribution!negative binomial|(}

The geometric distribution describes the probability of observing the first success on the $n^{th}$ trial. The \termsub{negative binomial distribution}{distribution!negative binomial} is more general: it describes the probability of observing the $k^{th}$ success on the $n^{th}$ trial.

\begin{example}{Each day a high school football coach tells his star kicker, Brian, that he can go home after he successfully kicks four 35 yard field goals. Suppose we say each kick has a probability $p$ of being successful. If $p$ is small -- e.g. close to 0.1 -- would we expect Brian to need many attempts before he successfully kicks his fourth field goal?}
We are waiting for the fourth success ($k=4$). If the probability of a success ($p$) is small, then the number of attempts ($n$) will probably be large. This means that Brian is more likely to need many attempts before he gets $k=4$ successes. To put this another way, the probability of $n$ being small is low.
\end{example}

To identify a negative binomial case, we check 4 conditions. The first three are common to the binomial distribution.\footnote{See a similar guide for the binomial distribution on page~\pageref{isItBinomialTipBox}.}

\begin{tipBox}{\tipBoxTitle{Is it negative binomial? Four conditions to check.}
(1) The trials are independent. \\
(2) Each trial outcome can be classified as a success or failure. \\
(3) The probability of a success ($p$) is the same for each trial. \\
(4) The last trial must be a success.}
\end{tipBox}

\begin{exercise}
Suppose Brian is very diligent in his attempts and he makes each 35 yard field goal with probability $p=0.8$. Take a guess at how many attempts he would need before making his fourth kick.\footnote{One possible answer: since he is likely to make each field goal attempt, it will take him at least 4 attempts but probably not more than 6 or 7.}
\end{exercise}

\begin{example}{In yesterday's practice, it took Brian only 6 tries to get his fourth field goal. Write out each of the possible sequence of kicks.} \label{eachSeqOfSixTriesToGetFourSuccesses}
Because it took Brian six tries to get the fourth success, we know the last kick must have been a success. That leaves three successful kicks and two unsuccessful kicks (we label these as failures) that make up the first five attempts. There are ten possible sequences of these first five kicks, which are shown in Table~\ref{successFailureOrdersForBriansFieldGoals}. If Brian achieved his fourth success ($k=4$) on his sixth attempt ($n=6$), then his order of successes and failures must be one of these ten possible sequences.

\begin{table}[ht]
\newcommand{\succObs}[1]{{\color{oiB}$\stackrel{#1}{S}$}}
\centering
\begin{tabular}{c|c ccc cl | r}
\multicolumn{8}{c}{\hspace{10mm}Kick Attempt} \\
& & 1 & 2 & 3 & 4 & \multicolumn{2}{l}{5\hfill6} \\
\hline
1&& $F$ & $F$ & \succObs{1} & \succObs{2} & \succObs{3} & \succObs{4} \\
2&& $F$ & \succObs{1} & $F$ & \succObs{2} & \succObs{3} & \succObs{4} \\
3&& $F$ & \succObs{1} & \succObs{2} & $F$ & \succObs{3} & \succObs{4} \\
4&& $F$ & \succObs{1} & \succObs{2} & \succObs{3} & $F$ & \succObs{4} \\
5&& \succObs{1} & $F$ & $F$ & \succObs{2} & \succObs{3} & \succObs{4} \\
6&& \succObs{1} & $F$ & \succObs{2} & $F$ & \succObs{3} & \succObs{4} \\
7&& \succObs{1} & $F$ & \succObs{2} & \succObs{3} & $F$ & \succObs{4} \\
8&& \succObs{1} & \succObs{2} & $F$ & $F$ & \succObs{3} & \succObs{4} \\
9&& \succObs{1} & \succObs{2} & $F$ & \succObs{3} & $F$ & \succObs{4} \\
10&& \succObs{1} & \succObs{2} & \succObs{3} & $F$ & $F$ & \succObs{4} \\
\end{tabular}
\caption{The ten possible sequences when the fourth successful kick is on the sixth attempt.}
\label{successFailureOrdersForBriansFieldGoals}
\end{table}

\end{example}

\begin{exercise} \label{probOfEachSeqOfSixTriesToGetFourSuccesses}
Each sequence in Table~\ref{successFailureOrdersForBriansFieldGoals} has exactly two failures and four successes with the last attempt always being a success. If the probability of a success is $p=0.8$, find the probability of the first sequence.\footnote{The first sequence: $0.2\times0.2\times0.8\times0.8\times0.8\times0.8 = 0.0164$.}
\end{exercise}

If the probability Brian kicks a 35 yard field goal is $p=0.8$, what is the probability it takes Brian exactly six tries to get his fourth successful kick? We can write this as
{\small\begin{align*}
&P(\text{it takes Brian six tries to make four field goals}) \\
& \quad = P(\text{Brian makes three of his first five field goals, and he makes the sixth one}) \\
& \quad = P(\text{$1^{st}$ sequence OR $2^{nd}$ sequence OR ... OR $10^{th}$ sequence})
\end{align*}
}where the sequences are from Table~\ref{successFailureOrdersForBriansFieldGoals}. We can break down this last probability into the sum of ten disjoint possibilities:
{\small\begin{align*}
&P(\text{$1^{st}$ sequence OR $2^{nd}$ sequence OR ... OR $10^{th}$ sequence}) \\
&\quad = P(\text{$1^{st}$ sequence}) + P(\text{$2^{nd}$ sequence}) + \cdots + P(\text{$10^{th}$ sequence})
\end{align*}
}The probability of the first sequence was identified in Guided Practice~\ref{probOfEachSeqOfSixTriesToGetFourSuccesses} as 0.0164, and each of the other sequences have the same probability. Since each of the ten sequence has the same probability, the total probability is ten times that of any individual sequence.

The way to compute this negative binomial probability is similar to how the binomial problems were solved in Section~\ref{binomialModel}. The probability is broken into two pieces:
\begin{align*}
&P(\text{it takes Brian six tries to make four field goals}) \\
&= [\text{Number of possible sequences}] \times P(\text{Single sequence})
\end{align*}
Each part is examined separately, then we multiply to get the final result.

We first identify the probability of a single sequence. One particular case is to first observe all the failures ($n-k$ of them) followed by the $k$ successes:
\begin{align*}
&P(\text{Single sequence}) \\
&= P(\text{$n-k$ failures and then $k$ successes}) \\
&= (1-p)^{n-k} p^{k}
\end{align*}

We must also identify the number of sequences for the general case. Above, ten sequences were identified where the fourth success came on the sixth attempt. These sequences were identified by fixing the last observation as a success and looking for all the ways to arrange the other observations. In other words, how many ways could we arrange $k-1$ successes in $n-1$ trials? This can be found using the $n$ choose $k$ coefficient but for $n-1$ and $k-1$ instead:
\begin{eqnarray*}
{n-1 \choose k-1} = \frac{(n-1)!}{(k-1)! \left((n-1) - (k-1)\right)!} = \frac{(n-1)!}{(k-1)! \left(n - k\right)!}
\end{eqnarray*}
This is the number of different ways we can order $k-1$ successes and $n-k$ failures in $n-1$ trials. If the factorial notation (the exclamation point) is unfamiliar, see page~\pageref{factorialDefinitionInTheBinomialSection}.

\begin{termBox}{\tBoxTitle{Negative binomial distribution}
The negative binomial distribution describes the probability of observing the $k^{th}$ success on the $n^{th}$ trial:
\begin{eqnarray}
P(\text{the $k^{th}$ success on the $n^{th}$ trial}) = {n-1 \choose k-1} p^{k}(1-p)^{n-k}
\label{negativeBinomialEquation}
\end{eqnarray}
where $p$ is the probability an individual trial is a success. All trials are assumed to be independent.}
\end{termBox}

\textC{\pagebreak}

\begin{example}{Show using Equation~\eqref{negativeBinomialEquation} that the probability Brian kicks his fourth successful field goal on the sixth attempt is 0.164.}
The probability of a single success is $p=0.8$, the number of successes is $k=4$, and the number of necessary attempts under this scenario is $n=6$.
\begin{align*}
{n-1 \choose k-1}p^k(1-p)^{n-k}\ 
	=\ \frac{5!}{3!2!} (0.8)^4 (0.2)^2\ 
	=\ 10\times 0.0164\ 
	=\ 0.164
\end{align*}
\end{example}

\begin{exercise}
The negative binomial distribution requires that each kick attempt by Brian is independent. Do you think it is reasonable to suggest that each of Brian's kick attempts are independent?\footnote{Answers may vary. We cannot conclusively say they are or are not independent. However, many statistical reviews of athletic performance suggests such attempts are very nearly independent.}
\end{exercise}

\begin{exercise}
Assume Brian's kick attempts are independent. What is the probability that Brian will kick his fourth field goal within 5 attempts?\footnote{If his fourth field goal ($k=4$) is within five attempts, it either took him four or five tries ($n=4$ or $n=5$). We have $p=0.8$ from earlier. Use Equation~\eqref{negativeBinomialEquation} to compute the probability of $n=4$ tries and $n=5$ tries, then add those probabilities together:
\begin{align*}
& P(n=4\text{ OR }n=5) = P(n=4) + P(n=5) \\
&\quad = {4-1 \choose 4-1} 0.8^4 + {5-1 \choose 4-1} (0.8)^4(1-0.8) = 1\times 0.41 + 4\times 0.082 = 0.41 + 0.33 = 0.74
\end{align*}}
\end{exercise}

\begin{tipBox}{\tipBoxTitle{Binomial versus negative binomial}
In the binomial case, we typically have a fixed number of trials and instead consider the number of successes. In the negative binomial case, we examine how many trials it takes to observe a fixed number of successes and require that the last observation be a success.}
\end{tipBox}

\begin{exercise}
On 70\% of days, a hospital admits at least one heart attack patient. On 30\% of the days, no heart attack patients are admitted. Identify each case below as a binomial or negative binomial case, and compute the probability.\footnote{In each part, $p=0.7$. (a) The number of days is fixed, so this is binomial. The parameters are $k=3$ and $n=7$: 0.097. (b) The last ``success'' (admitting a heart attack patient) is fixed to the last day, so we should apply the negative binomial distribution. The parameters are $k=2$, $n=4$: 0.132. (c) This problem is negative binomial with $k=1$ and $n=5$: 0.006. Note that the negative binomial case when $k=1$ is the same as using the geometric distribution.}

(a) What is the probability the hospital will admit a heart attack patient on exactly three days this week?

(b) What is the probability the second day with a heart attack patient will be the fourth day of the week?

(c) What is the probability the fifth day of next month will be the first day with a heart attack patient?
\index{distribution!negative binomial|)}
\end{exercise}


\textC{\pagebreak}


\subsection{Poisson distribution}
\label{poisson}

\index{distribution!Poisson|(}

\begin{example}{There are about 8 million individuals in New York City. How many individuals might we expect to be hospitalized for acute myocardial infarction (AMI), i.e. a heart attack, each day? According to historical records, the average number is about 4.4 individuals. However, we would also like to know the approximate distribution of counts. What would a histogram of the number of AMI occurrences each day look like if we recorded the daily counts over an entire year?} \label{amiIncidencesEachDayOver1YearInNYCExample}
A histogram of the number of occurrences of AMI on 365 days for NYC is shown in Figure~\ref{amiIncidencesOver100Days}.\footnote{These data are simulated. In practice, we should check for an association between successive days.} The sample mean (4.38) is similar to the historical average of 4.4. The sample standard deviation is about 2, and the histogram indicates that about 70\% of the data fall between 2.4 and 6.4. The distribution's shape is unimodal and skewed to the right.
\end{example}

\begin{figure}[h]
\centering
\includegraphics[width=0.7\textwidth]{ch_distributions/figures/amiIncidencesOver100Days/amiIncidencesOver100Days}
\caption{A histogram of the number of occurrences of AMI on 365 separate days in NYC.}
\label{amiIncidencesOver100Days}
\end{figure}

The \termsub{Poisson distribution}{distribution!Poisson} is often useful for estimating the number of events in a large population over a unit of time. For instance, consider each of the following events:
\begin{itemize}
\setlength{\itemsep}{0mm}
\item having a heart attack,
\item getting married, and
\item getting struck by lightning.
\end{itemize}
The Poisson distribution helps us describe the number of such events that will occur in a short unit of time for a fixed population if the individuals within the population are independent.

The histogram in Figure~\ref{amiIncidencesOver100Days} approximates a Poisson distribution with rate equal to 4.4. The \term{rate} for a Poisson distribution is the average number of occurrences in a mostly-fixed population per unit of time. In Example~\ref{amiIncidencesEachDayOver1YearInNYCExample}, the time unit is a day, the population is all New York City residents, and the historical rate is 4.4. The parameter in the Poisson distribution is the rate -- or how many events we expect to observe -- and it is typically denoted by $\lambda$\marginpar[\raggedright\vspace{-5mm}

$\lambda$\vspace{0mm}\\\footnotesize Rate for the\\Poisson dist.]{\raggedright\vspace{-5mm}

$\lambda$\vspace{0mm}\\\footnotesize Rate for the\\Poisson dist.}\index{Greek!lambda@lambda ($\lambda$)}
(the Greek letter \emph{lambda})  or $\mu$. Using the rate, we can describe the probability of observing exactly $k$ events in a single unit of time.

\begin{termBox}{\tBoxTitle{Poisson distribution}
Suppose we are watching for events and the number of observed events follows a Poisson distribution with rate $\lambda$. Then
\begin{align*}
P(\text{observe $k$ events}) = \frac{\lambda^{k} e^{-\lambda}}{k!}
\end{align*}
where $k$ may take a value 0, 1, 2, and so on, and $k!$ represents $k$-factorial, as described on page~\pageref{factorialDefinitionInTheBinomialSection}. The letter $e\approx2.718$ is the base of the natural logarithm. The mean and standard deviation of this distribution are $\lambda$ and $\sqrt{\lambda}$, respectively.}
\end{termBox}

We will leave a rigorous set of conditions for the Poisson distribution to a later course. However, we offer a few simple guidelines that can be used for an initial evaluation of whether the Poisson model would be appropriate.

\begin{tipBox}{\tipBoxTitle{Is it Poisson?}
A random variable may follow a Poisson distribution if we are looking for the number of events, the population that generates such events is large, and the events occur independently of each other.}
\end{tipBox}

Even when events are not really independent -- for instance, Saturdays and Sundays are especially popular for weddings -- a Poisson model may sometimes still be reasonable if we allow it to have a different rate for different times. In the wedding example, the rate would be modeled as higher on weekends than on weekdays. The idea of modeling rates for a Poisson distribution against a second variable such as \var{dayOfTheWeek} forms the foundation of some more advanced methods that fall in the realm of \termsub{generalized linear models}{generalized linear model}. In Chapters~\ref{linRegrForTwoVar} and~\ref{multipleAndLogisticRegression}, we will discuss a foundation of linear models.

\index{distribution!Poisson|)}


